\section{Funções Discretas}

Nesta seção, iremos dissertar um pouco mais com o tipo de funções com as quais
esteremos lidando ao longo do texto. Assim como na seção anterior, iremos
dissecar a ideia do que seria a propriedade de discretude aplicada a funções,
e, em seguida introduziremos alguns tipos de funções discretas que nos acompanharão.
Pontua-se aqui que, nos exemplos, estará implícito que as funções manipulam objetos
discretos.

\subsection{Discretude de Funções}

Como funções são um tipo de relação especial entre dois conjuntos, então
é razoável pensar que se tivermos estruturas discretas envolvidas numa relação,
então essa propriedade influenciará a própria estrutura da relação. Seja
a função $f: A \to B$, $f$ será uma função discreta caso $A$ tenha uma estrutura
discreta. Para defender que essa ideia faz sentido, podemos argumentar que:

\begin{enumerate}
  \item O conjunto imagem sempre terá cardinalidade menor ou igual a do domínio.
    Dessa forma, a propriedade de enumerabilidade é preservada para imagem se
    ela vale para o domínio.

  \item Se os objetos do domínio são separáveis, o fato de que cada elemento
    possui uma única imagem implicará que as imagens também poderão ser vistas
    a partir de uma estrutura que separa os seus pontos, sendo, pois, discreta.
\end{enumerate}

No geral, a forma como se trata uma função discreta é a mesma como se trata
uma contínua, com a diferença que a estrutura subjacente dos domínios envolvidos
é que lhes dará comportamentos distintos. Nas partes as seguir, iremos
observar alguns exemplos típicos de funções discretas.

\subsection{Multiconjuntos}

Multiconjuntos, apesar do nome, é um tipo de função que é utilizada como
artifício para criar estruturas em que há repetição de elementos. Formalmente,
chamaremos de multiconjuntos qualquer função na forma:
\[
  m: X \to \mathbb{N}
\]
em que a imagem $m(x)$ de $x \in X$ é chamada de multiplicidade. Um multiconjunto
trivial é aquele tal que
\[
  \forall x \in X,\ ( m(x) = 0 )
\]
e chamaremos de binário aqueles que satisfazem
\[
  \forall x \in X,\ ( m(x) = 0 \lor m(x) = 1 )
\]
que será usado para representar situações em que fazemos uma seleção sem repetição
dos elementos de $X$.

Dado um conjunto odenado $X = \{x_1, \ldots, x_n\}$ e um multiconjunto $m$,
denotaremos por $m(X)$ o conjunto ordenado $\{m_1, \ldots, m_n\}$ tal que
$m(x_i) = m_i$, para $i \in [n]$.

\subsection{Funções de Contagem}

Um dos problemas fundamentais da matemática combinatória é a contagem de objetos
membros de uma dado universo que satisfazem certas restrições. Muitas vezes,
é possível detectar um padrão nessas estruturas de forma a possibilitar
a abstração da contagem por meio de uma função que relaciona uma tupla de
parâmetros característicos da estrutura ao número de objetos que satisfazem as
restrições impostas. Chamaremos essas relações de funções de contagem, que
assumem a forma
\[
  f: \mathbb{N}^k \to \mathbb{N}
\]
onde $k$ é o número de parâmetros envolvidos na contagem.

As funções de contagem possuem um forte aspecto computacional embutido, pois
muitas vezes a sua lei de relação é uma especificação de um método que permite,
por meio de uma sequência finita de passos, determinar a contagem dos objetos.
Em outras palavras, essas funções especificam algoritmos.

\subsection{Fatoriais}

Um caso particular de função discreta que surge naturalmente em problemas
combinatórios são os fatoriais. Essas funções computam o produto de todos
os inteiros não negativos até um $n \in \mathbb{N}$. Visto isso, o fatorial
de um número $n$, denotado por $n!$, é recursivamente definindo como
\[
  n! = \begin{cases}
    1,       &\text{ se } n = 0\\
    n(n-1)!, &\text{ se } n > 0
  \end{cases}
\]

Os fatoriais tem uma propriedade interessante de que os valores se tornam
absurdamente grandes mesmo com variações graduais na entrada da função. A
calculadora do meu celular, por exemplo, só é capaz de computar o fatorial de
até 170, cujo resultado supera -- com grande folga -- o número de partículas
no universo.

\subsection{Sequências}

As sequências discretas são uma forma de representar uma sucessão de valores,
que pode ser finita ou infinita, que possuem um primeiro elemento.
Formalmente, diremos que $S$ é uma sequência se for uma função injetora
na forma $S: D \to A$ tal que $D$ satisfaz uma das opções a seguir:
\begin{enumerate}
  \item $D = [n]^*$, tal que $n \in \mathbb{N}$, caso em que a sequência é dita
    finita.

  \item $D = \mathbb{N}$, caso onde a sequência é dita infinita.
\end{enumerate}
Diremos que $S$ é uma sequência para um conjunto $A$ se ele for o contradomínio
da sequência.
Se $S: D \to A$, representaremos essa sequência enquanto uma tupla ordenada
$(a_1, \ldots, a_n)$ de $A^n$, com $n = |D|$ de modo que
\[
  a_i = a \Longleftrightarrow S(i) = a.
\]
Quando a sequência for infinita, usaremos as reticências, como em
$(1, 2, \ldots)$. Dessa maneira, conseguiremos tratar sequências e
tuplas de maneira intercambiável. Rotineiramente, iremos dizer
que uma sequência é igual a sua tupla correspondente.

Usaremos as sequências também para representar conjuntos cujos elementos estão
ordenados por idexados por naturais. Resumidamente, se $A$ é enumerável e bem ordenado
pela relação estrita linear $R$, com menor elemento $a_1$, i.e.,
\[
  \forall a \in A,\ ( a \cancel{R} a_1 )
\]
e $S$ for uma sequência para $A$ com $S(1) = a_1$, então representaremos
$A$ como:

\begin{enumerate}
  \item A é finito, então seja $a_n$ aquele que statisfaz
    \[
      \forall a \in A,\ ( a R a_n ),
    \]
    com $|A| = n$ e
    \[
      A = [a_1, \ldots, a_n] \Longleftrightarrow \forall i, j \in [n],\ (
        i < j \leftrightarrow S(i)\ R\ S(j)
      )
    \]
  \item Se A for infinito enumerável, então
    \[
      A = [a_1, \ldots] \Longleftrightarrow \forall i, j \in \mathbb{N},\ (
       i < j \leftrightarrow S(i)\ R\ S(j)
      )
    \]
\end{enumerate}
Omitindo a relação $R$ e a sequência $S$, iremos usar a representação
de um conjunto enquanto sequência para denotar uma situação em que
queremos tratar os elementos do conjunto em uma determinada ordem\footnote{
  Em outros textos é comum denotar os elementos de conjuntos ordenados
  estando entre parênteses, mas preferiremos os colchetes pela opinião
de que os primeiros já estão semanticamente sobrecarregados.}.


Ainda sobre notações, tomemos a sequência $S: D \to A$ e a função $f: A \to B$.
Construiremos a sequência $S_f: D \to \im{f}$ como aquela que satisfaz
\[
 \forall k \in D\,\forall a \in A,\ ( S(k) = a \leftrightarrow S_f(k) = f(x) )
\]
Com essa formalização, iremos admitir que, quando houver uma sequência $S$
para $A$, e $f$ for uma função com domínio em $A$, então podemos tomar
a tupla da sequência
\[
  (a_1, a_2, \ldots)
\]
e substituir os termos pelas suas respectivas imagens em $f$, obtendo a
representação em tupla da sequência $S_f$:
\[
  (f(a_1), f(a_2), \ldots).
\]
Também usaremos essa forma para representar uma função $f$ qualquer, intercalando
com as formas já mostradas em seções anteriores..

Existem três formas principais de definir uma sequência discreta:

\subsubsection{Relações de Recorrência}

Aplicando as definições recursivas vistas na \ref{sec:recursao},
as relações de recorrência, ou funções recursivamente definidas, definem sequências
em que os termos dependem seus anteriores na sequência. Um exemplo clássico
é a sequência de Fibonacci, em que cada termo é a soma dos outros dois anteriores.
\[
  S(n) = \begin{cases}
    1,               &\text{ se } n = 1\\
    1,               &\text{ se } n = 2 \\
    f(n-1) + f(n-2), &\text{ se } n > 2
  \end{cases}
\]

\subsubsection{Funções de Índice}

Ao contrário das definições recursivas, as sequências com função do índice são
aquelas em que cada termo pode ser dado por uma fórmula fechada apenas em termos
do índice da posição dele na sequência. Em outras palavras, a lei de formação
de $S$ é expressa numa fórmula apenas em termos de $n$.

\subsubsection{Propriedade dos Termos}

Alguns tipos de sequência, em vez de uma lei algébrica -- como nos casos
anteriores -- utilizam uma determinada propriedade para definir os seus termos.
Exemplos desse casos incluem: a sequência dos números primos ou a sequência
do número de divisores de cada natural.

\subsection{Princípio da Casa dos Pombos}

O princípio da casas dos pombos é um interessante, e ao mesmo tempo simples,
resultado sobre funções entre dois conjuntos finitos. Esse princípio carrega
aquele nome devido ao forma como ele é geralmente explicado; admita que tenhamos
$n$ pombos e queiramos construir casas para eles, porém somente dispomos de material
o suficiente para construir $m$ casas, tal que $n > m$. Consequentemente, conclui-se
que pelo menos dois pombos terão de compartilhar a mesma casa.

Na sua forma formal, o princípio estabelece que, se $A$ e $B$ são conjuntos
finitos, de modo que $|A| > |B|$, então toda função $f: A \to B$ não poderá
ser uma injeção. É verdade que, presumindo que $f$ seja uma injeção, então dois
casos podem ocorrer:
\begin{enumerate}
  \item Se $f$ for sobrejetiva, então ela é uma bijeção, e como vimos na seção
    de conjunto, $A$ e $B$ teriam de ter a mesma cardinalidade.

  \item Se $f$ não é sobrejetiva, então há pelo menos um elemento de $B$ que
    não se relaciona com nenhum de $A$. Como $|A| = |\im{f}|$ pela injetividade,
    e $|\im{f}| \leq B$, então $|A| \leq |B|$.
\end{enumerate}

O princípio, apesar de simples, é suficiente poderoso para demonstrar alguns
teoremas interessantes, como o resultado de que num grupo qualquer de pessoas,
em que alguém pode ou não ser amigo de outros, sempre haverá duas pessoas com
o mesmo número de amigos.
