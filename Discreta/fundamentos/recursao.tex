\section{Recursão e Indução}
\label{sec:recursao}

Nesta seção, discutiremos em alto nível uma das ferramentas mais úteis para
solução de problemas matemáticos: o Princípio de Indução Finita. Este
princípio permite demonstrar que, quando determinados estruturas podem
ter seus elementos definidos a partir de outros mais simples, e provarmos
que certa propriedade vale para esses últimos, então a propriedade vale
também para elementos mais complexos. Antes disso, explicaremos a ideia de
recursão e a usaremos para construir intuitivamente o Princípio de Indução.

\subsection{Recursão}

A recursão é uma forma de definição aplicada a objetos que podem ser descritos
em termos de outros objetos do mesmo tipo, porém mais simples. Em geral, uma
definição recursiva é composta por casos base, que correspondem aos objetos mais
simples e não dependem de outras definições, e por casos recursivos, nos quais
um objeto é definido a partir de instâncias menores ou mais simples de si mesmo.

Uma definição recursiva é dita bem definida quando todo objeto que não é um caso
base pode ser decomposto, em um número finito de passos, até alcançar um dos
casos base, garantindo assim que o processo de definição termine.

De modo mais formal, um conjunto admite uma definição recursiva quando
seus elementos podem ser descritos a partir de casos bases e de regras de
construção que utilizam objetos previamente definidos. Os casos base são
constituem os elementos mínimos da definição, enquanto que as regras recursivas
garantem que todo objeto, pode ser reduzido, num número finito de aplicações, a esses
casos.

As definições recursivas são, sem dúvida, uma das ferramentas mais
úteis para o estudo de objetos discretos pelo fato de muitos deles terem a
propriedade de serem definidos a partir de objetos discretos menores. Com isso,
nos aproveitaremos dessa forma de definição para criar descrições elegantes
sobre muitas estruturas.

\subsection{Princípio de Indução Finita}

As definições recursivas permitem a demonstração de resultados
a partir do poderoso Princípio de Indução Matemática. A demonstração por
indução, geralmente aplicada em situações em que deseja-se provar que
uma propriedade é satisfeita por todos os elementos de um conjunto, envolve
dois passos principais: provar que a propriedade é verdadeira para os casos
base e, supondo que a propriedade vale para um caso recursivo, provar
que, ao obter um novo caso pelas regras de construção da definição recursiva,
a propriedade valerá para esse novo caso.

Em resumo, a indução consiste em demonstrar que, se a propriedade vale para os
casos base e que ela se mantém verdadeira sempre que aplicamos a regra de
construção recursiva para um novo caso, então ela é verdadeira para todos os
casos.
