\section{Introdução}

\subsection{O que é Matmética Discreta}

A definição de Matmática Discreta é difícil de se obter, sendo comum dizer
que é mais fácil a definir pelo que está fora dela do que de
fato está incluso nela. Ela estuda uma coleção ampla e heterogênea de objetos,
e que dialoga com diversas outras áreas, muito pelo fato dela lidar, entre
outras coisas, com princípios fundamentais da matemática, como a contagem.

Numa tentiva de definir o conceito, podemos dizer que a Matemática Discreta não é um
campo da matemática, mas sim a parte dela cujo foco está nas estruturas algébricas
nas quais não há continuidade entre seus elementos.
Seus objetos de estudo são estruturas, conjuntos e funções que lidam
com elementos distintos e separados, tais como números inteiros, grafos
e fórmulas lógicas.

\subsection{Áreas da Matemática Discreta}

Neste texto, iremos explorar alguns campos que são estudados, primariamente,
sob luz da Matmética Discreta. Todos os campos compartilham a propriedade
fundamental de estarem lidando com objetos como conjunto e funções discretas,
implicando que eles podem ser distinguidos um dos outros, enumerados, ordenados
e contados.

\subsubsection{Combinatória}

A combinatória, também chamada de Análise Combinatória em contextos mais
elementares, estuda modos de contar, selecionar, combinar ou ordenar elementos de um
conjunto finito sem necessariamente listar todas as possibilidades. Por
sua vez, a combinatória pode ser dividida em alguns campos mais especializados,
como:

\begin{itemize}
  \item Enumerativa - busca contar o números de elementos de um conjunto
    que satisfazem uma determinada propriedade. É a área mais elementar da
    combinatória, sendo também chamada de contagem.

  \item Extremal - busca determinar o quão grande ou pequena uma coleção
    de elementos pode ser caso tenha que satisfazer determinadas restrições.
    Também estuda como selecionar objetos que satisfaçam uma condição de
    extremalidade ou otimiladade.
    Surge em problemas combinatórios relacionados a otimização.

  \item Algébrica - um campo que visa empregar métodos da Álgebra Abstrata,
    especialmente da Teoria de Grupos e da Representação, em contextos
    combinatórios, enquanto que, ao mesmo tempo, emprega métodos combinatórios
    para manipular objetos de uma estrutura algébrica.

  \item Probabilística -  um campo com maior grau de especialização introduzido
    por Paul Erdös que estuda o emprego de métodos probablísticos para
    demonstrar a existência de um determinado objeto combinatório. Em essência,
    busca-se provar que, ao selecionar aleatoriamente um objeto em dado
    universo, a probabilidade de que o objeto escolhido satisfaça uma propriedade
    desejada é estritamente maior do que zero.
\end{itemize}

\subsubsection{Relações de Recorrência}

As relações de recorrência são fórmulas que definem termos de uma sequência
em função de termos anteriores. A relação de recorrência mais famosa é com
certeza a Sequência de Fibonacci; nela, o termo base é $a_0 = 0$, e a fórmula
de recorrência é dada por $a_n = a_{n-1} + a_{n-2}$, ou seja, um termos é
definido como a soma dos outros dois imediatamente anteriores. Também é
comum denotar recorrências como funções dos índices dos termos da sequeência,
que no mesmo exemplo dado seria $f(0) = 0$ e $f(n) = f(n-1) + f(n-2)$.

As recorrências surgem naturalmente em diversos problemas que envolvem objetos
discretos, de modo que a solução do problema torna-se resolver a recorrência,
isto é, achar uma fórmula não recursiva -- também chamada de fórmula fechada --
em termos do próprio índice para obter qualquer termo da sequência, sem precisar
computar os anteriores na ordem.

\subsubsection{Teoria dos Grafos}

A Teoria dos Grafos é um grande ramo da Matemática Discreta que lida
com conjuntos de objetos que estão relacionados entre si. Os objetos de um
grafo são representados por pontos, enquanto que as relações entre eles são
denotados por setas ou segmentos de retas, a depender do tipo da relação sendo
representada.

O estudo dos grafos gera ferramentas poderosas para a modelagem de diversos
problemas, que passam a poder ser solucionados por algoritmos que resolvem
questões como conectividade, caminhos mínimos, fluxos, emparelhamentos e
colorações. Dessa forma, problemas oriundos de áreas como ciência da
computação, engenharia, logística, biologia, redes sociais e economia podem
ser formalizados de maneira precisa e analisados sistematicamente por meio
dessas estruturas.

\subsection{Notação}

A seguir, serão apresentados, em grande parte, os símbolos e seus respectivos
significados que virão ser usados no texto. Muitas delas terão seu significado
melhor explicado quando for oportuno.

\begin{multicols}{2}
\noindent
$\mathbb{N}$ -- Conjunto dos Naturais, incluindo o $0$.

\noindent
$\mathbb{Z}$ -- Conjunto dos Inteiros.

\noindent
$\mathbb{R}$ -- Conjunto dos Reais.

\noindent
$\{x \in U \mid P(x)\}$ -- Conjunto dos elementos em $U$ que satisfazem a
propriedade $P(x)$.

\noindent
$[n]$ -- conjunto dos naturais menores ou iguais a $n$.

\noindent
$[n]^*$ -- conjunto dos naturais não nulos menores ou iguais a $n$.

\noindent
$\parts{A}$ -- conjunto das partes de $A$.

\noindent
$[x]_R$ -- classe de equivalência $x$ na relação $R$.

\noindent
$\lor$ -- operador lógico de conjunção (ou).

\noindent
$\land$ -- operador lógico de disjunção (e).

\noindent
$p \rightarrow q$ -- se $p$, então $q$.

\noindent
$p \Longrightarrow q$ -- $p$ implica $q$.

\noindent
$p \leftrightarrow q$ -- $p$ se, e somente se, $q$.

\noindent
$p \Longleftrightarrow q$ -- $p$ equivale a $q$.

\noindent
$\forall$ -- para todo.

\noindent
$\exists$ -- existe.

\noindent
$!\exists$ -- existe um único.

\noindent
$(a_1, \ldots, a_n)$ -- tupla ou sequência finita de $n$ elementos.

\noindent
$(a_1, a_2, \ldots)$ -- sequência infinita.

\end{multicols}
