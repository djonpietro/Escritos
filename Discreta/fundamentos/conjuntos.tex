\section{Teoria de Conjuntos}

Nesta seção, abordaremos alguns conceitos principais da Teoria de Conjuntos,
mais precisamente da sua versão axiomática dada pela teoria de Zermelo-Fraenkel
acrescido do Axioma da Escolha (ZFC). Não nos prenderemos muitos aos detalhes
e demonstrações formais dos resultados da teoria, pois nossa intenção é
apresentar as ideias que estarão permeando os capítulos futuros do texto.

Passaremos rapidamente pelos Axiomas de ZFC, apresentando-os conforme a
necessidade de abordar uma forma especial de definir conjuntos, operações e
outros objetos de interesse. Em seguida, apresentamos os conceitos de relação
e função. Adiante, discutiremos o tópico dos números ordinais e cardinais,
que formalizam os conceitos fundamentais de ordenação e quantidade tratados
em Matemática Discreta. Por fim, daremos uma definição satisfatória para um
dos principais objetos manipulados neste material: os conjuntos discretos.

\subsection{Definindo Conjuntos}

A ZFC é uma teoria de primeira ordem, o que significa que suas fórmulas
\footnote{Fórmulas são qualquer sequência de símbolos que possuem um significado
	em uma teoria} lógicas quantificam e afirmam a respeito de objetos individuais,
chamados de conjuntos. Conjuntos é uma ideia primitiva que representa uma
coleção de objetos que podem também ser partes de outras coleções.

Entre conjuntos, persiste uma relação binária fundamental chamada de pertinência,
denotada por $\in$. Desse modo, se $a \in A$, então dizemos que $a$ é elemento
-- ou membro -- de $A$, ou que $a$ pertence a $A$. A igualdade de conjuntos é
definida pelo \textbf{Axioma da Extensão}, que determina que dois conjuntos são iguais
quando possuem os mesmos elementos, não importando a ordem ou as multiplicidades
deles.
\[
	A = B \Longleftrightarrow \forall x\ (x \in A \leftrightarrow x \in B)
\]
Por essa razão, diz-se que a identidade de um conjunto é definida inteiramente
por seus elementos. Em especial, o conjunto $x$ que satisfaz a fórmula
\[
	\forall y,\ ( y \notin x )
\]
é chamado de conjunto vazio, sendo denotado por $\varnothing$. Do Axioma da
Extensão, segue que esse conjunto é único.

A partir da relação fundamental de pertinência, podemos definir outra chamada
de ``inclusão'', denotada por $\subset$. Dizemos que $A$ inclui um conjunto $B$
caso todo elemento de $B$ seja um elemento de $A$. Da mesma forma, podemos
dizer que $B$ é incluído por $A$, o que $B$ é subconjunto de $A$.
\[
	B \subset A  \Longleftrightarrow \forall x\ (x \in B \rightarrow x \in A)
\]
Se a relação vale nas duas direções, então $A$ e $B$ são iguais pelo Axioma da
Extensão. A definição de inclusão implica também que o vazio e o próprio $A$
são subconjuntos de $A$, de modo que subconjuntos diferentes desses últimos
são chamados de subconjuntos próprios.

O \textbf{Axioma da Separação} nos oferece uma forma conveniente de definir subconjuntos
de um conjunto por meio de uma propriedade. Se $A$ é um conjunto, então podemos
definir um subconjunto $S \subseteq A$  cujos elementos satisfazem uma propriedade
$P(x)$. Denotaremos essa maneira de definir conjuntos como
\[
	S = \{x \in A \mid P(x)\}
\]
Perceba que, na verdade, esse axioma trata-se de um esquema, pois ele é
enunciado para cada uma das infinitas propriedades $P(x)$\footnote{Como
	mencionado, a ZFC é uma teoria de primeira ordem, não sendo capaz de
	enunciar fórmulas que quantifiquem outras fórmulas. Com isso,
	cada propriedade $P(x)$ necessita do seu próprio axioma.}

Para facilitar representação de conjuntos, muitas vezes iremos construir o
conjunto $S$ omitindo o superconjunto $A$. Logo, dada a propriedade $P(x)$,
diremos simplesmente que $S$ será o conjunto de todos os indivíduos que
satisfazem ela.
\[
	S = \{x \mid P(x)\}
\]
Esse modo de definição é chamado de ``por abstração''.

\subsection{Operações com Conjuntos}

Entre conjuntos, podemos definir algumas operações básicas usando os axiomas
anteriores. Com o Axioma da Separação, definimos a interseção $I$ de conjuntos
$A$ e $B$, que possui membros em comum desses dois últimos
\[
	I = \{x \in A \mid x \in A \land x \in B\}
\]
Representaremos a interseção dos membros de um conjunto $\mathcal{F}$ como
\[
	\bigcap_{A \in \mathcal{F}} A
\]
Se a interseção de $A$ e $B$ for vazia, então dizemos que são disjuntos. Quando
os elementos de um conjunto $\mathcal{F}$ forem todos disjuntos entre si, diremos
que $\mathcal{F}$ é disjunto.

Com o Axioma da Separação também definimos a operação de diferença entre $A$
e $B$, que resulta no conjunto de todos os membros do primeiro que não são
membros do segundo.
\[
	D = A \setminus B \Longleftrightarrow D = \{x \in A \mid x \notin B\}
\]

Por fim, a operação de união entre $A$ e $B$ resulta no conjunto que possui
tantos os membros de $A$ quanto de $B$ como elementos. No entanto,
diferente das duas operações anteriores, o resultado não é subconjunto de
um dos operandos, o que impede a utilização do Axioma da Separação e torna
necessário a introdução de um novo axioma que garanta a existência da união.
O \textbf{Axioma da União} enuncia que se $\mathcal{F}$ é um conjunto, então existe um
conjunto $U$ que reúne os membros dos membros de $\mathcal{F}$. Representaremos
esse conjunto como
\[
	U = \bigcup_{A \in \mathcal{F}} A
\]
Em particular, se $\mathcal{F} = {A, B}$, é comum escrever simplesmente que
$U = A \cup B$.
\[
	U = A \cup B \Longleftrightarrow \forall x,\ (x \in U \leftrightarrow x \in A \lor x \in B)
\]

% Caso seja necessário, aqui pode serintroduzida a diferença simétrica
%-------

\subsection{Conjuntos das Partes e Partições}

O \textbf{Axioma da Potência} garante, para todo conjunto $A$, a existência
do conjunto das partes de $A$, denotado por $\parts{A}$, que reúne todos os
subconjuntos de $A$ como membros.
\[
	\forall A \exists \mathcal{P} \forall S,\ (S \subset A \rightarrow S \in \mathcal{P})
\]
Usaremos a notação $\partsnn{A}$ para denotar $\parts{A} \setminus \varnothing$.

No conjunto das partes, há subconjuntos especiais que são chamados de partições.
Se $P \subset \partsnn{A}$ é um conjunto que satisfaz
\[
	\forall x, y \in P ( x \cap y = \varnothing ) \quad \text{e} \quad A = \bigcup_{x \in P} x
\]
chamamos $P$ de uma partição para o conjunto $A$.

\subsection{Produto Cartesiano}

O \textbf{Axioma do Par} enuncia que, dados conjuntos $A$ e $B$, podemos construir um
novo conjunto $C = \{A, B\}$. Esse axioma, em conjunto dos anteriores, torna
a definição de Kuratowski para pares ordenados -- uma coleção de dois objetos em
que a ordem importa -- satisfatória:
\[
	(a, b) = \{a, \{a, b\}\},
\]
pois dela pode ser demonstrado que
\[
	(a, b) = (c, d) \Longleftrightarrow a = c \land b = d
\]

Exposto isso, definiremos a operação de produto cartesiano de dois conjuntos $A$
e $B$, denotada por $A \times B$, como o conjunto de todos os
pares ordenados $(a, b)$ tal que que $a \in A$ e $b \in B$. É possível mostrar
que esse conjunto existe agrumentando que
\[
	A \times B \subset \parts{\parts{A \cup B}}
\]

\subsection{Outros Axiomas de ZFC}

Dando sequência, aparesentaremos os demais axiomas da ZFC. O Axioma da
Regularidade afirma que todo conjunto não vazio é disjunto com pelo menos
um de seus elementos. Esse axioma impede a construção de uma série
conjuntos que introduziriam paradoxos na Teoria de Conjuntos, a exemplos
de
\[
	A = \{A\} \quad A = \{A, \varnothing\}
\]
ou tembém situações como $A \in B$ e $B \in A$.

Chamamos de sucessor de um conjunto $A$ aquele definido por $S(A) = A \cup \{A\}$.
Com isso, definiremos o conjunto $I$ que satisfaz
\[
	\varnothing \in I \land \forall A (A \in I \rightarrow S(A) \in I ).
\]
O conjunto $I$ é chamado de indutivo, e a existência de pelo menos um conjunto
indutivo é enunciada pelo Axioma do Infinito. A motivação para ele é garantir
não só a existência de um conjunto infinito, mas também permitir a definição dos
números naturais. Cada natural poderia, individualmente, ser definido como
\begin{gather*}
	0 = \varnothing \\
	1 = S(0)   = \{\varnothing\} \\
	2 = S(S(0)) = \{\varnothing, \{\varnothing\}\} \\
	\vdots
\end{gather*}
No entanto, ainda não teríamos garantido a existência de um conjunto que
contenha todos os naturais definidos daquela maneira. Pela definição de
conjunto indutivos, o vazio e seus sucessores precisam pertencer a qualquer
conjunto deste tipo; em outras palavras, todos os conjuntos indutivos incluem
os números naturais. Dessa forma, o conjunto dos naturais é definido como a
interseção de todos os conjuntos indutivos.
\[
	\mathbb{N} = \{x \in I \mid \forall J,\ (J \text{ é indutivo} \rightarrow x \in J \}
\]

O próximo é o Axioma da Substituição, o qual afirma que, em qualquer conjunto
$A$, se para todo elemento $x \in A$ existe um único $y$, denominado imagem,
que satisfaça uma relação $\phi$, então existe um conjunto $B$ que reúne
todos os $y$ para os quais existe $x \in A$ que satisfaça $\phi$.
Em resumo, o axioma garante a existência de um conjunto que reúne as
imagens dos elementos de $A$ sobre a relação $\phi$. Esse axioma está
intimamente ligado ao conceito de funções e o aplicaremos mais adiante.

Por fim, temos o Axioma da Escolha. Ele afirma que, dado um conjunto $\mathcal{F}$
com membros não vazios, existe um conjunto $C$ tal que, para todo $A \in \mathcal{F}$,
existe um único $(A,a) \in C$ com $a \in A$. O conjunto $C$ é chamado de
função de escolha, que toma um conjunto $A \in \mathcal{F}$ e retorna um
elemento $a \in A$.


\subsection{Relações}

Chamaremos $R$ uma relação entre $A$ e $B$ se $R$ for um conjunto em $\parts{A \times B}$, e
se $(a, b) \in R$, então denotaremos esse fato por $aRb$. O conjunto $\dm{R} \subset A$,
chamado de domínio, é o conjunto de todos os elementos $a \in A$ tal que existe
um $b \in B$ e $(a, b) \in R$. Já o conjunto $B$ é chamado de contradomínio da relação,
e o seu subconjunto $\mathrm{Im}$ que contém todos os $b$ para os quais há
um $a \in A$ e $(a, b) \in R$ é chamado de imagem da relação.
Geralmente, definimos uma relação por meio de uma dada fórmula $\varphi(a, b)$
envolvendo símbolos para elementos daqueles conjuntos, possibilitando definir
a relação tal como
\[
	R = \{(a, b) \in A \times B \mid a \in A \land b \in B \land \varphi(a, b)\}
\]
A fórmula $\varphi(a, b)$ é chamada de lei de correspondência.


O conceito de relações também podem ser extendido para além das binárias, que
ocorrem entre dois conjuntos. Intuitivamente, o conceito de aridade duma relação
está ligado a quantidade de conjuntos envolvidos nela, ou também com o tamanho
das ordenações de elementos que pertencem a ela. Exemplo, uma relação ternária $R$
entre conjuntos $A$, $B$ e $C$ terá como domínio $A \times B$ e contradomíno $C$,
logo
\[
	R \subset (A \times B) \times C
\]
Um exemplo de elemento em $R$ seria $((a, b), c)$, mas iremos facilitar a notação
impondo que
\[
	((a, b), c) = (a, b, c)
\]
que é uma tripla ordenada. Em geral, uma relação de aridade $n$ será um subconjunto
no produto cartesiano de $n$ conjuntos e cujos elementos serão tuplas ordenadas
(ou tuplas somente).\footnote{Fundamentalmente, nossas definições permitem apenas
	construírmos relações binárias, já que o produto cartesiano é uma operação
	binária por definição. Todavia, isso trata-se apenas de um detalhe de formalização,
	e, neste texto e em outros, adotaremos as tuplas para denotar elementos que
	estão relacionados entre si numa relação que envolve mais de dois conjuntos.}


Há algumas relações que valem a pena serem destacadas para menções futuras.
A primeira delas será a relação de ordem estrita linear, que, em essência,
define uma maneira de comparar elementos num conjunto. Formalmente,
uma relação $R$ é uma ordem estrita linear para o conjunto $A$ se satisfaz três
propriedades\footnote{O conjunto $A$ é também o contradomíno da relação}:
\begin{enumerate}
	\item Assimetria - a relação não vale em duas direções.
	      \[
		      \forall x, y \in A ( x R y \rightarrow y \cancel{R} x )
	      \]
	\item Transitividade - numa cadeia de elementos relacionados, a relação
	      vale para os elementos nas pontas da cadeia.
	      \[
		      \forall x, y, z \in A ( x R y \land y R z \rightarrow x R z)
	      \]
	\item Totalidade - todos os elementos estão relacionados entre si em ao
	      menos uma direção.
	      \[
		      \forall x, y \in A ( x \neq y \rightarrow x R y \lor y R x)
	      \]
\end{enumerate}
A relação $R$ será uma boa ordem para $A$ se for uma ordem estrita linear, e todo
subconjunto $B \subset A$ possui um menor elemento $x$, isto é,
\[
	\forall y \in B ( x R y)
\]
O conjunto $A$ será bem ordenado se admite alguma boa ordem.

A segunda relação a ser mencionada é a de equivalência, que define um tipo
especial de partição. Uma relação $R$ é uma relação de equivalência num
conjunto $A$ se satisfaz:
\begin{enumerate}
	\item Transitividade - já apresentada na relação de ordem.

	\item Reflexividade - todo elemento está relacionado consigo mesmo.
	      \[
		      \forall x \in A ( x R x )
	      \]
	\item Simestria - a relação é bidirecional.
	      \[
		      \forall x, y \in A ( x R y \rightarrow y R x )
	      \]
\end{enumerate}
Pelo Axioma da Separação, podemos formar um conjunto $[x]_R \subset A$, com $x \in A$
e $\varphi(y) = x R y$ tal que
\[
	[x]_R = \{y \in A \mid \varphi(y) \}
\]
Considere então os conjuntos $[x]_R$ e $[z]_R$. Se tivermos que $x R z$,
segue que $[x]_R = [z]_R$, no contrário, $[x]_R \cap [z]_R = \varnothing$. O
conjunto $[a]_R$ é chamado de classe de equivalência de $A$ sobre $R$, e
usando os axiomas do Par e da União, podemos construir o conjunto das
classes de equivalência de $A$ sobre $R$, denotado por $A / R$, que é uma
partição de $A$.

\subsection{Funções}

Um tipo especial de relação são as famosas funções. Dados conjuntos
$A$ e $B$, a relação $f$ é chamada de função de $A$ em $B$ se:
\begin{enumerate}
	\item $A = \dm{f}$

	\item Se $(a, b) \in f$ e $(a, c) \in f$, então $b = c$, para todo $a$,
	      $b$ e $c$.
\end{enumerate}
Em resumo, uma função relaciona todo elemento do domínio $A$ a um único
elemento do contradomínio $B$. Graças ao Axioma da Substituição, nem sempre
é necessário explicitar o conjunto $B$, pois se temos o domínio $A$, uma fórmula
$\varphi(x, y)$ e sabemos que, para todo $x \in A$, existe um único $y$, então
já vimos que o axioma garantirá a existência de um conjunto imagem $\mathrm{Im}$ para a relação
definida por $\varphi(x, y)$. Desse modo, a função $f$ poderia ser definida com
contradomínio no próprio $\mathrm{Im}$ ou qualquer conjunto que o contenha.

A notação para representar uma função $f$ com domínio $A$ no contradomínio
$B$ é
\begin{gather*}
	f: A \to B \\
	x \mapsto f(x)
\end{gather*}
em que $f(x)$ é o elemento $y \in B$ que é imagem de $x \in A$. Denotaremos
$\im{f}$ como o conjunto imagem da função $f$.

Uma função pode ser classificada de três maneiras
\begin{itemize}
	\item Injetiva - se um elemento no contradomínio é imagem de outro
	      no domínio, então ele é imagem apenas desse último.
	      \[
		      \forall y \in B,\ (y \in \im{f} \rightarrow \exists! x \in A,\ (f(x) = y))
	      \]
	\item Sobrejetiva - contradomínio e o cojunto imagem são iguais
	      \[
		      B = \im{f}
	      \]
	\item Bijetiva - a função é injetiva e sobrejetiva, construindo uma
	      correspondência de 1 para 1 entre os elementos no domínio e contradomínio
	      da função.
\end{itemize}

\subsection{Números Ordinais e Cardinais}

Conjuntos transitivos são aqueles que possuem todos os elementos de
seus elementos.
\[
	\forall x \forall y (x \in y \land y \in \alpha \rightarrow x \in \alpha)
\]
Por sua vez, um ordinal $\alpha$ é um conjuntos que satisfaz
\begin{enumerate}
	\item $\alpha$ e seus elementos são conjuntos transitivos

	\item A relação $\in$ é uma boa ordem para $\alpha$
\end{enumerate}

Da definição segue que\footnote{As afirmações listadas não são óbvias, mas
	as suas demonstrações em termos formais fugiria do escopo proposto para o texto}:
\begin{enumerate}
	\item[I] - $\varnothing$ é um ordinal por vacuidade, e os elementos de um
	      ordinal são ordinais.

	\item[II] - Todo ordinal é o conjunto de seus predecessores.

	\item[III] - O suscessor de um ordinal é um ordinal.
\end{enumerate}

Em especial, os elementos do conjunto dos naturais $\mathbb{N}$ são ordinais, ou
seja, os números 1, 2, 3, \dots, são ordinais (finitos).
Além disso, o ordinal indentificado pelo próprio $\mathbb{N}$ é denotado por
$\omega$, sendo o primeiro ordinal transfinito.

Apresentados os ordinais, diremos que duas boas ordens $(A, R)$ e $(B, S)$
são isomorfas se existe uma bijeção $f: A \to B$ tal que
\[
	\forall x, y \in A ( x R y \leftrightarrow f(x) S f(y) )
\]
Neste ponto é que a principal propriedade dos ordinais surge: todo conjunto
bem ordenado é isomorfo a um, e somente um, ordinal. Isso implica que, se
$(A, R)$ é uma boa ordenação isomorfa a $(\alpha, \in)$, onde $\alpha$ é um
ordinal, então podemos rotular cada elemento de $A$ por um único elemento de
$\alpha$. Em outras palavras, os elementos de $A$ podem ser indexados por
ordinais, de acordo com sua posição na ordem. Ademais, como $\alpha$ é único
para $(A, R)$, então ele é uma representação canônica da estrutura daquela boa
ordem.

Por fim, introduziremos a ideia dos cardinais, que, diferente dos ordinais,
não indetifica a estrutura do conjunto, mas o seu tamanho.

Antes de tudo, diremos que dois conjuntos são equipotentes se existe uma bijeção
entre eles. Do Axioma da Escolha, segue o interessante Teorema da
Boa Ordenação, o qual afirma que todo conjunto pode ser bem ordenado. Portanto,
pelo que vimos dos ordinais, para todo conjunto $A$, existe um ordinal $\alpha$
equipotente a ele. Chamaremos de cardinal -- ou cardinalidade -- de um conjunto
$A$, denotado por $|A|$, o menor ordinal equipotente a $A$.

Um conjunto será finito se o menor ordinal equipotente a ele for finito, isto é,
um elemento de $\omega$. Dessa forma, a cardinalidade de um conjunto finito
é igual a algum número natural. O cardinal $\aleph_0$ é representado pelo ordinal
$\omega$, sendo, pois, a cardinalidade dos naturais. Um conjunto $A$ é infinito
se possui cardinalidade maior ou igual a $\aleph_0$\footnote{Existe outras
	definições alternativas para um conjunto
	infinito: (a) se existe uma injeção $f: A \to A$ não sobrejetiva, ou (b) se
	existe uma injeção $f: \mathbb{N} \to A$.}.

\subsection{Estruturas Discretas}

Um conjunto é contável, ou enumerável, se sua cardinalidade é menor ou igual aos
dos números naturais. Isso significa que é possível listar todos os seus
elementos e indexá-los utilizando somente os naturais. Uma condição necessária
e suficiente para que dois conjuntos tenham a mesma cardinalidade é que exista
alguma bijeção entre eles. Dado isso, ser enumerável equivale a ter uma bijeção
com algum subconjunto dos naturais\footnote{Georg Cantor provou que os naturais,
	os inteiros e os racionais possuem a mesma cardinalidade, logo são todos
	enumeráveis. No caso de $\mathbb{R}$, ele mostrou que é impossível criar uma lista
	que contenha todos os números reais, provando que a cardinalidade desse último
	é maior do que de $\mathbb{N}$.}.

A ideia de estruturas discretas é intuitivamente simples, porém, formalmente,
sua natureza é difícil de especificar. Intuitivamente, estruturas discretas são
conjuntos finitos ou enumeráveis tais que todos os elementos distintos estão bem
separados. A forma padrão de formalizar ``separação'' de elementos em um conjunto
é por meio de topologias, que não iremos entrar em muitos detalhes por fugir
do escopo do que a matemática discreta, especialmente a combinatória, necessitam.

A propriedade de discretude não é dada como uma característica de um conjunto,
mas como parte de uma estrutura que, neste caso, consiga isolar os elementos
um dos outros.
% dá para falar um pouco mais sobre elas aqui, mas fica para depois.

% -------------- Futuro Apêndice de Formalização de Discretude
Uma topologia $\tau \subset \parts{X}$ é uma coleção de partes de um conjunto $X$ cujos
elementos, denominados abertos, satisfazem:
\begin{enumerate}
	\item $\varnothing$ e $X$ são membros de $\tau$.
	\item A interseção de um número finito de abertos é um aberto
	\item A união de um número potencialmente infinito de abertos é um aberto.
\end{enumerate}
O par $(X, \tau)$ é chamado de espaço topológico.
A ideia por detrás dos abertos é a de vizinhança de um ponto, composto por
elementos que estão arbitrariamente próximos desse último. Uma vizinhança $V$ de
um ponto $x \in X$ é um subconjunto de $X$ tal que exista um aberto $A$ com $ x \in A \subset V$.

Para representar a ideia de separção de um ponto, usamos as suas vizinhanças,
como se essas fossem as identidades do ponto na topologia especificada, como
na famosa frase ``digamas com quem tu andas que direi que tu és''. Dessa forma,
um ponto isolado seria aquele que não necessita de ninguém além dele mesmo para
ser identificado. Seja $(X, \tau)$ um espaço topológico, dizemos que $x \in X$
é um ponto isolado no subconjunto $S \subset X$, com $x \in S$, se há
vizinhança $V$ para $x$ tal que
\[
	V \cap S = \{x\}
\]
Isso equivale a dizer que, no subespaço topológico $(S, \tau_S)$, tal que
\[
	\tau_S = \{ U \cap S \mid U \in \tau\}
\]
temos que $\{x\}$ é um aberto de $\tau_S$. Se for satisfeito que
\[
	\tau = \parts{X}
\]
então o espaço topológico $(X, \tau)$ é discreto, pois
\[
	\forall x \in X, \{x\} \in \tau
\]
e todo ponto é uma vizinhança de si mesmo, ou que todo ponto é isolado no próprio
$X$.
