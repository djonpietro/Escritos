\section{Argumentos Combinatórios}

Para alguns problemas, uma abordagem simplesmente algébrica, que consistiria
na manipulação de termos em uma fórmula, pode ser difícil, cansativo e até
mesmo deselagante. Essa situação ocorre frequentemente na matemática discreta,
em que, muitas vezes, daremos prioridade aos argumentos combinatórios, que
tentam atacar um problema dando mais atenção a estrutura dos objetos envolvidos,
como sua quantidade, ordenação e formas de combiná-los para obter outros elementos,
do que a fórmulas algébricas.

\subsection{Contagem Duplas}

A contagem dupla é dos argumentos combinatórios mais simples e comuns de serem
observados em demonstrações. Ela é geralmente usada para provar a validade
de uma identidade algébrica sobre alguma estrutura discreta, mostrando a igualdade
dos dois membros ao argumentar que eles contam as mesmas coisas naquela estrutura.

\subsection{Argumento da Bijeção}

Já apresentado na seção de conjuntos, o Argumento da Bijeção visa construir uma
bijeção entre dois conjuntos a fim de mostrar que ambos possuem a mesma quantidade
de elementos. Dessa forma, se formos capazes de mostrar a equivalência entre pares
de objetos de dois conjuntos, então teremos mostrado que eles possuem a mesma
cardinalidade.

\subsection{Argumento Extremal}

Esse argumento usa a estratégia da prova por contradição para demonstrar um
critério de extremalidade para objetos de uma estrutura. Em essência, ele se
baseia na intuição trivial de que: se um objeto é maximiza uma propriedade,
então não pode existir outro mais extremo. Para tanto, admite-se que haja um
objeto que atenda a condição de extremalidade e de um objeto que seja mais
extremo que o anterior, mostrado, em sequência, que isso leva a uma contradição.

\subsection{Argumento Probabilístico}

Os argumentos probabilísticos, popularizados pelo matemático Paul Erdös,
é uma classe de argumentos não construtivos que visam mostrar a existência
de um objeto numa estrutura que satisfaça uma dada propriedade. Eles fazem isso
mostrando que, se selecionássemos objetos aleatoriamente da estrutura,
então a probabilidade de eventualmente um objeto com a propriedade desejada
ser escolhido é maior do que zero.


