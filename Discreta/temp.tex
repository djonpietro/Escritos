\usepackage[T1]{fontenc}
\usepackage[utf8]{inputenc}
% Fontes
\usepackage{kpfonts}
\usepackage{calrsfs}
\usepackage{amssymb}
\usepackage{amsmath}
\usepackage{amsfonts}
\usepackage{amsthm}
% Parágrafos
\usepackage{parskip}
\usepackage{indentfirst}
% Colunas
\usepackage{multicol}
% Margem
\usepackage[
	a4paper,
	inner=3cm,
	outer=2cm,
	top=2.5cm,
	bottom=2.5cm
]{geometry}
% Figuras, Gráficos e Plots
\usepackage{tikz}
\usepackage{graphicx}
\usepackage{caption}
\usepackage{subcaption}
\usepackage{float}
\usepackage{tikz-3dplot}
% Riscar símbolos
\usepackage{cancel}
% Algoritmos
\usepackage[linesnumbered,ruled,vlined,portuguese]{algorithm2e}
\usepackage{hyperref}
\hypersetup{
    colorlinks,
    linkcolor={blue!50!black},
    citecolor={blue!50!black},
    urlcolor={blue!80!black}
}

% Tamanho do parágrafo
\setlength{\parindent}{1.25cm}
% Níveis de Título no Sumário
\setcounter{tocdepth}{1}

% Ambiente de Prova
\renewcommand{\proofname}{Demonstração}
\renewcommand{\qedsymbol}{\ensuremath{\blacksquare}}

% Redefinir o ambiente proof para desativar a indentação dentro dele
\makeatletter
\renewenvironment{proof}[1][\proofname]{\par
	\pushQED{\qed}%
	\normalfont\topsep6\p@\@plus6\p@\relax
	\trivlist
	\item[\hskip\labelsep
	\itshape
	#1\@addpunct{.}]\ignorespaces
	\setlength{\parindent}{0pt} % Remove indentação
}{%
	\popQED\endtrivlist\@endpefalse
}
\makeatother

\newcommand{\parts}[1]{\mathcal{P}(#1)}
\newcommand{\partsnn}[1]{\mathcal{P}^*(#1)}
\newcommand{\dm}[1]{\mathrm{Dm}(#1)}
\newcommand{\im}[1]{\mathrm{Im}(#1)}
\newcommand{\Arr}[1]{\mathrm{Arr}(#1)}
\newcommand{\Pm}[1]{\mathrm{P}(#1)}
\newcommand{\PR}[2]{\mathrm{PR}_{#1}^{#2}}
\newcommand{\C}[2]{\mathrm{C}_{#1}^{#2}}
\newcommand{\CR}[1]{\mathrm{CR}(#1)}


