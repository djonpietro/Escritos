\section{O que Matmética Discreta}

A definição de Matmática Discreta é difícil de se obter, pois pode-se dizer
que é mais fácil a definir pelo que está de fora dela do que pelo que de
fato está incluso nela. Ela é um vasto campo que estudo uma coleção
ampla e heterogênea de objetos, e que dialoga com diversas outras áreas,
muito pelo fato dela lidar, entre outras coisas, com princípios fundamentais
da matemática, como a contagem.

Numa tentiva de a definir, podemos dizer que a
Matemática Discreta é uma das grandes áreas da matemática que estuda
estruturas algébricas nas quais não há continuidade entre seus elementos.
Seus objetos de estudo são estruturas, conjuntos e funções que lidam
com elementos distintos e separados, tais como números inteiros, grafos
e fórmulas lógicas.

\section{Conjuntos Discretos}

Relembrando, conjuntos são coleções bem definidas de objetos, chamados de elementos
do conjunto. Geralmente, definimos um conjunto $C$ a partir de uma propriedade
$P(x)$ que todos os seus elementos satisfazem, o que é denotado por
\[
  C = \{ x \mid P(x)  \}
\]
Na Teoria de Conjuntos, há uma relação fundamental de pertinência, denotada
por $\in$, que indica que um elemento é membro de um dado conjunto.

A Teoria de Conjutnos também define uma propriedade chamada de cardinalidade
do conjunto, que, intuitivamente, diz a respeito do quão ``grande'' um
conjunto é. Se $C$ é um conjunto, então sua cardinalidade pode ser denotada
por $|C|$, ou por $\#C$, de forma que, neste texto, adotaremos a primeira
opção. Dois conjuntos possuem a mesma cardinalidade se há uma bijeção entre
os seus elementos, i.e., existe uma correspondência de um para um entre os
seus elementos. Conjuntos discretos, por definição, sempre possuem uma
bijeção com um subconjunto dos números naturais, o que, em termos matemáticos,
siginifica que eles podem ser enumerados.

Com essa propriedaade, podemos classificar os conjuntos discretos em dois tipos:
os conjuntos finitos e os conjuntos infinitos enumeráveis. Conjuntos finitos
são aqueles que tem correspondência um para um com um subconjunto dos naturais
diferente de $\mathbb{N}$, e sua cardinalidade pode ser entendida como a própria
quantidade de elementos do conjunto. Os infinitos enumeráveis são aqueles cuja
cardinalidade é igual a de $\mathbb{N}$.

\section{Funções Discretas}

Uma função -- ou aplicação -- $f$ é uma relação entre dois conjuntos $A$ e $B$,
denotada por
\[
  f: A \to B
\]
O conjunto $A$ é chamado de domínio da função, enquanto o conjunto $B$,
é denominado de contradomínio. A função $f$ associa todo elemento do domínio
a um, e somente um, elemento do contradomínio.
\[
  \forall x \in A, \exists ! y \in B \mid f(x) = y
\]
Se $x$ é um elemento do domínio associado a um elemento do contradomínio $y$,
dizemos que $y$ é a imagem de $x$ pela função $f$, o que é denotado por $f(x) = y$.
A notação $f(x)$ também pode ser usada para definir a lei de correspondência
da função, por exemplo, se $f: \mathbb{N} \to \mathbb{N}$ é a função que associa
todo número natural ao seu dobro, então a lei de correspondência pode ser
dada como $f(x) = 2x$.

O conjunto imagem de $f$, que pode ser denotado por $\im(f)$, é formado
pelos elementos do contradomínio que são imagem de algum elemento do
domínio
\[
  \im{f} = \{ y \in B \mid \exists x \in A \text{ tal que } f(x) = y \}
\]

Funções podem ser classificadas, normalmente, em três categorias

\begin{itemize}
  \item Injetora - uma função $f: A \to B$ é dita injetora se
    diferentes elementos do domínio são mapeados em diferentes elementos
    do contradomínio. Formalmente, $f$ é injetora se
    \[
      \forall x_1, x_2 \in A, f(x_1) = f(x_2) \implies x_1 = x_2
    \]
  \item Sobrejetora - uma função $f: A \to B$ é dita sobrejetora se o
    contradomínio coincide com o conjunto imagem da função
    \[
      \im{f} = B
    \]
  \item Bijetora - uma função $f: A \to B$ é dita bijetora se ela é injetora
    e sobrejetora ao mesmo tempo. Intuitivamente, uma função bijetora estabelece
    uma correspondência de um para um entre os elementos do domínio e do
    contradomínio.
\end{itemize}

As funções discretas são aquelas cujo o domínio é um conjunto discreto. Desse
modo, para o conjunto imagem, haveriam dois casos: a função é injetora, logo
todo elemento da imagem é imagem de um único elemento no domínio, ou há pelo
menos um elemento no contradomínio que imagem de mais de um no domínio. Em ambos
os casos, a consequência é que a imagem terá caradinalidade menor ou igual
à do domínio. Como o domínio é discreto, então a imagem também será um conjunto
discreto.

Um dos casos mais comuns de funções discretas são as sequências, como as
prograssões aritmémeticas e geométricas, cujos os termos podem ser determinados
em função do índice do termo. Outros exemplos de funções discretas comuns
são os fatoriais de inteiros não negativos, dado como o produto de todos os
inteiros não negativos até o número dado, e também os multiconjuntos.

Apesar do nome, multiconjuntos é um artifício para representar um conjunto
em que há repetição de elementos. Neste texto, um multiconjunto será
uma função discreta $m$ cujo contradomínio é $\mathbb{N}$, de modo que
a imagem $m(x)$ de um elemento $x$ do domínio será denominado multiplicidade
desse elemento no multiconjunto. Logo, quando tivermos o interesse
de representar uma coleção em que um elemento $x$ se repete $k$ vezes,
definiremos um multiconjunto $m$ com domínio ao qual $x$ pertença tal que
$m(x) = k$. Diremos que um multiconjunto trivial será aquele tal que
para todo elemento no domínio, sua multiplicidade é zero, e chamaremos de
suporte do multiconjunto $m$ o conjunto $U = \{x \mid m(x) > 0\}$.

\section{Áreas da Matemática Discreta}

Neste texto, iremos explorar alguns campos que são estudados primariamente
sob os olhos da Matmética Discreta. Todos os campos compartilham a propriedade
fundamental de estarem lidando com objetos como conjunto e funções discretas,
o que implica que eles podem distinguidos um dos outros, enumerados, ordenados
e contados.

\subsubsection{Recursão}

A recursão é uma forma de definição aplicada a objetos que podem ser descritos
em termos de outros objetos do mesmo tipo, porém mais simples. Em geral, uma
definição recursiva é composta por casos base, que correspondem aos objetos mais
simples e não dependem de outras definições, e por casos recursivos, nos quais
um objeto é definido a partir de instâncias menores ou mais simples de si mesmo.

Uma definição recursiva é dita bem definida quando todo objeto que não é um caso
base pode ser decomposto, em um número finito de passos, até alcançar um dos
casos base, garantindo assim que o processo de definição termine.

De modo mais formal, um conjunto admite uma definição recursiva quando
seus elementos podem ser descritos a partir de casos bases e de regras de
construção que utilizam objetos previamente definidos. Os casos base são
constituem os elementos mínimos da definição, enquanto que as regras recursivas
garantem que todo objeto, pode ser reduzido, num número finito de aplicações, a esses
casos.

O estudo da recursão em si não é necessariamente um campo da matemática discreta.
No entanto, as definições recursivas são, sem dúvida, uma das ferramentas mais
úteis para o estudo de objetos discretos pelo fato de muitos deles terem a
propriedade de serem definidos a partir de objetos discretos menores.

Além do mais, as definições recursivas permitem a demonstração de resultados
a partir do poderoso Princípio de Indução Matemática. A demonstração por
indução, geralmente aplicada em situações em que seja-se provar que
uma propriedade é satisfeita por todos os elementos de um conjunto, envolve
dois passos princiapais: provar que a propriedade é verdadeira para os casos
base e, supondo que a propriedade vale para um caso recursivo, provar
que, ao obter um novo caso pelas regras de construção da definição recursiva,
a propriedade valerá para esse novo caso.

Em resumo, a indução consiste em demonstrar que, se a propriedade vale para os
casos base e que ela se mantém verdadeira sempre que aplicamos a regra de
construção recursiva para um novo caso, então ela é verdadeira para todos os
casos.


\subsubsection{Combinatória}

A combinatória, também chamada de Análise Combinatória em contextos mais
elementares, estuda modos de contar, selecionar ou ordenar elementos de um
conjunto finito sem necessariamente listar todas as possibilidades. Por
sua vez, a combinatória pode ser dividida em alguns campos mais especializados,
como:

\begin{itemize}
  \item Enumerativa - busca contar o números de elementos de um conjunto
    que satisfazem uma determinada propriedade. É a área mais elementar da
    combinatória, sendo também chamada de contagem.

  \item Extremal - busca determinar o quão grande ou pequena uma coleção
    de elementos pode ser caso tenha que satisfazer determinadas restrições.
    Surge em problemas combinatórios relacionados a otimização.

  \item Algébrica - um campo que visa empregar métodos da Álgebra Abstrata,
    especialmente da Teoria de Grupos e da Representação, em contextos
    combinatórios, enquanto que, ao mesmo tempo, emprega métodos combinatórios
    para selecionar problema algébricos.

  \item Probabilística -  um campo com maior grau de especialização introduzido
    por Paul Erdös que estuda o emprego de métodos probablísticos para
    demonstrar a existência de um determinado objeto combinatório. Em essência,
    busca-se provar que, ao selecionar aleatoriamente um objeto em dado
    universo, a probabilidade de que o objeto escolhido satisfaça uma propriedade
    desejada é estritamente menor do que zero.
\end{itemize}

\subsubsection{Relações de Recorrência}

As relações de recorrência são fórmulas que definem termos de uma sequência
em função de termos anteriores. A relação de recorrência mais famosa é com
certeza a Sequência de Fibonacci; nela, o termo base é $a_0 = 0$, e a fórmula
de recorrência é dada por $a_n = a_{n-1} + a_{n-2}$, ou seja, um termos é
definido como a soma dos outros dois imediatamente anteriores. Também é
comum denotar recorrências como funções dos índices dos termos da sequeência,
que no mesmo exemplo dado seria $f(0) = 0$ e $f(n) = f(n-1) + f(n-2)$.

As recorrências surgem naturalmente em diversos problemas que envolvem objetos
discretos, de modo que a solução do problema torna-se resolver a recorrência,
isto é, achar uma fórmula não recursiva -- também chamada de fórmula fechada --
em termos do próprio índice para obter qualquer termo da sequência, sem precisar
computar os anteriores na ordem.

\subsubsection{Teoria dos Grafos}

A Teoria dos Grafos é um grande ramo da Matemática Discreta que lida
com conjuntos de objetos que estão relacionados entre si. Os objetos de um
grafo são representados por pontos, enquanto que as relações entre eles são
denotados por setas ou segmentos de retas, a depender do tipo da relação sendo
representada.

O estudo dos grafos gera ferramentas poderosas para a modelagem de disversos
problemas, que passam a poder ser solucionados por algoritmos que resolvem
questões como conectividade, caminhos mínimos, fluxos, emparelhamentos e
colorações. Dessa forma, problemas oriundos de áreas como ciência da
computação, engenharia, logística, biologia, redes sociais e economia podem
ser formalizados de maneira precisa e analisados sistematicamente por meio
dessas estruturas.
