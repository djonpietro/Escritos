\section{Anagramas}

Nesta seção iremos introduzir os anagramas de um multiconjunto, também
chamados de permutações com repetição. Depois, iremos aplicar o conceito
para contar as soluções não negativas de uma equação, e, no fim,
iremos demonstrar a função de contagem das combinações com repetição.

\subsection{Contagem de Anagramas}

A ideia de anagramas é análoga àquela para strings\footnote{sequência de símbolos}:
diferentes sequências de elementos em que trocar elementos repetidos em posições
distintas não altera a sequência.
\begin{def:anagrama}
    Seja $A$ um conjunto, e $s: A \to \mathbb{N}$ um multiconjunto.
    Chamamos $g: [n] \to A$ de anagrama de $s$ se
    \[
        n = \sum_{a \in A} s(a)
    \]
    e se $I(a)$ é o conjunto dos índices em $[n]$ que tem $a$ como imagem em $g$, i.e.,
    \[
        I(a) = \{k \in A \mid s(k) = a \land a \in A\}
    \]
    então $|I(a)| = s(a)$.
\end{def:anagrama}

O que vai nos interressar é a contagem dos anagramas de um multiconjunto $s$.
\begin{thm:enum anagramas}
\label{thm:enum anagramas}
	Seja $s: A \to \mathbb{N}$ um multiconjunto com $|A| = m$. Se
	\[
	    n = \sum_{a \in A} s(a),
	\]
	então o número de anagramas de $s$ é dado por
	\[
	    \frac{n!}{\prod_{a \in A} s(a)!}
	\]
\begin{proof}
	Com $A = [a_1, \ldots, a_m]$, denotemos $s_i = s(a_i)$ para $i \in [m]$.
	A prova usará contagem dupla para enumerar permutações de $n$ elementos.
	A primeira contagem é dada por simplesmente $Pm(n) = n!$. Para segunda,
	presumindo que o número de anagramas seja $x$, seja $g: [n] \to A$ representado
	pela tupla
	\[
	    (a_{ij})_{i \in [m] \land j \in [n]},
	\]
	em que o i-ésimo elemento de $A$ ocupa a j-ésima posição. Com efeito, para toda
	sequência $g$ teremos que
	\[
	    x \cdot s_i!
	\]
	é equivalente ao número permutações de $m + s_i - 1$ elementos considerando que
	permutar os termos iguais a $a_i$ geram sequências diferentes. Isso vale pelo PFC,
	pois para cada anagrama, podemos obter $s_i!$ ordenações em que considera-se a repetições
	de $a_i$. Generalizando, termos então que
	\[
	    x \cdot \prod_{i \in [m]} s_i!
	\]
	conta o número de permutações em que se considera a repetição de todos os elementos
	de $A$, equivalendo ao número de permutações de
	\[
	   m + \sum_{i \in [m]} s_i - 1 = m + n - m = n
	\]
	elementos. Ora, temos então que
	\begin{align*}
        x &\cdot \prod_{i \in [m]} s_i! = n! \\
        x &= \frac{n!}{\prod_{i \in [m]} s_i!}
    \end{align*}
\end{proof}
\end{thm:enum anagramas}

\subsection{Soluções Não Negativas de Equações}

Aqui demonstraremos bravemente uma aplicação do que vimos até aqui para resolver
um problema interessante. Equações são entidades algébricas que representam a
igualdade entre dois membros, e nosso foco será naquelas que assumem a forma:
\begin{equation}
\label{eq:soma-linear}
    \sum_{i = 1}^n x_i = r.
\end{equation}
O problema combinatório em questão é enumerar as soluções não negativas desse
tipo de equação, ou melhor, quantas tuplas $(a_1, \ldots, a_n)$, tal que
$a_i \in \mathbb{N},\ i \in [n]$, são solução para aquela equação.

Para chegar à solução, iremos utilizar um argumento de bijeção, determinando
uma associação entre uma tupla solução e uma forma única de a codificar.
Introduzamos o seguinte algoritmo:

\begin{algorithm}
\SetAlgoLined
\Entrada{Uma tupla $s = (a_1, \ldots, a_n)$ solução de uma equação na forma de \ref{eq:soma-linear}}
\Saida{Uma string que codifica a tupla $s$}
\Inicio{
    $str \gets ``''$\;
    \ParaCada{$i \in [n]$}{
        $p \gets$ string com $a_i$ símbolos iguais a ``.''\;
       $str \gets \mathrm{concatena}(str,p)$\;
       \Se{$i \neq n$}{
            $str \gets \mathrm{concatena}(str, ``+'')$\;
       }
    }
    \Retorna{$str$}
}
\caption{Codificação de uma solução inteira não negativa de uma equação}
\label{alg:codificacao solucao}
\end{algorithm}

Agora devemos provar a terminação e a corretude do algoritmo. A terminação
é trivial, pois a tupla é finita. A demonstração da corretude consiste
em provar que duas soluções distintas nunca terão a mesma codificação.
Para isso, basta argumentar que, se $(a_i)_{i \in [n]}$ e $(b_j)_{j \in [n]}$
são soluções de uma equação na forma de \ref{eq:soma-linear} que tem a mesma
codificação dada pelo algoritmo \ref{alg:codificacao solucao}, então elas
são a mesma solução. Isso é verdade, pois, para cada iteração no loop iniciado
na linha 3, a string $p$ da linha 4 será a mesma para duas entrada se, e somente,
se $a_i = b_i$. Com isso, se a codificação produzida for a mesma, então é verdade
que, para todo $i \in [n]$, temos $a_i = b_i$, provando que o algoritmo produz
uma única representação para cada solução.

Mas por que a representação dada pelo \ref{alg:codificacao solucao} é razoável?
Se $(a_i)_{i \in [n]}$ é uma solução, então
\[
    \sum_{i=1}^n a_i = r,
\]
e o total de símbolos iguais a ``.'' na representação é igual ao próprio $r$. Por
exemplo: a equação $x_1 + x_2 = 3$ tem como algumas de suas soluções as tuplas
$(1, 2)$ e $(3, 0)$, cujas representações seriam, respectivamente: ``.+..'' e
``...+''. Em ambas, temos 3 símbolos iguais a $3$. Outra invariante das
representações é que haverão $n-1$ símbolos iguais a ``+'', em que $n$ é
o número de variáveis, já que o algoritmo adiciona o simbolo citado para
cada iteração exceto a última.

Como toda solução admite uma representação dada pelo algoritmo \ref{alg:codificacao solucao},
e essa representação é única, então contar o número de soluções inteiras não negativas
de uma equação como em \ref{eq:soma-linear} equivale a contar quantos anagramas existem
com $r$ símbolos iguais a ``.'' e $n-1$ símbolos iguais a ``+''. Pelo teorema
\ref{thm:enum anagramas}, teremos que essa quantidade é exatamente
\[
    \frac{(r+n-1)!}{r!(n-1)!}
\]

\subsection{Combinações com Repetição}

Quando abordamos combinações de um conjunto finito, apenas consideramos aquelas
sem repetição, isto é, aos subconuntos do conjunto de escolhas. No entanto, neste
momento, estamos aptos a prover um argumento combinatório para demonstrar a
função de contagem de uma combinação $s: A \to \mathbb{N}$ qualquer, isto é,
com repetição de elementos.

\begin{thm:enum combinacao com repeticao}
    Seja $A$ um conjunto de $n$ elementos. A função de contagem das combinações
    $s: A \to \mathbb{N}$ de $r$ elementos de $A$ é dada por
    \[
       \CR{r, n} = \frac{(r+n-1)!}{r!(n-1)!}
    \]
\begin{proof}
	Admitindo que $A = [a_1, \ldots, a_n]$, toda combinação com repetição de $r$
	elementos pode ser representada por uma sequência $(s_1, \ldots, s_n)$ tal que
	\[
	    s_i = s(a_i)
	\]
	para $i \in [n]$. Como essas combinações são ditas com $r$ elementos, então
	todas elas devem satisfazer que
	\[
	    \sum_{i = 1}^n s_i = r.
	\]
	Dessa forma, toda representação de tupla $(s_i)_{i \in [n]}$ para
	uma combinação de $r$ elementos equivale a uma solução da equação anterior, e,
	pelo visto na seção anterior, temos que há
	\[
	   \frac{(r+n-1)!}{r!(n-1)!}
	\]
	combinações com repetição de $r$ elementos de $A$.
\end{proof}
\end{thm:enum combinacao com repeticao}
