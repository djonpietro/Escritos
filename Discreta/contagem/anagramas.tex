\section{Anagramas}

Nesta seção iremos introduzir os anagramas de um multiconjunto, também
chamados de permutações com repetição. Depois, iremos aplicar o conceito
para contar as soluções não negativas de uma equação, e, no fim,
iremos demonstrar a função de contagem das combinações com repetição.

\subsection{Contagem de Anagramas}

A ideia de anagramas é análoga àquela para strings\footnote{sequência de símbolos}:
diferentes sequências de elementos em que trocar elementos repetidos em posições
distintas não altera a sequência.
\begin{def:anagrama}
    Seja $A$ um conjunto, e $s: A \to \mathbb{N}$ um multiconjunto.
    Chamamos $g: [n] \to A$ de anagrama de $s$ se
    \[
        n = \sum_{a \in A} s(a)
    \]
    e se $I(a)$ é o conjunto dos índices em $[n]$ que tem $a$ como imagem em $g$, i.e.,
    \[
        I(a) = \{k \in A \mid s(k) = a \land a \in A\}
    \]
    então $|I(a)| = s(a)$.
\end{def:anagrama}

O que vai nos interressar é a contagem dos anagramas de um multiconjunto $s$.
\begin{thm:enum anagramas}
	Seja $s: A \to \mathbb{N}$ um multiconjunto com $|A| = m$. Se
	\[
	    n = \sum_{a \in A} s(a),
	\]
	então o número de anagramas de $s$ é dado por
	\[
	    \frac{n!}{\prod_{a \in A} s(a)!}
	\]
\begin{proof}
	Com $A = [a_1, \ldots, a_m]$, denotemos $s_i = s(a_i)$ para $i \in [m]$.
	A prova usará contagem dupla para enumerar permutações de $n$ elementos.
	A primeira contagem é dada por simplesmente $Pm(n) = n!$. Para segunda,
	presumindo que o número de anagramas seja $x$, seja $g: [n] \to A$ representado
	pela tupla
	\[
	    (a_{ij})_{i \in [m] \land j \in [n]},
	\]
	em que o i-ésimo elemento de $A$ ocupa a j-ésima posição. Com efeito, para toda
	sequência $g$ teremos que
	\[
	    x \cdot s_i!
	\]
	é equivalente ao número permutações de $m + s_i - 1$ elementos considerando que
	permutar os termos iguais a $a_i$ geram sequências diferentes. Isso vale pelo PFC,
	pois para cada anagrama, podemos obter $s_i!$ ordenações em que considera-se a repetições
	de $a_i$. Generalizando, termos então que
	\[
	    x \cdot \prod_{i \in [m]} s_i!
	\]
	conta o número de permutações em que se considera a repetição de todos os elementos
	de $A$, equivalendo ao número de permutações de
	\[
	   m + \sum_{i \in [m]} s_i - 1 = m + n - m = n
	\]
	elementos. Ora, temos então que
	\begin{align*}
        x &\cdot \prod_{i \in [m]} s_i! = n! \\
        x &= \frac{n!}{\prod_{i \in [m]} s_i!}
    \end{align*}
\end{proof}
\end{thm:enum anagramas}
