\section{Combinações}

\subsection{Definição e Enumeração}

Combinações são uma abstração da ação de selecionar elementos de um conjunto,
de modo que a ordem em que os selecionamos não importa, apenas a coleção obtida
ao final. Essa seleção pode ser feita sem ou com repetição de elementos: no
primeiro caso, a seleção gera um sobconjunto do conjunto original, enquanto
que na segunda, obtemos um multiconjunto, cuja eliminação das redundâncias
também se reduz a um subconjunto da coleção original.

A nossa definição formal de combinação parte da ideia de criar uma coleção
de objetos de um conjunto discreto associando uma multiplicidade a cada
objeto, que poderia ser entendido como ``quantas vezes o objeto aparece
nessa coleção''.

\begin{def:combinacao}
  Seja $A$ um conjunto discreto. Uma combinação dos elementos de $A$
  de tamanho $r$ é um multiconjunto $c: A \to \mathbb{N}$. Se $c$
  é binário, então a combinação é dita sem repetição.
\end{def:combinacao}

O resultado a seguir apresenta a função de contagem de
combinações de um conjunto finito no caso sem repetição.

\begin{thm:enum combinacao}
 Seja $A$ um conjunto de $n$ elementos. O número de combinações com $r$
 elementos de $A$ será dada por
 \[
   \C{n}{r} = \frac{n!}{(n-r)!r!}
 \]
 \begin{proof}
   Nosso argumento será por contagem dupla.
   A cada elemento de $A$, a combinação pode associar somente dois valores:
   0 e 1, que pode ser compreendido como ao pertencimento ou não do elemento
   à combinação. Diante disso, contar quantas combinações de $r$ elementos
   podem ser obtidas de $A$ equivale ao problema de contar o total de
   subconjuntos de $r$ elementos que podem ser obtidos de $A$.

   Suponha que haja
   $x$ subconjuntos de $r$ elementos de $A$ denotados por $C_1, C_2, \ldots, C_x$.
   Dos elementos de um subconjunto $C_i$, $1 \leq i \leq n$, podemos obter $r!$
   arranjos dos elementos de $A$, e como não é possível produzir um desses
   arranjos a partir de outro sobconjunto $C_j$, então podemos afirmar que a
   contagem dos arranjos de $r$ de elementos de $A$ pode ser dada por $x \cdot r!$.

   No entato, pela função de contagem de arranjos, temos então:
   \begin{gather}
      x \cdot r! = \frac{n!}{(n-r)!} \\
      x = \frac{n!}{(n-r)!r!}
   \end{gather}
 \end{proof}
\end{thm:enum combinacao}

\subsection{Permutações com Repetição}

As permutações com repetição devem ser tratadas como um caso à parte das
permutações tradicionais. Isso porque essas últimas são nada mais do que
um caso especial dos arranjos, enquanto que a noção geral de permutação
com repetição carrega consigo a noção de multiplicidade de elementos,
como ocorre com as combinações.

Dado um conjunto $A$ finito, permutações com repetição dos elementos desse
conjunto só podem ser construídas se tivermos, além do conjunto, a
multiplicidade de cada elemento. A forma usual de as representar
é como ordenações de elementos de $A$ tal que cada elemento repete-se
tantas vezes quanto for a sua multiplicidade.

\begin{def:permutacao com repeticao}
  Seja $A$ um conjunto de $n$ elementos. Dado um multiconjunto $m: A \to \mathbb{N}$,
  uma permutação com repetição de elementos de $A$ é uma função sobrejetora
  $f: [T] \to A$ tal que
  \[
    T = \sum_{a \in A} m(a)
  \]
  e que satisfaz
  \[
    \forall a \in A,\ C = \{x \in [T] \mid f(x) = a\} \rightarrow |C| = m(a)
  \]
\end{def:permutacao com repeticao}
Vamos explicar a última definição em mais detalhes. É dado um conjunto finito
$A$ de $n$ elementos e um multiconjunto $m: A \to \mathbb{N}$, que dará
a multiplicidade de cada elemento $a \in A$ na permutação. Em seguida, definimos
$T$ como a soma das multiplicidades, e adicionamos a restrição de que o
número de elementos em $[T]$, que são os índices das posições da ordenação
representada na permutação, ligados a um mesmo elemento de $A$ deve ser igual
a multiplicidade desse elemento\footnote{Repare que o caso sem repetição
é equivalente a quando $m(a) = 1$ para todo $a \in A$.}.

Tendo, então, definido as permutações com repetição, podemos usar as combinações
para determinar a função de contagem delas dado um conjunto finito $A$ e
multiplicidades definidas por $m: A \to \mathbb{N}$.

\begin{thm:enum permutacao com repeticao}
  Seja $A$ um conjunto de $n$ elementos. O número de permutações com repetição
  de elementos de $A$ com multiplicidades dadas por um multiconjunto $m: A \to \mathbb{N}$,
  tal que $m(a_i) = r_i$, $a_i \in A$, e $1 \leq i \leq n$, é dado por
  \[
    \PR{T}{r_1\ldots r_n} = \frac{ T! }{ r_1! r_2! \ldots r_n! }
  \]
  onde $T$ é a soma das multiplicidades.
  \begin{proof}
    Mais uma vez, recorreremos à técnica de contagem dupla. Nesta prova,
    iremos contar as permutações com repetições as igualando ao número de sequências
    de $T$ elementos em que um elemento $a_i \in A$ pode se repetir $r_i$
    vezes, de modo que a posição do elemento na tupla representa ele
    ser imagem do índice $i \in [T]$ daquela posição.

    Dadas as multiplicidades dos elementos, podemos afirmar que a tupla terá
    tamanho igual a $T = r_1 + \ldots + r_n$. A construção de uma permutação poderá
    ser feita da seguinte maneira: para o primeiro elemento, selecionamos $r_1$
    posições que serão ocupadas por ele na tupla dentre as $T$ possíveis, isto
    é, de C$_T^{r_1}$ maneiras, restando $T - r_1$ posições para alocar os demais.
    Para o segundo elemento, é possível escolher $r_2$ possições dentre $T - r_1$
    de $C_{T - r_1}^{r_1}$, sobrando, por consequeência, $T - r_1 - r_2$ para os outros.
    Generalizando: para cada índice $k$, teremos $C_{T - r_1 - \ldots - r_k}^{r_k}$
    maneiras de escolher $r_k$ posições de um total de
    $T - \sum_{j=1}^{k-1} r_j$. Dessa forma, o número de permutações com
    repetição que podem ser feitas, pelo PFC, é dado por
    \[
      C_T^{r_1} \cdot C_{T - r_1}^{r_2} \cdot \ldots \cdot C_{T - r_1 - \ldots - r_n}^{r_n}
    \]
    expandindo os termos e cancelando os semelhantes,
    \begin{align}
    \frac{T!}{\cancel{(T - r_1)!} \, r_1!}
    \cdot
    \frac{\cancel{(T - r_1)!}}{\cancel{(T - r_1 - r_2)!} \, r_2!}
    \cdot \ldots \cdot
    \frac{\cancel{(T - r_1 - \ldots - r_{n-1})!}}{\cancelto{1}{(T - r_1 - \ldots - r_n)!} \, r_n!}
    \end{align}
    donde segue a expressão desejada
    \[
      \frac{ T! }{ r_1!r_2!\ldots r_n! }
    \]
  \end{proof}

\end{thm:enum permutacao com repeticao}

\subsection{Soluções não Negativas de uma Equação e Combinações com Repetição}

Vamos agora explorar um famoso problema elementar de combinatória antes de nos
aprofundarmos mais nas combinações com repetição. O problema em questão é
enumerar as soluções não negativas de uma equação na forma
\[
  \sum_{i=1}^{n} x_i = r
\]
onde $x_i$ são variáveis inteiras não negativas e $r$ é uma constante não
negativa dada. Como poderemos resolver esse problema?

Nossa estratégia será contruir uma representação para cada uma das soluções
da equação por meio de uma string formada pelos símbolos ``$.$'' e
``$+$''. Suponha que $(a_1, \ldots, a_n)$ seja uma solução da equação
$x_1 + \ldots + x_n = r$. Para produzir a string que repsenta a solução, siga os
passos:

\begin{enumerate}
  \item Tome o valor da posição $a_i$ da tupla e escreva, da esquerda para
    direita, tantos símbolos ``.'' quanto for o valor de $a_i$.
  \item Se o último $a_i$ foi avaliado no passo anterior, termine, senão escreve
    um símbolo de ``$+$'' e volte ao passo anterior.
\end{enumerate}

Vamos a exemplos. A equação $x_1 + x_2 = 3$ tem, entre suas soluções, as tuplas
$(1, 2)$ e $(0, 3)$. As strings produzidas pelos passos mencionados seriam,
respectivamente, ``$.+..$'' e ``$+...$''. É fácil ver que cada tupla produz apenas
uma string na forma proposta, e que não é possível que tuplas diferentes produzam
a mesma string. Portanto, há uma bijeção entre uma solução da equação e uma
string produzida conforme os passos especificados.

Obtemos, então, que o número de soluções inteiras não negativas de uma equação
é equivalente ao número de anagramas de uma string que tenha $r$ símbolos
iguais a ``$.$'' e $n-1$ símbolos iguais a ``$+$''. Isso configura um problema de
permutação com repetição, que, como vimos na seção anterior, pode ser resolvido
ao computar a função de contagem $\PR{r+n-1}{r,n-1}$.
\[
  \PR{r+n-1}{r,n-1} = \frac{ (r + n - 1)! }{r! (n-1)! }
\]
Temos, pois, que essa será a função de contagem das soluções inteiras de uma
equação na forma $x_1 + \ldots + x_n = r$. Com isso, estamos prontos para
determinar a função de contagem das combinações com repetição.

\begin{thm:enum combinacao com repeticao}
Seja $A$ um conjunto de $n$ elementos. A função de contagem das combinações
com repetição com $r$ elementos é definida por
\[
  \CR{n, r} = \frac{ (r + n - 1)! }{r! (n-1)! }
\]

\begin{proof}
  Se $C$ é uma combinação com repetição de $r$ elementos de $A$, então, por
  definição, $C$ associa cada elementos $a_i \in A$, $ 1 \leq i \leq n$, a
  um valor $C(a_i)$ tal que
  \[
    \sum_{i=1}^n C(a_i) = r
  \]
  Cada solução da equação acima equivale a uma associação que define uma
  combinação com repetição de $r$ elementos. Logo,
  contar quantas combinações desse tipo podem ser feitas equivale a contar
  quantas soluções inteiras não negativas a equação anterior possui, que,
  como havíamo deduzido, será
  \[
    \frac{ (r + n - 1)! }{r! (n-1)! }
  \]
\end{proof}
\end{thm:enum combinacao com repeticao}
