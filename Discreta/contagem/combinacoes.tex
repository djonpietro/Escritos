\section{Combinações}

\subsection{Definição e Enumeração}

Combinações são uma abstração da ação de selecionar elementos de um conjunto,
de modo que a ordem em que os selecionamos não importa, apenas a coleção obtida
ao final. Essa seleção pode ser feita sem ou com repetição de elementos: no
priemeiro caso, a seleção gera um sobconjunto do conjunto original, enquanto
que na segunda, obtemos um multiconjunto, cuja eliminação das redundâncias
também se reduz a um subconjunto da coleção original.

\begin{def:combinacao}
  Seja $A$ um conjunto discreto. Uma combinação $C$ dos elementos de $A$
  é um multiconjunto não trivial dos elementos de $A$. Quando a imagem de $C$
  está contida em $\{0, 1\}$, então dizemos que ela é uma combinação
  sem reptição, do contrário, com repetição.
\end{def:combinacao}

Daqui em diante, iremos acordar algumas terminologias. Quando nos referimos
apenas a ``combinação'', estaremos apenas omitindo o complemento ``sem repetição'',
equanto que sempre iremos adicionar ``com repetição'' quando estivermos lidando
com esse caso. Além disso, sempre que dissermos que uma combinação possui $r$
elementos, estaremos dizendo que a soma das multiplicidades associadas pelo
multiconjunto é igual a $r$.

Acordado a terminologia, iremos agora determinar a função de contagem de uma
ombinação de $r$ elementos.

\begin{thm:enum combinacao}
 Seja $A$ um conjunto de $n$ elementos. O número de combinações com $r$
 elementos de $A$ será dada por
 \[
   \C{n}{r} = \frac{n!}{(n-r)!r!}
 \]
 \begin{proof}
   Contar quantas combinações de $r$ elementos podem ser obtidas
   de $A$ equivale ao problema de contar o total de subconjuntos de $r$
   elementos que podem ser obtidos de $A$. Suponha que haja $x$ subconjuntos
   de $r$ elementos de $A$ denotados por $C_1, C_2, \ldots, C_x$. Dos elementos
   de um subconjunto $C_i$, $1 \leq i \leq n$, podemos obter $r!$ arranjos
   dos elementos de $A$, e como não é possível produzir um desses arranjos
   a partir de outro sobconjunto $C_j$, então podemos afirmar que a contagem
   dos arranjos de $r$ de elementos de $A$ pode ser dada por $x \cdot r!$.
   Pela fórmula canônica de enumeração de arranjos, temos então:
   \begin{gather}
      x \cdot r! = \frac{n!}{(n-r)!} \\
      x = \frac{n!}{(n-r)!r!}
   \end{gather}
 \end{proof}
\end{thm:enum combinacao}

\subsection{Permutações com Repetição}

Seções atrás, comentamos sobre os arranjos com repetição, cuja fómula de
contagem poderia ser deduzida de um argumento combinatório elementar. No entanto,
a mesma situação não se aplica sobre as permutações com repetição, que são
tuplas em que um mesmo elemento se repete um número fixo de vezes.

\begin{def:permutacao com repeticao}
  Seja $A$ um conjunto de $n$ elementos. Dado um multiconjunto $m: A \to \mathbb{N}$,
  uma permutação com repetição de elementos de $A$ é uma tupla em que
  um elemento $a \in A$ ocorre em $m(a)$ posições.
\end{def:permutacao com repeticao}

Permutações com repetição de um conjunto finito são gerada ao selecionar
elementos desse conjunto zero ou mais vezes e os ordenar de diferentes maneiras.
Um ponto de atenção ocorre no fato de que, se um elemento ocorre em duas ou mais
posições, permutações nessas posições não geram novas ordenações, o que torna
a enumeração desse tipo de estrutura um pouco mais criteriosa.

Tendo, então, definido as combinações. a tarefa de determinar uma função de
contagem de permutações com repetição de conjunto finito dadas as multiplicidades
de cada elemento torna-se possível por um argumento combinatório envolvendo
as estruturas definidas neste capítulo.

\begin{thm:enum permutacao com repeticao}
  Seja $A$ um conjunto de $n$ elementos. O número de permutações com repetição
  de elementos de $A$ com multiplicidades dadas por um multiconjunto $m: A \to \mathbb{N}$,
  tal que $m(a_i) = r_i$, $a_i \in A$, e $1 \leq i \leq n$, é dado por
  \[
    \PR{T}{r_1\ldots r_n} = \frac{ T! }{ r_1! r_2! \ldots r_n! }
  \]
  onde $T$ é a soma das multiplicidades.
  \begin{proof}
    Dadas as multiplicidades dos elementos, podemos afirmar que a tupla terá
    tamanho igual a $T = r_1 + \ldots + r_n$. A construção de uma permutação poderá
    ser feita da seguinte maneira: para o primeiro elemento, selecionamos $r_1$
    posições que serão ocupadas por ele na tupla dentre as $T$ possíveis, isto
    é, de C$_T^{r_1}$ maneiras, restando $T - r_1$ posições para alocar os demais.
    Para o segundo elemento, é possível escolher $r_2$ possições dentre $T - r_1$
    de $C_{T - r_1}^{r_1}$, sobrando, por consequeência, $T - r_1 - r_2$ para os outros.
    Generalizando: para cada índice $k$, teremos $C_{T - r_1 - \ldots - r_k}^{r_k}$
    maneiras de escolher $r_k$ posições de um total de
    $T - \sum_{j=1}^{k-1} r_j$. Dessa forma, o número de permutações com
    repetição que podem ser feitas, pelo Princípio Fundamental da Contagem,
    é dado por
    \[
      C_T^{r_1} \cdot C_{T - r_1}^{r_2} \cdot \ldots \cdot C_{T - r_1 - \ldots - r_n}^{r_n}
    \]
    expandindo os termos e cancelando os semelhantes,
    \begin{align}
    \frac{T!}{\cancel{(T - r_1)!} \, r_1!}
    \cdot
    \frac{\cancel{(T - r_1)!}}{\cancel{(T - r_1 - r_2)!} \, r_2!}
    \cdot \ldots \cdot
    \frac{\cancel{(T - r_1 - \ldots - r_{n-1})!}}{\cancelto{1}{(T - r_1 - \ldots - r_n)!} \, r_n!}
    \end{align}
    donde segue a expressão desejada
    \[
      \frac{ T! }{ r_1!r_2!\ldots r_n! }
    \]
  \end{proof}

\end{thm:enum permutacao com repeticao}
\subsection{Soluções não Negativas de uma Equação e Combinações com Repetição}
Vamos agora explorar um famoso problema elementar de combinatória antes de nos
aprofundarmos mais nas combinações com repetição. O problema em questão é
enumerar as soluções não negativas de uma equação na forma
\[
  \sum_{i=1}^{n} x_i = r
\]
onde $x_i$ são variáveis inteiras não negativas e $r$ é uma constante não
negativa dada. Como poderemos resolver esse problema?

Nossa estratégia será contruir uma representação para cada uma das soluções
da equação por meio de uma string formada pelos símbolos ``$.$'' e
``$+$''. Suponha que $(a_1, \ldots, a_n)$ seja uma solução da equação
$x_1 + \ldots + x_n = r$. Para produzir a string que repsenta a solução, siga os
passos:

\begin{enumerate}
  \item Tome o valor da posição $a_i$ da tupla e escreva, da esquerda para
    direita, tantos símbolos ``.'' quanto for o valor de $a_i$.
  \item Se o último $a_i$ foi avaliado no passo anterior, termine, senão escreve
    um símbolo de ``$+$'' e volte ao passo anterior.
\end{enumerate}

Vamos a exemplos. A equação $x_1 + x_2 = 3$ tem, entre suas soluções, as tuplas
$(1, 2)$ e $(0, 3$. As strings produzidas pelos passos mencionados seriam,
respectivamente, ``$.+..$'' e ``$+...$''. É fácil ver que cada tupla produz apenas
uma string na forma proposta, e que não é possível que tuplas diferentes produzam
a mesma string. Portanto, há uma bijeção entre uma solução da equação e uma
string produzida conforme os passos especificados.

Obtemos, então, que o número de soluções inteiras não negativas de uma equação
é equivalente ao número de anagramas de uma string que tenha $r$ símbolos
iguais a $.$ e $n-1$ símbolos iguais a $+$. Isso configura um problema de
permutação com repetição, que, como vimos na seção anterior, pode ser resolvido
ao computar a função de contagem $\PR{r+n-1}{r,n-1}$.
\[
  \PR{r+n-1}{r,n-1} = \frac{ (r + n - 1)! }{r! (n-1)! }
\]
Temos, pois, que essa será a função de contagem das soluções inteiras de uma
equação na forma $x_1 + \ldots + x_n = r$. Com isso, estamos prontos para
determinar a função de contagem das combinações com repetição.

\begin{thm:enum combinacao com repeticao}
Seja $A$ um conjunto de $n$ elementos. A função de contagem das combinações
com repetição com $r$ elementos é definida por
\[
  \CR{n, r} = \frac{ (r + n - 1)! }{r! (n-1)! }
\]

\begin{proof}
  Se $C$ é uma combinação com repetição de $r$ elementos de $A$, então, por
  definição, $C$ associa cada elementos $a_i \in A$, $ 1 \leq i \leq n$, a
  um valor $C(a_i)$ tal que
  \[
    \sum_{i=1}^n C(a_i) = r
  \]
  Cada solução da equação acima equivale a uma associação que define uma
  combinação com repetição de $r$ elementos. Logo,
  contar quantas combinações desse tipo podem ser feitas equivale a contar
  quantas soluções inteiras não negativas a equação anterior possui, que,
  como havíamo deduzido, será
  \[
    \frac{ (r + n - 1)! }{r! (n-1)! }
  \]
\end{proof}

\end{thm:enum combinacao com repeticao}

