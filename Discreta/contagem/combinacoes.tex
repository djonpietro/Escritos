\section{Combinações}

\subsection{Definição e Enumeração}

Combinações são uma abstração da ação de selecionar elementos de um conjunto,
de modo que a ordem em que os selecionamos não importa, apenas a coleção obtida
ao final. Essa seleção pode ser feita sem ou com repetição de elementos: no
priemeiro caso, a seleção gera um sobconjunto do conjunto original, enquanto
que na segunda, obtemos um multiconjunto, cuja eliminação das redundâncias
também se reduz a um subconjunto da coleção original.

\begin{def:combinacao}
  Seja $A$ um conjunto discreto. Uma combinação $C$ dos elementos de $A$
  é um multiconjunto não trivial dos elementos de $A$. Quando a imagem de $C$
  está contida em $\{0, 1\}$, então dizemos que ela é uma combinação
  sem reptição, do contrário, com repetição.
\end{def:combinacao}

O próximo passo será contar quantas combinações (sem repetição) de $r$ elementos
um conjunto finito possui.

\begin{thm:enum combinacao}
 Seja $A$ um conjunto de $n$ elementos. O número de combinações com $r$
 elementos sem repetição de $A$ será dada por
 \[
   \frac{n!}{(n-r)!r!}
 \]
 \begin{proof}
   Contar quantas combinações de $r$ elementos sem repetição podem ser obtidas
   de $A$ equivale ao problema de contar o total de subconjuntos de $r$
   elementos que podem ser obtidos de $A$. Suponha que haja $x$ subconjuntos
   de $r$ elementos de $A$ denotados por $C_1, C_2, \ldots, C_x$. Dos elementos
   de um subconjunto $C_i$, $1 \leq i \leq n$, podemos obter $r!$ arranjos
   dos elementos de $A$, e como não é possível produzir um desses arranjos
   a partir de outro sobconjunto $C_j$, então podemos afirmar que a contagem
   dos arranjos de $r$ de elementos de $A$ pode ser dada por $x \cdot r!$.
   Pela fórmula canônica de enumeração de arranjos, temos então:
   \begin{gather}
      x \cdot r! = \frac{n!}{(n-r)!} \\
      x = \frac{n!}{(n-r)!r!}
   \end{gather}
 \end{proof}
\end{thm:enum combinacao}

O valor da função de contagem de combinações de tamanho $r$ dum conjunto finito
de cardinalidade $n$ também pode ser denotada por $C^r_n$

\subsection{Permutações com Repetição}

Seções atrás, comentamos sobre os arranjos com repetição, cuja fómula de
contagem poderia ser deduzida de um argumento combinatório elementar. No entanto,
a mesma situação não se aplica sobre as permutações com repetição, que são
tuplas em que um mesmo elemento se repete um número fixo de vezes.

\begin{def:permutacao com repeticao}
  Seja $A$ um conjunto de $n$ elementos. Dado um multiconjunto $m: A \to \mathbb{N}$,
  uma permutação com repetição de elementos de $A$ é uma tupla em que
  um elemento $a \in A$ ocorre em $m(a)$ posições.
\end{def:permutacao com repeticao}

Permutações com repetição de um conjunto finito são gerada ao selecionar
elementos desse conjunto zero ou mais vezes e os ordenar de diferentes maneiras.
Um ponto de atenção ocorre no fato de que, se um elemento ocorre em duas ou mais
posições, permutações nessas posições não geram novas ordenações, o que torna
a enumeração desse tipo de estrutura um pouco mais criteriosa.

Tendo, então, definido as combinações. a tarefa de determinar uma função de
contagem de permutações com repetição de conjunto finito dadas as multiplicidades
de cada elemento torna-se possível por um argumento combinatório envolvendo
as estruturas definidas neste capítulo.

\begin{thm:enum permutacao com repeticao}
  Seja $A$ um conjunto de $n$ elementos. O número de permutações com repetição
  de elementos de $A$ com multiplicidades dadas por um multiconjunto $m: A \to \mathbb{N}$,
  tal que $m(a_i) = r_i$, $a_i \in A$, e $1 \leq i \leq n$, é dado por
  \[
    \frac{ T! }{ r_1! r_2! \ldots r_n! }
  \]
  onde $T$ é a soma das multiplicidades.
  \begin{proof}
    Dadas as multiplicidades dos elementos, podemos afirmar que a tupla terá
    tamanho igual a $T = r_1 + \ldots + r_n$. A cnstrução de uma permutação poderá
    ser feita da seguinte maneira: para o primeiro elemento, selecionamos $r_1$
    posições que serão ocupadas por ele na tupla dentre as $T$ possíveis,
    restando $T - r_1$ posições para alocar os demais. Para o segundo elemento,
    escolheremos $r_2$ possições dentre as $T - r_1$ possíveis, sobrando, por
    consequeência, $T - r_1 - r_2$ para os outros. Generalizando: para cada
    índice $k$, teremos de escolher $r_k$ posições de um total de
    $T - \sum_{j=1}^{k-1} r_j$. Dessa forma, o número de permutações com
    repetição será dada por
    \[
      C_T^{r_1} \cdot C_{T - r_1}^{r_2} \cdot \ldots \cdot C_{T - r_1 - \ldots - r_n}^{r_n}
    \]
    expandindo os termos e cancelando os semelhantes,
    \begin{align}
    \frac{T!}{\cancel{(T - r_1)!} \, r_1!}
    \cdot
    \frac{\cancel{(T - r_1)!}}{\cancel{(T - r_1 - r_2)!} \, r_2!}
    \cdot \ldots \cdot
    \frac{\cancel{(T - r_1 - \ldots - r_{n-1})!}}{\cancelto{1}{(T - r_1 - \ldots - r_n)!} \, r_n!}
    \end{align}
    donde segue a expressão
    \[
      \frac{ T! }{ r_1!r_2!\ldots r_n! }
    \]
  \end{proof}

\end{thm:enum permutacao com repeticao}
\subsection{Número de Seoluções não Negativas de uma Equação }
\subsection{Combinações com Repetição}
