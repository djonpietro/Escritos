\section{Princípio da Inclusão-Exclusão e Desarranjos}

Quando analisamos os princípio básicos de contagem, havíamos visto que o
Princípio Aditivo somente vale para determinar a cardinalidade de uma família finita
de conjuntos finitos caso ela fosse disjunta. Nesta seção, iremos mostrar um resultado
que permite determinar a cardinalidade de uma união de conjuntos finitos sem a restrição
de disjunção, chamado de Princípio de Inclusão-Exclusão. Após isso, iremos
mostrar um novo tipo de sequência chamada de desarranjo, ou permutação caótica.

\subsection{Princípio de Inclusão-Exclusão}

Intuitivamente, o princípio consiste em, ao mesmo tempo em que contamos os elementos
de cada conjunto para determinar a cardinalidade de sua união, contamos os elementos em suas
interseções a fim de subtrair contagens repetidas dum mesmo elemento. No entanto, essas
subtrações podem acabar removendo um elemento da contagem, sendo necessário reincluí-lo.
O procedimento procede até que tenhamos contado cada elemento exatamente uma vez.

Antes de enunciarmos formalmente o princípio, vajamos o lema;

\begin{lema:pie}
    Seja $\mathcal{F}$ uma família de conjuntos $\{A_1, \ldots, A_n\}$. Temos que
    \[
        \left| \bigcup_{A \in \mathcal{F}} A \right| \leq \sum_{A \in F} |A|
    \]
\begin{proof}
	Caso assim não o fosse, existiria algum elemento $a \in \bigcup_{A \in \mathcal{F}}A$
	tal que $a \notin A$ para todo $A \in \mathcal{F}$, o que é impossível.
\end{proof}
\end{lema:pie}

\begin{thm:pie}
Se $\mathcal{F}$ uma família de conjuntos $\{A_1, \ldots, A_n\}$, então
\[
\left| \bigcup_{A \in \mathcal{F}} A \right| = \sum_{k = 1}^n (-1)^{k+1}
\sum_{P \in \parts{\mathcal{F}} \land |P| = k}
\left| \bigcap_{A \in P} A \right|
\]
\begin{proof}
	A demonstração se dará por contagem dupla, mostrando que todo elemento contado no
	primeiro membro da expressão é contado exatamente uma vez no segundo.

	Seja $v$ um elemento em
	\[
	    U = \bigcup_{A \in \mathcal{F}},
	\]
	e seja $Q$ o conjunto definido tal como
	\[
	    Q = \{A \in \mathcal{F} \mid v \in A\}.
	\]
	Presuma que $|Q| = m \geq 1$, e $P \subset \mathcal{F}$, então
	\[
	    v \in \bigcap_{A \in P} A \leftrightarrow P \subseteq Q.
	\]
	Em outras palavras, $v$ será considerado numa contagem tantas vezes quanto forem
	os subconjuntos de $Q$. Em especial, para os subconjuntos $P \subset Q$ de classe
	$k \leq m$, teremos que $v$ será conseiderado
    \[
        \sum_{P \in \parts{Q} \land |P| = k} 1 = \binom{m}{k}
    \]
    vezes. Disse temos
    \[
        \sum_{k = 1}^m (-1)^{k+1} \binom{m}{k} = -\sum_{k = 1}^m (-1)^k \binom{m}{k} =
        -(-1 + (1-1)^m)
    \]
    pelo Binômio de Newton, e, uma vez que $m \geq 1$, então
    \[
        -(-1 + (1-1)^m) = -(-1) = 1
    \]
    e $v$ é contado exatamente uma vez no segundo membro da expressão sendo demonstrada.
\end{proof}
\end{thm:pie}

\subsection{Desarranjos}

Os desarranjos representam uma sequência em que os elmentos estão sempre fora de ordem,
a qual é definida a partir de uma ordenação prévia dos elementos. Por exemplo, os desarranjos
da ordenação $[1, 2, 3]$ incluem, $(2, 3, 1)$ e $(3, 1, 2)$, mas não $(1, 3, 2)$ nem
$(3, 2, 1)$, pois nesses dois últimos $1$ e $2$ ocupam as mesmas posições na ordenação
base respectivamente.

\begin{def:desarranjo}
    Se $n \in \mathbb{N}$, chama-se desarranjo toda sequência bijetora $d: [n] \to [n]$
    tal que
    \[
        \forall k \in [n],\ (d(k) \neq k).
    \]
    Se $A$ for um conjunto ordenado como $[a_1, \ldots, a_n]$,
    então chamaremos $\hat d: [n] \to A$ de desarranjo de $A$ caso seja bijeção e
    satisfaça
    \[
        \forall k \in [n],\ (\hat d(k) \neq a_k \leftrightarrow d(k) \neq k).
    \]
\end{def:desarranjo}
Daqui em diante, definiremos desarranjos para conjuntos finitos quaisquer sem
necessariamente explicitar o desarranjo $d: [n] \to [n]$ subjacente. Também
facultamos o uso do símbolo $\hat{}$ para denotar esses desarranjos.

Assim como fizemos para outros tipos de estruturas combinatórias -- arranjos, permutações,
combinações, anagramas, etc. -- iremos determinar o número de maneiras de desarranjar os
elementos de um conjunto. Isso será dado pelo resultado a seguir.

\begin{thm:enum desarranjo}
    Se $n \in \mathbb{n}$, então existem
    \[
        D(n) = n! \sum_{k = 0}^n \frac{(-1)^k}{k!}
    \]
    desarranjos para $[n]$.

\begin{proof}
	Todo desarranjo é um arranjo sem repetição, sendo, pois, uma função bijetora de $[n]$
	para $[n]$. Desse modo, a estratégia a ser utilizada será subtrair de todas as bijeções
	$d: [n] \to [n]$ aquelas em que existe pelo menos um $k \in [n]$ tal que $d(k) = k$.

	O número de bijeções $[n] \to [n]$ é dado por $n!$. Resta então contarmos quantas não
	são desarranjos. Para tanto, admita que $\mathcal{F} = \{A_1, \ldots, A_n\}$ seja uma
	família tal que $A_k$, com $k \in [n]$, seja o conjunto das bijeções $d: [n] \to [n]$
	em que $d(k) = k$. A fim de encontar a cardinalidade da união dos membros de
	$\mathcal{F}$, precisamos determinar:
	\[
	    \sum_{P \in \parts{F} \land |P| = k} \left|\bigcap_{A \in P} A \right|
	\]
	para todo $k \in [n]$. Para isso basta contar em quantas bijeções pelo menos $k$
	elementos têm a sim mesmo como imagem, o que é igual ao número de subconjuntos de
	classe $k$ de $[n]$ vezes o número de bijeções nas quais os elementos desses
	subconjuntos possuem a si como imagem. Desse modo,
	\[
    \sum_{P \in \parts{F} \land |P| = k} \left|\bigcap_{A \in P} A \right| = \binom{n}{k}\, (n-k)! =
    \frac{n!}{k!}
	\]
	donde segue
	\[
        D(n) = n! -\sum_{k=1}^n (-1)^{k+1} \frac{n!}{k!} = n! \left(1 - \sum_{k=1}^n
        \frac{(-1)^{k+1}}{k!}\right)
    \]
    e, finalmente
    \[
        D(n) = n! \left(1 + \sum_{k=1}^n  \frac{(-1)^{k}}{k!}\right) =
        n! \left(\sum_{k=0}^n  \frac{(-1)^{k}}{k!}\right)
    \]
\end{proof}
\end{thm:enum desarranjo}

A fórmula demonstrada no resultado anterior pode ser computacionalmente custosa. No entanto,
há uma curiosa forma de estimar o número de desarranjos de $[n]$. Do Cálculo e Análise
Matemática, conhece-se que a função $e^x$, em que $e$ é a constante de Euler, admite um
definição por uma expansão, chamada Série de Taylor, dada por
\[
    e^x = \sum_{k=0}^\infty \frac{x^k}{k!}
\]
que equivale a
\[
    e^x = \sum_{k=0}^n \frac{x^k}{k!} + \sum_{k=n+1}^\infty \frac{x^k}{k!}
\]
Conclui-se então que o número de desarranjos de $[n]$ pode ser estimado por
\[
    n! \cdot e^{-1}
\]
cujo erro é exatamente
\[
    n!\sum_{k=n+1}^\infty \frac{x^k}{k!}
\]
É possível mostrar ainda com as ferramentas da análise que a magnitude da última expressão
não é superior a $1/2$. Em outras palavras,
\[
    D(n) \approx n! \cdot e^{-1} .
\]
