\section{Arranjos e Permutações}

Nesta seção iremos explorar algumas estruturas combinatórias baseadas em
tuplas ordenada, e determinaremos fórmulas para que consigamos enumerar quantas
dessas estruturas podem ser obtidas a partir de um conjunto finito.

\subsection{Fatoriais}

O fatorial é uma função discreta que admite uma definição recursiva no domínio
dos inteiros não negativos, dada como:

\begin{def:fatorial}
 Se $n$ é um inteiro positivo, então o seu fatorial, denotado por $n!$ é
 definido por:
 \[
  n! = \begin{cases}
    1, &\text{ se } n = 0\\
    n \cdot (n-1)!,&\text{ se } n > 0\\
  \end{cases}
 \]
\end{def:fatorial}

A função fatorial é importante, pois ela surge naturalmente em diversos problemas
combinatórios, especialmente de enumeração, como alguns que veremos a seguir.

\subsection{Arranjos}

Os arranjos são nada mais do que uma tupla ordenada obtida sem repetição a
partir de um conjunto discreto.

\begin{def:arranjo}
 Seja $A$ um conjunto discreto. Um arranjo dos elementos de $A$ é uma tupla
 ordenada com elementos de $A$.
\end{def:arranjo}

Nosso problema será agora enumerar quantos arranjos são possíveis de se obter
a partir de um conjunto finito de $n$ elementos, já que para conjuntos discretos
inifinitos, infinitos arranjos podem ser formados. O primeiro caso será para
arranjos sem repetição, i.e., se $(a_1, \ldots, a_r)$ é um arranjo de $A$, então
não há $a_i = a_j$ para qualquer $i$ e $j$ inteiros positivos menores ou iguais
a $r$.

\begin{thm:enum arranjo}
  Se $A$ é um conjunto finito com $n$ elementos, então o número de arranjos de
  $r$ elementos desse conjunto é dado por
  \[
    \frac{n!}{(n-r)!}
  \]
  \begin{proof}
    Para provar a fórmula, iremos prover um argumento combinatório usando
    o Princípio Fundamental da Contagem. Sempre que escolhemos um elemento de
    $A$ para ocupar a posição de índice $i$, não podemos repetir esse elemento,
    ou seja, nosso universo de escolhas se reduz em 1 cada vez que fixamos
    um elemento. Diante disso, representemos nossos universos de escolhas em
    cada passo da construção da tupla com uma definição recursiva,
    presumindo que $a_1 \in A$ é o primeiro elemento escolhido, $a_2 \in A$
    o segundo, e assim em diante:
    \[
      A_i = \begin{cases}
        A, &\text{ se }i=0\\
        A_{i-1} \backslash \{a_i\}, &\text{ se }i>1\\
      \end{cases}
    \]
    Dessa forma, o conjunto $A_i$ representa os elementos que ainda podemos
    escolher para compor a tupla após já termos escolhido $i$ elementos.
    Por coseguinte, ao preencher as $r$ posições da tupla, teremos escolhido
    respectivamente, elementos da família $A_0, A_1, \dots, A_{r-1}$. Logo,
    enumerar arranjos é equivalente a enumerar quantas tuplas odenadas podem
    ser obtidas ao escolher, respectivamente, elementos da família citada, que
    pelo Princípio Fundamental da Contagem, deve ser igual ao produto das
    cardinalidades dos conjuntos. Uma vez que $|A_i| = |A| - i = n - i$, então
    o número de arranjos é dado por:
    \begin{gather*}
     n \cdot (n -1) \cdot \ldots \cdot (n - r - 1) \\
     n \cdot (n -1) \cdot \ldots \cdot (n - r - 1) \cdot \frac{(n-r)!}{(n-r)!} \\
    \end{gather*}
    que equivale a
    \[
      \frac{n!}{(n-r)!}
    \]
  \end{proof}
\end{thm:enum arranjo}
O caso de arranjos com repetição é siginificativamente mais trivial, pois
nosso universo de escolhas nunca se reduz, ou seja, para cada posição da tupla,
teremos $n$ escolhas possíveis, de forma que o número total de arranjos será
dado por $n^r$, provando o corolário abaixo.

\begin{cor:enum arranjo 1}
  O número de arranjos de tamanho $r$ com repetição de um conjunto $A$ de $n$
  elementos é $n^r$
\end{cor:enum arranjo 1}

Um caso especial dos arranjos são as permutações, que são arranjos tomados,
com ou sem repetição, de todos os elementos de um conjunto finito.

\begin{def:permutacao}
  Uma permutação de um conjunto finito $A$ de cardnalidade $n$ é um arranjo
  de tamanho $n$ dos elementos de $A$.
\end{def:permutacao}

\begin{cor:enum arranjo 2}
  Se $A$ poossui $n$ elementos, então há $n!$ permutações sem repetição de
  elementos de $A$.

  \begin{proof}
    Caso em que $n = r$ na fórmula de contagem de arranjos.
    \[
      \frac{n!}{(n-r)!} = \frac{n!}{0!} = n!
    \]
  \end{proof}
\end{cor:enum arranjo 2}
O caso de permutações com repetição iremos adiar por um momento.


