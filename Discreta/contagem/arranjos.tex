\section{Arranjos}

Arranjar os elementos de um conjunto finito significa criar sequência
de elementos desse conjunto. A repetição de cada elemento e a ordem da
sequência distinguem um arranjo um do outro.

\subsection{Arranjos}

A ideia de arranjos de um conjunto finito está ligada à maneiras de ordenar
elementos desse conjunto de diferentes formas em um número de posições
menor ou igual ao tamanho do conjunto.
sequência de elementos de um conjunto finito.

\begin{def:arranjo}
Seja $A$ um conjunto de $n$ elementos. Um arranjo de $r$ elementos de $A$,
tal que $r \leq n$, é uma função $f$ definida por:
\[
  f: [r] \to A
\]
Quando $f$ é injetora, o arranjo é dito sem repetição, ou com repetição
no caso contrário.
\end{def:arranjo}
Nos referenciaremos a arranjos sem repetição apenas como arranjos, enquanto
que a nomenclatura para o outro caso se mantém\footnote{Isso valerá para
outras estruturas combinatórias que admitam repetição de elementos.}.

Suponha que $f$ seja um arranjo de tamanho 2 do conjunto $\{a, b, c\}$, tal que
$f$ associe cada um desses elementos a uma posição numa sequência: a primeira ou
segunda. Disso, temos, por exemplo, que se $f(1) = a$ e $f(2) = b$, então
isso representa a sequência
\[
  (a, b),
\]
que é um caso sem repetição. Um caso com repetição poderia ser $f(1) = f(2) = a$,
gerando a sequência $(a, a)$.

Nosso problema será agora enumerar quantos arranjos são possíveis de se obter
a partir de um conjunto finito de $n$ elementos, já que para conjuntos discretos
infinitos, infinitos arranjos podem ser formados.

\begin{thm:enum arranjo}
  Se $A$ é um conjunto finito com $n$ elementos, então a função de contagem
  elementos desse conjunto é dado por
  \[
    \frac{n!}{(n-r)!}
  \]
  \begin{proof}
    A demonstração se baseia na própria definição do arranjo sem repetição:
    basta contarmos quantas funções injetivas $f: [r] \to A$ podem ser
    obtidas. Para isso, suponha que
    \[
      (a_1, \ldots, a_r)
    \]
    seja a sequência tal que $a_i$ é a imagem de $i \in [r]$.
    Pela restrição de injetividade, temos que $a_i \neq a_j$ para todos $i, j \in [r]$.
    Então, sempre que uma posição na sequência é definida,
    as possibilidades de preencher o próximo elemento
    é reduzido em 1. Se o conjunto original possui $n$ elementos, e o
    reduzimos de 1 em 1 por $r$ vezes a cada posição, deduz-se que
    o número de escolhas para cada posição é, respectivamente, $n$, $n-1$,
    $n-2$, \dots, $n-r+1$. Portanto, pelo Princípio Fundamental da Contagem temos,
    que o número de arranjos é dado por
    \begin{gather*}
     n \cdot (n -1) \cdot \ldots \cdot (n - r + 1) \\
     n \cdot (n -1) \cdot \ldots \cdot (n - r + 1) \cdot \frac{(n-r)!}{(n-r)!} \\
    \end{gather*}
    que equivale a
    \[
      \frac{n!}{(n-r)!}
    \]
  \end{proof}
\end{thm:enum arranjo}
O caso de arranjos com repetição é siginificativamente mais trivial, pois
não há restrição de injetividade, e para cada escolha de imagem $a \in A$
de um elemento $i \in [r]$, nosso universo de escolhas nunca se reduz, ou seja,
sempre teremos $n$ escolhas possíveis de imagens, de forma que o número total
de arranjos será dado por $n^r$, provando o corolário abaixo.

\begin{cor:enum arranjo 1}
  O número de arranjos de tamanho $r$ com repetição de um conjunto $A$ de $n$
  elementos é $n^r$
\end{cor:enum arranjo 1}

Um caso especial dos arranjos são as permutações, que são arranjos tomados de
todos os elementos de um conjunto finito.
\begin{def:permutacao}
  Uma permutação de um conjunto finito $A$ de cardnalidade $n$ é um arranjo
  de tamanho $n$ dos elementos de $A$, i.e., uma função bijetiva na forma
  \[
    f: [n] \to A
  \]
\end{def:permutacao}
O caso das permutações com repetição será discutido separadamente em seções
posteriores. Dito isso, a função de contagem de permutações de um conjunto finito
é facilmente deduzida pelo corolário a seguir.

\begin{cor:enum arranjo 2}
  Se $A$ poossui $n$ elementos, então há $n!$ permutações de elementos de $A$.
  \[
    \Pm{n} = n!
  \]
  \begin{proof}
    Caso em que $n = r$ na fórmula de contagem de arranjos.
    \[
      \frac{n!}{(n-r)!} = \frac{n!}{0!} = n!
    \]
  \end{proof}
\end{cor:enum arranjo 2}
