\section{Arranjos e Permutações}

Nesta seção iremos explorar algumas estruturas combinatórias básicas
e determinaremos fórmulas para que consigamos enumerar quantas
dessas estruturas podem ser obtidas a partir de um conjunto finito.

\subsection{Fatoriais}

O fatorial é uma função discreta que admite uma definição recursiva no domínio
dos inteiros não negativos, dada como:

\begin{def:fatorial}
 Se $n$ é um inteiro positivo, então o seu fatorial, denotado por $n!$ é
 definido por:
 \[
  n! = \begin{cases}
    1, &\text{ se } n = 0\\
    n \cdot (n-1)!,&\text{ se } n > 0\\
  \end{cases}
 \]
\end{def:fatorial}

A função fatorial é importante, pois ela surge naturalmente em diversos problemas
combinatórios, especialmente de contagem, como alguns que veremos a seguir.

\subsection{Arranjos}

A ideia de arranjos de um conjunto finito está ligada à ideia de ordenar
elementos desse conjunto de diferentes maneiras em um número de posições
menor ou igual ao tamanho do conjunto. Definiremos os arranjos como uma
forma de associar cada elemento a uma determinada posição nele, representada
por um índice numérico.

\begin{def:arranjo}
Seja $A$ um conjunto de $n$ elementos. Um arranjo de $r$ elementos de $A$,
tal que $r \leq n$, é uma função $f$ definida por:
\[
  f: [r] \to A
\]
Quando $f$ é injetora, o arranjo é dito sem repetição, ou com repetição
no caso contrário.
\end{def:arranjo}

Podemos representar arranjos como tuplas ordenas de $r$ elementos. Suponha
que $f$ seja um arranjo de tamanho 2 do conjunto $\{a, b, c\}$, tal que
$f$ associe cada um desses elementos a uma posição na tupla: a primeira ou
segunda. Disso, temos, por exemplo, que se $f(1) = a$ e $f(2) = b$, então
isso representa a tupla
\[
  (a, b)
\]
que é um caso sem repetição. Um caso com repetição poedria ser $f(1) = f(2) = a$,
gerando a tupla $(a, a)$.

Nosso problema será agora enumerar quantos arranjos são possíveis de se obter
a partir de um conjunto finito de $n$ elementos, já que para conjuntos discretos
inifinitos, infinitos arranjos podem ser formados. O primeiro caso será o sem
repetição, que é mostrado no teorema a seguir.

\begin{thm:enum arranjo}
  Se $A$ é um conjunto finito com $n$ elementos, então o número de arranjos de
  $r$ elementos desse conjunto é dado por
  \[
    \frac{n!}{(n-r)!}
  \]
  \begin{proof}
    A demonstração se baseia na própria definição do arranjo sem repetição:
    basta contarmos quantas funções injetivas $f: [r] \to A$ podem ser
    obtidas. Para isso, suponha que $(a_1, \ldots, a_r)$ seja a tupla
    tal que $a_i$ é a imagem de $i \in [r]$. Pela restrição de injetividade,
    temos que $a_i \neq a_j$ para todos $i, j \in [r]$. Então, sempre
    que uma posição na tupla é definida, se começássemos da esquerda para
    direita, o conjunto das possibilidades de preencher o próximo elemento
    é reduzido em 1. Se o conjunto original possui $n$ elementos, e o
    reduzimos de 1 em 1 por $r$ vezes a cada posição, deduz-se que
    o número de escolhas para cada posição é, respectivamente, $n$, $n-1$,
    $n-2$, \dots, $n-r+1$. Pelo Princípio Fundamental da Contagem temos,
    por consequência, que o número de arranjos é dado por
    \begin{gather*}
     n \cdot (n -1) \cdot \ldots \cdot (n - r + 1) \\
     n \cdot (n -1) \cdot \ldots \cdot (n - r + 1) \cdot \frac{(n-r)!}{(n-r)!} \\
    \end{gather*}
    que equivale a
    \[
      \frac{n!}{(n-r)!}
    \]
  \end{proof}
\end{thm:enum arranjo}
O caso de arranjos com repetição é siginificativamente mais trivial, pois
não há restrição de injetividade, e para cada escolha de imagem $a \in A$
de um elemento $i \in [r]$, nosso universo de escolhas nunca se reduz, ou seja,
sempre teremos $n$ escolhas possíveis de imagens, de forma que o número total
de arranjos será dado por $n^r$, provando o corolário abaixo.

\begin{cor:enum arranjo 1}
  O número de arranjos de tamanho $r$ com repetição de um conjunto $A$ de $n$
  elementos é $n^r$
\end{cor:enum arranjo 1}

Um caso especial dos arranjos são as permutações, que são arranjos tomados
sem repetição de todos os elementos de um conjunto finito.

\begin{def:permutacao}
  Uma permutação de um conjunto finito $A$ de cardnalidade $n$ é um arranjo
  de tamanho $n$ dos elementos de $A$, i.e., uma função bijetiva na forma
  \[
    f: [n] \to A
  \]
\end{def:permutacao}
O caso das permutações com repetição será discutido separadamente em seções
posteriores. Dito isso, a função de contagem de permutações de um conjunto finito
é facilmente deduzida pelo corolário a seguir.

\begin{cor:enum arranjo 2}
  Se $A$ poossui $n$ elementos, então há $n!$ permutações de elementos de $A$.
  \[
    \Pm{n} = n!
  \]
  \begin{proof}
    Caso em que $n = r$ na fórmula de contagem de arranjos.
    \[
      \frac{n!}{(n-r)!} = \frac{n!}{0!} = n!
    \]
  \end{proof}
\end{cor:enum arranjo 2}
