\section{Triângulo de Pascal}

O Triângulo de Pascal, ou Triângulo Aritmético, é uma representação triangular
dos coeficientes binomiais. Ele é construído de forma que os binomiais de mesmo
numerador estejam na mesma linha, enquanto os que possuem as mesmas classes estejam
na mesma coluna. O triângulos pode ser recursivamente construído utilizando a Relação
de Stifel e sobre ele podem ser provadas interessantes propriedades sobre sua estrutura,
especialmente a soma de números numa mesma linha, coluna ou diagonal. O triângulo
também possui uma importante relação com os coeficientes de um binômio $(a+b)^n$, também
chamado de binômio de Newton.

\subsection{Defnição dos Termos}

Como bem vimos, todo número binomial é definido a partir de um numerador $n$ e uma classe
$k$. Fazendo uso da Relação de Stifel, podemos construir uma definição recursiva para os
números binomiais da seguinte forma:

\begin{def:triangulo de pascal}
% \label{def:triangulo de pascal}
	Sejam $n$ e $k$ inteiros não negativos, e $f: \mathbb{N}^2 \to \mathbb{N}$ uma função
	definida tal como
	\[
	    f(n, k) = \begin{cases}
			0, &\text{ se } k > n\\
			1, &\text{ se } k = 0\\
			f(n - 1, k) + f(n - 1, k - 1), &\text{ caso contrário. }
		\end{cases}
	\]
	Chamos de Triângulo de Pascal a sequência bidimensional -- ou tabela -- definida
	por $f$.
\end{def:triangulo de pascal}

Nosso dever agora é mostrar que todo termo do triângulo equivale a um número binomial.

\begin{thm:eqvl binomial pascal}
	Todo termo do Triângulo de Pascal é um termo binomial e vice-versa.
\begin{proof}
	Para mostrar que um termo no triângulo é um binomial, usaremos indução.

	Base: Se $n$ e $k$ são naturais com $n < k$, então $f(n, k) = 0$, que equivalem aos
	binomiais cuja classe é maior do que o numerador, que são nulos por definição.
	Por outro lado, se $k = 0$, então $f(n, k) = 1$, que é equivalente ao
	binomial de numerador $n$ e classe nula, pois $\binom{n}{0} = 1$.

	Hipótese de Indução: Se $f(n, k)$ é um termo do Triângulo com $n \geq k$, com $k \neq 0$,
	então suponha que $f(n - 1, k)$ e $f(n - 1, k - 1)$ sejam os números binomiais
	\[
	    \binom{n-1}{k} \quad \text{ e } \quad \binom{n-1}{k - 1}
	\]
	respectivamente. Como esses números binomiais são consecutivos, então a sua soma
	é também um binomial pela Relação de Stifel, implicando que
	\[
	    f(n, k) = f(n-1, k) + f(n-1,k-1) = \binom{n-1}{k} + \binom{n-1}{k-1} = \binom{n}{k}
	\]
	e provando o caso indutivo. Logo, todo termo no Triângulo de Pascal é um número
	binomial pelo Princípio de Indução.

	Resta mostrar que todo binomial está presente no Triângulo. Isso é trivial uma vez que
	tenhamos mostrado o caso anterior, pois dele segue que
	\[
	    f(n, k) = \frac{n!}{(n-k)!k!} = \binom{n}{k}
	\]
	para quaisquer $n$ e $k$ inteiros não negativos com $k \leq n$.
\end{proof}
\end{thm:eqvl binomial pascal}

Dizemos que dois binomiais pertencem a mesma linha -- ou nível -- no triângulo se eles possuem os
mesmos numeradores, de tal modo a ter tantos termos numa mesma linha diferentes de 0 quanto
for a cardinalidade de $[n]$, em que $n \in \mathbb{N}$ é o índice da linha. Os numeradores
de uma mesma linha são todos iguais ao índice dela. Dadas duas linhas de índices $n$ e $m$
respectivamente, dizemos que a linha de índice $n$ é inferior a de índice $m$ se,
e somente se, $n > m$, pois é convencional construir o triângulo de cima para baixo,
em que os níveis mais baixos correspondem as linhas de maiores índices.

De forma análoga às linhas, dois termos pertencem a mesma coluna no triângulo se eles
possuirem a mesma classe. Cada coluna é indexada pela classe correspondente a ela e,
diferente das linhas, possui infinitos termos diferentes de 0. Uma coluna de índice
$k$ é dita à direita de uma de índice $\ell$ se, e somente se, $r > \ell$, já que
se convenciona dispor as colunas correspondentes às classes mais altas nesse sentido.

Por fim, temos as diagonais; dois termos $f(n, k)$ e $f(m, \ell)$ pertencem a mesma
diagonal caso satisfaçam
\[
    |m - n| = |\ell - k|
\]
Intuitivamente, dois termos estarão na mesma diagonal caso, a partir de um, é possível
chegar ao outro sempre deslocando-se, consecutivamente, uma linha abaixo e uma coluna à
direita, ou uma coluna à esquerda e uma linha acima.

\subsection{Propriedades do Triângulo de Pascal}

Exploraremos agora resultados que dizem a respeito do somatório dos termos que pertencem
a uma mesma linha, coluna ou diagonal no Triângulo de Pascal, essas que revelam
intressantes propriedades combinatórias dessas estruturas.

\begin{thm:linha pascal}
    O somatório de uma linha no triângulo de Pascal é uma potência de 2 com expoente
    igual ao índice da linha;
    \[
        \sum_{k=0}^n \binom{n}{k} = 2^n
    \]
\begin{proof}
	Cada termo de uma mesma linha é igual ao número de subconjuntos de $[n]$
	cuja classe é igual a coluna. Uma vez que a classe $k$ dos termos satisfaz
	$0 \leq k \leq n$, então esse problema é equivalente a contar o número de
	subconjuntos de $[n]$, ou seja, quantas combinações sem reptição
	$s: [n] \to \mathbb{N}$ existem para $[n]$.

	Todo elemento $x \in [n]$ admite dois valores de imagem em $s$: 0 ou 1.
	Com efeito, se para cada um dos elementos de um conjunto de cardinalidade $n$
	existem duas escolhas de imagem, então o número total de combinações sem
	repetição de $[n]$ deve ser igual a $2^n$ pelo PFC.
\end{proof}
\end{thm:linha pascal}

\begin{thm:coluna pascal}
	A soma de todos os termos da coluna $k$ até a linha $n$ do Triângulo
	de Pascal é dada por
	\[
	    \sum_{m = 1}^n \binom{m}{k} = \binom{n+1}{k+1}
	\]
\begin{proof}
    Para esta demonstração, utilizaremos um argumento de contagem dupla para o número de
    subconjuntos de classe $k+1$ de $[n+1]$, com $n \ge k$.

    A primeira contagem é direta: o número de subconjuntos de classe $k+1$ de $[n+1]$ é
    \[
    \binom{n+1}{k+1}.
    \]

    Para a segunda contagem, particionaremos esses subconjuntos de acordo com o seu maior
    elemento. Seja $m \in [n]$. Contaremos quantos subconjuntos de classe $k+1$
    de $[n+1]$ têm $m+1$ como maior elemento.

    Para que um subconjunto tenha $m+1$ como maior elemento, é necessário e suficiente
    escolher $k$ elementos entre os $m$ elementos de $[m]$. Portanto, o número desses
    subconjuntos é
    \[
    \binom{m}{k}.
    \]

    Como cada subconjunto de classe $k+1$ possui um único maior elemento, essa partição é
    disjunta e cobre todos os casos. Assim, somando sobre todos os valores possíveis de $m$,
    obtemos
    \[
    \sum_{m=0}^{n} \binom{m}{k} = \binom{n+1}{k+1}.
    \]
\end{proof}
\end{thm:coluna pascal}

\begin{thm:diagonal pascal}
    A soma da uma diagonal de índice $n$ do Triângulo de Pascal até a coluna de índice
    $k$ é igual a
    \[
        \sum_{\ell = 0}^k \binom{n + \ell}{\ell} = \binom{n + k + 1}{k}
    \]
\begin{proof}
	A abordagem para provar esse resultado consistirá em rescrever os termos dessa soma
	como o seus coeficientes binomiais complementares:
	\begin{align*}
		\binom{n}{0} &= \binom{n}{n}\\
		\binom{n+1}{1} &= \binom{n+1}{n}\\
		             &\vdots        \\
		\binom{n+k}{k} &= \binom{n+k}{n}\\
    \end{align*}
    Observe que o somatório dos termos de diagonal $n$ até a coluna $k$ é igual
    ao somatório da coluna $n$ até a linha $n+k$, que pode ser dado como
    \[
        \sum_{m = 0}^{n+k} \binom{m}{n} = \binom{n+k+1}{n+1} = \binom{n+k+1}{k}
    \]
\end{proof}
\end{thm:diagonal pascal}

\subsection{Binômio de Newton}

Vamos agora explorar uma das principais aplicações dos coeficientes binomiais
e do Triângulo de Pascal, que consiste na determinação dos coeficientes da expansão
de um binômio na forma
\[
    (a+b)^n
\]
em que $a, b \in \mathbb{R}$ e $n \in \mathbb{Z}$, comumente denominado Binômio de Newton.
Primeiramente, vejamos um exemplo. Seja o binômio $(a+b)^3$, que pode ser dado como
\[
    (a+b) \cdot (a+b)^2,
\]
e usando o polinômio notável do quadrado da soma, obtemos
\[
    (a+b) \cdot (a^2 + 2ab + b^2)
\]
donde resulta
\[
    a^3 + 3a^2b + 3ab^2 + b^3.
\]
Perceba que, se listarmos os coeficientes da expressão, da esquerda para a direita,
teremos os números 1, 3, 3 e 1. Esses valores, nesta ordem, tratam-se daqueles presentes
na linha da linha de índice três do Triângulo de Pascal, cuja representação em binômios
é dada por, respectivamente,
\[
    \binom{3}{0} \quad \binom{3}{1} \quad \binom{3}{2} \quad \binom{3}{3} \quad
\]
Observe, também que, o numerador trata-se do expoente $n = 3$ do binômio, enquanto
que a classe de cada coeficiente trata-se do expoente do índice da variável $b$ na
expansão. Mostraremos que isso não é apenas coincidência, e sim um padrão que se repete
para qualquer binômio com exponente não negativo.

\begin{thm:binomio de newton}
    Sejam $a, b \in \mathb{R}$ e $n \in \mathbb{Z}$. É verdade que
    \[
        (a+b)^n = \sum_{k = 0}^n \binom{n}{k}\, a^{n-k} b^k
    \]
\begin{proof}
	A prova se dará por indução em $n \in \mathbb{Z}$ não negativo.

	Base: Se $n = 0$, então
	\[
	    (a+b)^0 = \binom{0}{0} = 1
	\]

	Hipótese de Indução: suponha que a expansão vale para $n > 0$:
	\[
        (a+b)^n = \sum_{k = 0}^n \binom{n}{k}\, a^{n-k} b^k
    \]
    Mostremos que vale para $n+1$. Com efeito,
    \begin{gather}
    \nonumber
        (a+b)^{n+1} = (a+b) \cdot (a+b)^n = (a+b) \sum_{k = 0}^n \binom{n}{k}\, a^{n-k} b^k=\\
        \sum_{k = 0}^n \binom{n}{k}\, a^{n-k+1} b^{k} + \sum_{k = 0}^n \binom{n}{k}\, a^{n-k} b^{k+1}
        \label{eq:proof binomio}
    \end{gather}
    Definamos
    \[
        f(k) = \binom{n}{k}\, a^{n-k+1} b^k \quad\quad g(k) = \binom{n}{k}\, a^{n-k} b^{k+1}
    \]
    É verdade que, para todo $k \in [1, \ldots, n]$,
    \[
        f(k) = g(k-1)
    \]
    o que implica em
    \[
        f(k) + g(k-1) = \binom{n}{k}\, a^{n-k+1} b^k + \binom{n}{k-1}\, a^{n-k+1} b^{k} =
        \binom{n+1}{k}\, a^{n-k+1} b^k.
    \]
    Com isso, mostramos que a expressão \ref{eq:proof binomio} equivale a
    \begin{gather*}
        f(0) + \left(\sum_{k = 1}^{n}\, f(k) + g(k-1)\right) + g(n)=\\
        \binom{n}{0} a^{n+1} b^0 + \sum_{k=1}^n \binom{n+1}{k} a^{n-k+1} b^k + \binom{n}{n} a^0 b^{k+1}.
    \end{gather*}
    Uma vez que
    \[
        \binom{n}{0} = \binom{n+1}{0} \quad \text{ e } \quad  \binom{n}{n} = \binom{n+1}{n+1},
    \]
    então a penúltima expressão pode se rescrita como
    \[
        \sum_{k = 0}^{n+1} \binom{n+1}{k}\, a^{n-k+1} b^k
    \]
    Portanto, a expansão vale para o expoente $n+1$, e, pelo Princípio de Indução, para
    todo expoente não negaivo.
\end{proof}
\end{thm:binomio de newton}
