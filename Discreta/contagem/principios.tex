\section{Princípios Fundamentais de Contagem}

A grande maioria dos exercícios de contagem se baseiam no uso de dois
princípios básicos: o aditivo e o multiplicativo. A seguir, iremos enunciar
o problema fundamental de cada princípio e demonstrar o seu funcionamento.

\subsection{Princípio Aditivo}

O nosso primeiro problema de contagem será o de determinar a quantidade de
elementos na união de uma família finita de conjuntos finitos e disjuntos.
A seguir, enunciaremos o Princípio Aditivo para resolver esse problema.

\begin{thm:pa}
Se $\mathcal{F} = \{A_1, A_2, \dots, A_n\}$ é uma família de conjuntos finitos
disjuntos dois a dois, então cardinalidade da união $U$ desses
conjuntos é a soma de suas cardinalidades.
  \begin{proof}
  Considere o conjunto
  \[
  D = \bigcup_{A \in \mathcal{F}} (A \times \{A\}).
  \]
  Como os conjuntos de $\mathcal{F}$ são disjuntos dois a dois, para todo
  elemento $a \in U$ existe um único conjunto $A \in \mathcal{F}$ tal que
  $a \in A$. Definimos então a função
  \[
  f : U \to D,
  \qquad
  f(a) = (a,A).
  \]

  A função $f$ é bem definida e bijetiva. De fato, se $f(a) = f(b)$, então
  $(a,A) = (b,A)$, o que implica $a = b$, mostrando que $f$ é injetiva.
  Além disso, dado $(a,A) \in D$, temos $a \in U$ e $f(a) = (a,A)$, logo
  $f$ é sobrejetiva.

  Assim, $|U| = |D|$. Como $D$ é uma união disjunta e
  \[
  |D| = \sum_{A \in \mathcal{F}} |A|,
  \]
  concluímos que
  \[
  |U| = \sum_{A \in \mathcal{F}} |A|.
\]
  \end{proof}
\end{thm:pa}

\subsection{Princípio Multiplicativo}

Outro problema básico de contagem é determinar a cardinalidade
do produto cartesiano de dois ou mais conjuntos. Esse problema
também pode ser enunciado como o número de tuplas ordenas
podem ser geradas de forma que cada posição seja preenchida por
um elemento de um conjunto finito. A solução desse problema
encontra-se no Princípio Multiplicativo, também chamado de Princípio Fundamental
da Contagem (PFC) por ser a base das soluções de vários outros problemas
de contagem. O PFC é enunciado e demonstrado a seguir.

\begin{thm:pfc}
  Seja $\mathcal{F} = \{A_1, \ldots, A_n\}$ uma família de dois ou mais conjuntos
  finitos. A cardinalidade do produto cartesiano desses conjuntos
  é igual ao produto das cardinalides deles.
  \[
    |A_1 \times A_2 \times \ldots \times A_n| = \prod_{A \in \mathcal{F}} |A|
  \]
  \begin{proof}
    A prova será por indução no número de conjuntos $n$ em $\mathcal{F}$.

    Base: Se $n = 2$, então suponha que $\mathcal{F}$ seja formada
    pelos conjuntos $A$ e $B$, tal que
    \[
      A = [a_1, \ldots, a_m].
    \]
    Admita também uma partição de $R = A \times B$ dada por $\Psi \subset \parts{R}$
    tal que
    \[
      \Psi = \{P_1, \ldots, P_m\},
    \]
    em que, para $k \in [m]$, tem-se
    \[
      P_k = \{(a, b) \in A \times B \mid a = a_k\}
    \]
    Considere agora uma família de funções indexadas por $[m]$ tal que
    \begin{gather*}
      f_k: B \to P_k \\
      b \mapsto (a_k, b)
    \end{gather*}
    É verdade que, para todo $k \in [m]$, $f_k$ está bem definida e é
    bijetiva. Primeiramente, a sobrejetividade é garantida, pois
    se $(a_k, b) \in P_k$, então $\exists b \in B$ tal que $f_k(b) = (a_k, b)$.
    Também é verdade que $f_k$ é injetiva, uma vez que se $f_k(b_i) = f_k(b_j)$,
    para $i, j \in [|B|]$, então $(a_k, b_i) = (a_k, b_j)$, implicando
    que $b_i = b_j$. Isso mostra que $|B| = |P_k|$.

    Por outro lado, é fácil perceber que, se $g$ é a função definida por
    \begin{gather*}
      g: A \to \mathcal{P} \\
      a_k \mapsto P_k
    \end{gather*}
    então $g$ também é uma bijeção e $|A| = |\mathcal{P}| = m$. Além disso,
    pelo fato de $R$ ser a união da família de conjuntos disjuntos
    $\Psi$, temos que a cardinalidade de $R$ é igual a
    \[
      \sum_{P \in \Psi} |P|
    \]
    pelo Principio Aditivo, donde segue que
    \[
      |R| = \sum_{P \in \mathcal{P}} |P| = m \cdot |B| = |A| \cdot |B|
    \]
    o que prova o teorema para o caso base.

    Hipótese de Indução: Suponha que o teorema vale para uma família
    de $n$ conjuntos. É verdade que, se $\mathcal{F} = \{A_1, \ldots, A_{n+1}\}$
    e que
    \[
      R = A_1 \times \ldots \times A_{n+1}
    \]
    então o teorema vale para os primeiros $n$ conjuntos de $\mathcal{F}$ pela
    Hipótese de Indução. Por conseguinte, se $R' = A_1 \times \ldots \times A_n$,
    então $R = R' \times A_{n+1}$, que consiste no caso base de dois
    conjuntos, para qual o princípio de indução vale. Temos então
    \[
      |R| = |R'| \cdot |A_{n+1}| = \prod_{A \in R} |A|
    \]
    provando o teorema.
  \end{proof}
\end{thm:pfc}

\begin{cor:numero de escolhas}
Seja $\mathcal{F} = \{A_1, \ldots, A_n\}$ uma família de conjuntos finitos não
vazios. O número de funções de escolhas para $\mathcal{F}$ é igual a
\[
  \prod_{A \in \mathcal{F}} |A|
\]
\begin{proof}
  Toda função de escolha $f$ equivale a uma única tupla no produto
  cartesiano dos conjuntos de $\mathcal{F}$, o qual será equipotente ao
  conjunto das funções de escolha.
\end{proof}
\end{cor:numero de escolhas}
Esse resultado irá facilitar bastante nosso trabalho, pois ele pode ser
interpretado como: se uma escolha final é obtida por escolhas menores tomadas
individualmente de universos finitos, então o total de escolhas é igual
ao produto das quantidades de escolhas individuais em cada universo.
Em demonstrações de resultados futuros, usaremos essa abordagem para aplicar
o PFC.

\begin{cor:numero de funcoes}
\label{cor:numero de funcoes}
Sejam $A$ e $B$ conjuntos finitos tal que
\[
  n = |A| \quad \quad m = |B|.
\]
O número de funções de $f: A \to B$ é igual a
\[
  m^n.
\]
\begin{proof}
  Contar quantas funções de $A$ para $B$ existem equivale a contar
  de quantas formas podemos escolher um elemento de $B$ para ser
  imagem $A$. Dado que $A$ tem $n$ elementos, cada um para o qual
  há $m$ escolhas, então o número total de funções é dado por:
  \[
    \underbrace{m \cdot \ldots \cdot m}_{n} = m^n
  \]
\end{proof}
\end{cor:numero de funcoes}

\subsection{Arranjos}

Arranjar os elementos de um conjunto finito significa criar sequência
de elementos desse conjunto. A repetição de cada elemento e a ordem da
sequência distinguem um arranjo um do outro. Com isso, os arranjos são
equivalentes a sequências

\begin{def:arranjo}
Seja $A$ um conjunto de $n$ elementos. Diremos que uma sequência $S$ é
um arranjo de $r$ elementos de $A$ se
\[
  S: [r] \to A
\]
Quando $S$ for injetora, o arranjo é dito sem repetição.
\end{def:arranjo}
Do corolário \ref{cor:numero de funcoes} segue que o número de arranjos
de tamanho $r$ com de um conjunto com $n$ elementos é $n^r$. Para o caso sem
repetição, precisamos contar quantas funções injetivas existem entre dois conjuntos.

\begin{cor:numero de arranjos}
Se $A$ e $B$ são conjuntos finitos com cardinalidades $n$ e $m$ respectivamente,
então existem
\[
  \frac{n!}{(n - m)!}
\]
funções injetivas entre $A$ e $B$.

\begin{proof}
  O diferencial das injetivas para uma função qualquer é que, a cada escolha feita para
  imagem de um elemento de $A$, o universo de escolhas é reduzido em 1. Com efeito,
  o total de injeções será igual a
  \begin{gather*}
   n \cdot (n -1) \cdot \ldots \cdot (n - m + 1) \\
   n \cdot (n -1) \cdot \ldots \cdot (n - m + 1) \cdot \frac{(n-m)!}{(n-m)!}
  \end{gather*}
  que equivale a
  \[
    \frac{n!}{(n-m)!}
  \]
\end{proof}
\end{cor:numero de arranjos}
Desta forma, se o arranjo é de $r$ elementos, então a função de contagem de
arranjos dos elementos de um conjunto finito de $n$ elementos é
\[
  \Arr{n, r} = \frac{n!}{(n-r)!}
\]
