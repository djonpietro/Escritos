\section{Número Binomiais}

Os números binomiais são objetos combinatórios simples e que representam
o conceito de seleção de elementos de um conjunto finito. A seguir,
apresentaremos sua definição formal, identidades básicas e algumas
propriedades interessantes.

\subsection{Definição e Interpretação}

\begin{def:binomial}
Sejam $n$ e $k$ inteiros não negativos. Um número binomial, ou coeficiente
binomial, de numerador $n$ e classe $k$, denotado por
\[
\binom{n}{k}
\]
é igual ao número de subconjuntos de $k$ elementos presentes em $[n]$.
\end{def:binomial}

Conjuntos não admitem repetição de elementos, o que torna razoável
a interpretação combinatória de que um subconjunto $S$ dum conjunto $A$ é uma
seleção de membros desse último. Se chamarmos os subconjuntos de $A$, com
$|A| = n$, que possuem exatamente $k \leq n$ elementos de subconjuntos de classe
$k$, então o número binomial $\binom{n}{k}$ representa quantos subconjuntos
de classe $k$ o conjunto $A$ possui. Essa representação está de acordo
com a definição dos binomiais, uma vez que $|A| = |[n]|$, e ambos os conjuntos possuirão a
mesma quantidade de subconjuntos.

\subsection{Identidades Básicas}

Demonstraremos agora duas propriedades fundamentais sobre números binomiais.

\begin{def:binomial complementar}
 Seja o número binomial $\binom{n}{k}$, o seu complementar é aquele dado por
 \[
  \binom{n}{n - k}
 \]
\end{def:binomial complementar}

Se um subconjunto $S \subset A$ é uma seleção, então seu complementar $S^c$ é
a seleção complementar. Uma vez que para todo subconjunto há um único
complementar, então a contagem de seleção realizadas por um binomial
deve ser igual àquela feita pelo seu complementar, resultado esse formalizado
a seguir:

\begin{thm:binomial complementar}
Números binomiais complementares são iguais
\[
 \binom{n}{k} = \binom{n}{n - k}
\]
\begin{proof}
 Suponha que $C$ seja o conjunto dos subconjuntos de $[n]$ de classe $k$
 \[
  C = \{S \in \parts{[n]} \mid |S| = k\},
 \]
 e o conjunto $C'$ aquele cujos subconjuntos são de classe $n - k$
 \[
  C' = \{S \in \parts{[n]} \mid |S| = n-k\},
 \]
 Para todo $S \in C$, vale que $[n] \setminus S \in C'$, pois
 $|[n] \setminus S| = n - k$. Com efeito, $C'$ será o conjunto imagem da relação
 de complementar dos subconjuntos de classe $k$ e $n - k$. Em outras
 palavras, existe uma função $f$ sobrejetora tal que
 \begin{gather*}
  f: C \to C'  \\
  S \mapsto S^c
 \end{gather*}
 A função $f$ é injetora, pois vale que se $f(S_1) = f(S_2)$, então
 \begin{gather*}
  S_1^c = S_2^c \\
  S_1 = S_2
\end{gather*}
Sendo $f$ uma bijeção, então $|C| = |C'|$, implicando, finalmente, que
\[
 \binom{n}{k} = \binom{n}{n - k}
\]
\end{proof}
\end{thm:binomial complementar}

Introduzemos agora a relação de consecutividade entre binomiais.
\begin{def:binomial consecutivo}
Dois números binomiais são consecutivos se possuem o mesmo numerador
e o módulo da diferença de suas classes é 1.
\end{def:binomial consecutivo}
Uma propriedade fundamental dos números binomiais consecutivos é chamada
de Relação de Stifel, e é ela é a base da construção do famoso triângulo
de Pascal, que veremos mais adiante.

\begin{thm:stifel}[Relação de Stifel]
 Sejam $n$ e $k$ inteiros não negativos com $n - 1 > k$, então
\[
 \binom{n}{k} + \binom{n}{k+1} = \binom{n+1}{k+1}
\]
\begin{proof}
 Seja $S \subset [n]$ subconjunto de classe $k$.
 Com efeito, o complementar de $S$ em relação a $[n]$ terá $n - k$ elementos,
 e para todo subconjunto de classe $k$ de $[n]$, haverá $n - k$ formas
 de criar um novo subconjunto de classe $k+1$. Pelo PFC, temos
 que
 \[
  \binom{n}{k+1} = (n - k) \binom{n}{k},
 \]
 e somando $\binom{n}{k}$ a ambos os membros, obtemos
 \[
  \binom{n}{k} + \binom{n}{k+1} = (n - k + 1) \binom{n}{k}
 \]
 Usaremos agora um argumento combinatório para demonstrar o resultado desejado.
 Visto que $[n + 1] = [n] \cup \{n+1\}$, o complementar de $S$ em relação a
 $[n+1]$ terá $n - k + 1$ elementos. Desse modo, para todo subconjunto de classe
 $k$ de $[n]$, existem $n - k + 1$ maneiras de obter um subconjunto de classe
 $k + 1$ de $[n+1]$, e, pelo PFC,
\[
 (n - k + 1) \binom{n}{k} = \binom{n+1}{k+1},
\]
o que prova o teorema.
\end{proof}
\end{thm:stifel}

Por fim, iremos demonstrar como computar o valor de um número binomial.
\begin{thm:valor binomial}
	\[
	    \binom{n}{k} = \frac{n!}{(n-k)!r!}
	\]
\begin{proof}
	Suponhamos que
    \[
        x = \binom{n}{k}
    \]
    e seja $C = \{C_1, C_2, \ldots, C_x\}$ os subconjuntos de classe $k$ de $[n]$.
    Iremos utilizar agora o argumento de contagem dupla sobre o número de arrajos
    de tamnaho $k$ de $[n]$. Com efeito, para cada $i \in [x]$, podemos obter $k!$ arranjos
    de $k$ elementos de $C_i$, totalizando
    \[
        x \cdot k!
    \]
    arranjos de $k$ elementos de $[n]$. Mas sabemos que
    \[
        \Arr{n, k} = \frac{n!}{(n-r)!},
    \]
    donde segue
   \begin{align*}
       x \cdot k! &= \frac{n!}{(n-r)!} \\
       x &= \frac{n!}{(n-r)!r!}
   \end{align*}
\end{proof}
\end{thm:valor binomial}

\subsection{Combinações}

Combinações são seleções de objetos de um conjunto finito, estando intimamente conectadas
com os binomiais vistos anteriormente. A definição a seguir formalizará a intuição
de seleção que estávamos usando até então, representando-as como multiconjuntos.

\begin{def:combinacao}
	Seja $A$ um conjunto de $n$ elementos. Uma combinação $s$ de $A$ é um
    multiconjunto não trivial na forma
   \[
        s: A \to \mathbb{N}
   \]
   Se $s$ é uma combinação binária, então ela é dita sem repetição.
\end{def:combinacao}

Toda combinação $s$ sem repetição de $A$ equivale a um subconjunto desse conjunto.
A intuição é simples: o multiconjunto é função booleana que indica se o elemento pertence
ou não ao subconjunto. Se $S \subset A$, então $S$ equivale a combinação $s$ que satisfaz:
\[
    \forall x \in A\, (x \in S \leftrightarrow s(x) = 1)
\]
Dessa maneira, toda combinação de $r$ elementos de $A$ equivale a um subconjunto de
classe $r$ de $A$, e o número de combinações de um conjunto finito de $n$ elementos será,
portanto, igual ao binomial
\[
    \binom{n}{r} = \frac{n!}{(n-r)!r!}.
\]
