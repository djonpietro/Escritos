\section{Tipo de Variáveis}

Ao trabalhar com dados, geralmente lidamos com dois grandes grupos de variáveis:
as quantitativas, que representam uma medida ou contagem, e as qualitativas,
que representam uma qualidade ou categoria de um indivíduo ou objeto. As
primeiras ainda se subdividem em contínuas -- a exemplo de salário, altura e
peso -- e discretas, como número de filhos ou idade. Já as qualitativas podem
ser nominanais, que representam categorias, ordinais, cujos valores podem
ser ordenados, ou dicotômicas, que só podem ter duas realizações: sucesso ou
fracasso.

\section{Distribuição de Frequências}

Para variáveis qualitativas, uma forma de resumir seus dados é através de
tabelas de frequência, que geralmente constam dois atributos: a frequência
absoluta de cada classe, e a proporção de cada classe em relação ao total
de observações. Para variáveis quantitativas, visando a geração de tabelas
de frequência, é geralmente necessária agrupar os valores em intervalos ou
faixas. A construção desses intervalos pode ser feita de diversas formas,
a depender do tipo de pesquisa sendo realizado. Contudo, recomenda-se
a construção de 5 a 15 grupos de mesma amplitude. Poucos grupos podem acabar
omitindo informações importantes, enquanto muitos grupos podem dificultar a
análise dos dados.

\section{Escalas de Medição}

Alternativamente às classificações apresentadas, podemos tipificar variáveis
através de escalas de medição de seus valores. As classes são similares às
classificações apresentadas anteriormente, sendo elas:

\begin{itemize}
  \item Escala Nominal -- para dados nessa escala, só podemos dizer que seus
    valores são diferentes de outros, sendo usada para categorizar indivíduos.
    Um exemplo simples é o sexo de uma pessoa: feminino ou masculino. Essa
    escala não suporta operações matemáticas, e uma medida de centralidade
    comum para resumir dados nessa escala é a moda.

  \item Escala Ordinal -- nessa escala, os valores podem ser ordenados, e podemos
    então dizer que, além de diferentes, um valor é maior do que o outro. Essa
    escala tem sua estrutura preservada por operações que preservem a ordem.
    Um exemplo de variável ordinal é o nível de escolaridade: fundamental, médio
    e superior. Medidas de tendência central comuns para dados nessa escala são a
    moda e a mediana.

  \item Escala Intervalar - essa escala possui uma origem arbitrária e necessita
    de uma unidade de medida, sendo um exemplo a temperatura de um ambiente.
    Podemos afirmar que valores são diferentes, maiores e quão maior em relação
    a outro, e transformações afim -- do tipo $ax + b$ -- não alteram a estrutura
    dessa escala. Podemos utilizar medidas como média, moda e mediana.

  \item Escala Razão -- nessa escala, existe uma origem absoluta, e podemos dizer que um
    valor é o quão maior do que outro através de razões. Um exemplo de variável
    nessa escala é o peso de um indivíduo. Transformações do tipo $ax$ preservam
    a estrutura dessa escala, e podemos utilizar medidas como média, moda e mediana.
\end{itemize}
