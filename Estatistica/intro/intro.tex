A estatítica é um ramo da matemática aplicada e da metodologia científica
que busca extrair, reduzir, analisar e modelar um conjunto de dados que
amostram uma população. Para isso, são usadas principalmente as ferramentas
teóricas providas pela Teoria das Probabilidades, pois entende-se que os
dados originam-se de fenômenos aleatórios, e, uma vez modelados, geralmente
dois objetivos visam ser cumpridos: estimar ou predizer o comportamento dos
dados no futuro, ou realizar inferência, que consiste em detectar os padrões
embutidos. A inferência pode ser feita por duas abordagens: a inferência
dedutiva, que aceita determinadas premissas para chegar às conclusões, ou
a inferência indutiva, que parte de casos particulares para generalizar.

Ainda sobre a inferência estatística, quando tenta-se modelar um determinado
conjunto de dados, tentamos identificar padrões e tendências de comportamento
através de modelos estatísticos. Supondo que temos um conjunto de dados $D$,
e indentificamos que conseguimos ajustar a eles um modelo $M$, teremos então que
\[
 D = M + R
\]
onde $R$ descreve a parte aleatória dos dados que não pode ser capturada pelo
modelo. Normalmente, busca-se que $R$ deve conter nenhuma suavidade, pois isso
implicaria em padrões que o modelo falhou em capturar, e mais suavização por
parte de $M$ será necessária.
