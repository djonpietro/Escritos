\section{Medidas de Posição}

Além das técnicas de tabelas de frequência ou métodos gŕaficos para resumir
informações de dados, há formas de resumir ainda mais os dados. Geralmente,
as mais comuns são medidas de centralidade: moda, média e mediana. Primeiramente,
vamos tratar do caso amostral.

\begin{def:moda}
  Se $X$ é uma variável aleatória com função de distribuição $f$, a moda de $X$,
  designidada por $\moda{X}$, é o valor $x$ que maximiza $f$.
  \[
    \moda{x} = \arg\max_{x} f(x)
  \]
\end{def:moda}
Desse modo, num conjunto de dados que amostra uma variável, podemos estimar
a moda apenas contando o valor mais frequente.

A média populacional de uma variável aleatória é o seu valor esperado $E[X]$,
e, dada uma amostra, a média amostral que estima a populacional é dada por:
\[
  \mu = \frac{1}{n} \sum_{i=1}^n x_i
\]
onde $n$ é o número de amostras obtidas. Por fim, a mediana de uma variável
aleatória é o valor $x$ tal que o quantil de probabilidade seja $1/2$, ou seja,
\[
  P(X \leq x) = 1/2
\]

Para estimar a mediana de $X$, suponha que $\vec v = (x_1, \dots, x_n)$
seja um vetor de realizações de $X$, e $k = \lfloor n/2 \rfloor$, então
a mediana $q$ será dada por:

\begin{enumerate}
  \item Se $n$ é impar, então faça a mediana igual a $x_k$.

  \item Se $n$ é par, então a mediana é dada por
    \[
      \frac{x_k + x_{k+1}}{2}
    \]
\end{enumerate}
Repare que, no caso de $n$ par, deve-se ter cuidado com variáveis não contínuas.

\section{Medidas de Dispersão}

Além da análise de centralidade e de valores típicos, pode ser que deseja-se
estudar a variabilidade de um conjunto de dados. Com isso, introduziremos
algumas medidas de dispersão.

O primeiro deles será o desvio médio, dado como
\[
  \dm{X} = \sum_{i = 1}^n \frac{ \left| x_i - \bar x \right| }{n}
\]
O segundo é a variância, que se assemelha ao desvio-médio, mas com a exceção
de que os desvios são elevados ao quadrado para se acentuar os desvios maiores.
\[
  \var{X} = \sum_{i = 1}^n \frac{ \left| x_i - \bar x \right|^2 }{n-1}
\]
A variância gera um valor cuja a unidade está elevada ao quadrado, sendo menos
interpretável. Por isso, utiliza-se também o desvio-padrão, dado como
\[
  \std{X} = \sqrt{ \sum_{i = 1}^n \frac{ \left| x_i - \bar x \right|^2 }{n-1} }
\]

Com base no que foi dito sobre o quadrado dos desvios, a variância e o
desvio-padrão são mais sensíveis a outliers, e a amplitude dos dados de modo
geral, sendo adequadas para quando a distribuição dos dados é aproximadamente
normal. O desvio-médio é uma medida menos sensível.

\section{Quantis}

O quantil $q$ de probabilidade $p$ é um número que satisfaz
\[
  \mathrm P(X \leq q) = p
\]
por exemplo: a mediana é o quantil de probabilidade de 50\%. Os quantis de
probabilidade de 0.25, 0.5 e 0.75 são, respectivamente, chamados de quartis,
sendo os mais utilizados em resumos de dados.

A estimação de um quantil, quando a função de distribuição não está disponível,
pode ser calculado da seguinte forma: dada a probabilidade $p$ acumulada pelo
quantil $q$, ordene os dados e faça $k = np$. Se $k \in \mathbb{N}$, então
o quantil é $x_k$, se não obtenha $k' = \lfloor k \rfloor$ e obtanha o
quantil como
\[
  \frac{x_{k'} + x_{k'+1}}{2}
\]
