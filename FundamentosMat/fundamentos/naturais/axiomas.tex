\section{Axiomas}

Antes de tudo, presumiremos, como princípios lógicos dados, o seguinte:
\begin{enumerate}
    \item A relação de igualdade denotada por $=$ que satisfaz
    \begin{itemize}
        \item Reflexividade: $x = x$.
        \item Simetria: se $x = y$, então $y = x$.
        \item Transitividade: se $x = y$ e $y = z$, então $x = z$
    \end{itemize}
    e o seu oposto, denotada por $\neq$.
	\item A existência de um conjunto de objetos denominados \textbf{Números Naturais},
	denotado por $\mathbb{N}$ e que satisfaz os seguintes axiomas:
	\begin{ax:primeiro natural}
	\label{ax:primeiro natural}
		O número $0$ é um natural.
    \end{ax:primeiro natural}
    \begin{ax:sucessor}
    \label{ax:sucessor}
        Existe uma função $S: \mathbb{N} \to \mathbb{N}$ denominada sucessor.
    \end{ax:sucessor}
    \begin{ax:sucessor 0}
    \label{ax:sucessor 0}
        Não existe um natural cujo sucessor é $0$.
        \[
            \forall x \in \mathbb{N},\ (S(x) \neq 0)
        \]
    \end{ax:sucessor 0}
    \begin{ax:unico antecessor}
    \label{ax:unico sucessor}
        Dois naturais com mesmo sucessor são iguais
        \[
            \forall x_1, x_2 \in \mathbb{N},\ (S(x_1) = S(x_2) \rightarrow x_1 = x_2)
        \]
    \end{ax:unico antecessor}
    \begin{ax:inducao}
    \label{ax:inducao}
        Se $\mathfrak{M}$ é um conjunto de números naturais e é satisfeito que
        \begin{enumerate}
            \item $1 \in \mathfrak{M}$
            \item $\forall x \in \mathbb{N},\ (x \in \mathfrak{M} \rightarrow S(x) \in \mathfrak{M})$
        \end{enumerate}
        então $\mathbb{N} = \mathfrak{M}$.
    \end{ax:inducao}
\end{enumerate}

O axioma \ref{ax:primeiro natural} define que os naturais tem pelo menos um elemento,
logo é não vazio. Em seguida, os axiomas \ref{ax:sucessor}, \ref{ax:sucessor 0} e
\ref{ax:unico sucessor} definem uma função injetora cujo único elemento fora da imagem
é o próprio $0$. Por fim, o axioma \ref{ax:inducao} enuncia o Princípio de Indução
aplicado aos números naturais.
