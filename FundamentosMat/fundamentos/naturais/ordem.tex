\section{Ordenação}

\begin{def:maior que}
\label{def:maior que}
    Sejam $x$, $y$ e $u$ naturais. Se \[x = y + u\]com $u \neq 0$, então dizemos
    que $x$ é maior que $y$, fato denotado por \[x > y,\]ou também que $y$ é menor do
    que $x$, representado por \[y < x.\]
\end{def:maior que}

\begin{prop: landau 10}
	Para quaisquer naturais $x$ e $y$, uma das situações deve ocorrer:
	\[
	    x = y \quad x < y \quad x > y
	\]
\begin{proof}
	Consequência direta da definição \ref{def:maior que} e do teorema \ref{prop:landau 9}.
\end{proof}
\end{prop: landau 10}

\begin{def:maior igual}
    Dizemos que $x$ é maior ou igual a $y$, denotado por $x \geq y$ se, e somente se,
    ocorre de
    \[
        x > y \lor x = y.
    \]
    Alternativamente, pode-se dizer que $y$ é menor ou igual a $x$, representado como
    $y \leq x$.
\end{def:maior igual}

\begin{prop: landau 15}
\label{prop:landau 15}
    A relação $<$ é transitiva nos naturais\footnote{A afirmação vale também para
    $>$, e a demonstração é análoga.}.
    \[
        \forall x, y, z \in \mathbb{N},\ ( x < y \land y < z \rightarrow x < z)
    \]
\begin{proof}
    Presumindo que $x < y$ e $y < z$, então existem naturais $u$ e $v$ não nulos tais que
    \[
        y = x + u \quad \text{e} \quad z = y + v.
    \]
    Com efeito,
    \begin{gather*}
        z = (x + u) + v \\
        z = x + (u + v),
    \end{gather*}
    donde segue que $x < z$.
\end{proof}
\end{prop: landau 15}

\begin{prop: landau 16}
\label{prop:landau 16}
    Para quaisquer naturais $x$, $y$ ou $z$, se tivermos que
    \[
        x \leq y \land y < z
    \]
    ou que
    \[
        x < y \land y \leq z
    \]
    então $x < z$
\begin{proof}
	A demonstração é trivial se a igualdade $x = y$ ou $z = y$ valem. No contrário,
	a proposição \ref{prop:landau 15} garante que a tese é válida.
\end{proof}
\end{prop: landau 16}

\begin{prop: landau 17}
    A relação de $\leq$ é transitiva nos naturais.
    \[
        \forall x, y, z \in \mathbb{N},\ (x \leq y \land y \leq z \rightarrow y \leq z)
    \]
\begin{proof}
	Se vale que $x = y$ e $y = z$, então a consequência é direta. Se não, a proposição
	\ref{prop:landau 16} garante a validade da tese.
\end{proof}
\end{prop: landau 17}

\begin{prop: landau 18}
	Para quaisquer naturais $x$ e $y$,
	\[
	    x + y > x
	\]
\begin{proof}
    Trivial, pois $x + y = x + y$.
\end{proof}
\end{prop: landau 18}

\begin{prop: landau 19}
\label{prop:landau 19}
    Sejam $x$, $y$ e $z$ naturais. Se
    \[
        (x > y) \lor (x = y) \lor (x < y),
    \]
    então
    \[
        (x + z > y + z) \lor (x + z = y + z) \lor (x + z < y + z),
    \]
    respectivamente.
\begin{proof}
\
    \begin{enumerate}
        \item Se $x > y$, então há um natural $u$ não nulo tal que
        \[
            x = y + u,
        \]
        e
        \[
            x + z = (y + u) + z = (y + z) + u
        \]
        Logo $x + z > y + z$.

        \item Se $x < y$, então a demonstração de que $x + z < y + z$ é análoga ao caso anterior.

        \item Se $x = y$, então é trivial que $x + z = y + z$.
    \end{enumerate}
\end{proof}
\end{prop: landau 19}

\begin{prop: landau 20}
    Sejam $x$, $y$ e $z$ naturais. Se
    \[
        (x + z > y + z) \lor (x + z = y + z) \lor (x + z < y + z),
    \]
    então
    \[
        (x > y) \lor (x = y) \lor (x < y),
    \]
    respectivamente.
\begin{proof}
	Para cada um dos três casos, podemos presumir a contrapositiva, e a conclusão seguirá
	da proposição \ref{prop:landau 19}.
\end{proof}
\end{prop: landau 20}

\begin{prop: landau 21}
	Para quaisquer naturais $x$, $y$, $z$ e $u$, se
	\[
	    x > y \land z > u,
	\]
	então
	\[
	    x + z > y + u
	\]
\begin{proof}
	Com efeito, existem $v, w \in \mathbb{N}$ não nulos tais que
	\[
	    x = y + v \quad \text{e} \quad z = u + w.
	\]
	Então,
	\[
	    x + z = (y + v) + (u + w) = (y + u) + (v + w)
	\]
	donde segue $x + z > y + u$.
\end{proof}
\end{prop: landau 21}

Os casos para $<$, $\leq$ e $\geq$ são análogos.

\begin{prop: landau 24}
	Para todo natural $x$, tem-se que $x \geq 0$
\begin{proof}
	Tem-se que $x = 0$ ou $x = 0 + x$.
\end{proof}
\end{prop: landau 24}
