% ------------------------- Capítulo 2 ---------------------------------

% ---------------------------- Axiomas --------------------------------
\newtheorem{ax:base2}{Axioma}[chapter]
\newtheorem{ax:supremo}[ax:base2]{Axioma}

% ------------------------- Definição  ---------------------------------
\newtheorem{def:base2}{Definição}[chapter]
\newtheorem{def:ordem}[def:base2]{Definição}
\newtheorem{def:modulo}[def:base2]{Definição}
\newtheorem{def:majorante}[def:base2]{Definição}
\newtheorem{def:supremo}[def:base2]{Definição}
\newtheorem{def:maximo}[def:base2]{Definição}

\newtheorem{def:interior}[def:base2]{Definição}
\newtheorem{def:vizinhanca}[def:base2]{Definição}
\newtheorem{def:aberto}[def:base2]{Definição}
\newtheorem{def:ponto de aderencia}[def:base2]{Definição}
\newtheorem{def:fecho}[def:base2]{Definição}
\newtheorem{def:fechado}[def:base2]{Definição}

% ------------------------- Proposições --------------------------------
\newtheorem{prop:base2}{Proposição}[chapter]
\newtheorem{prop:mult por 0}[prop:base2]{Proposição}
\newtheorem{prop:mult menos 1}[prop:base2]{Proposição}
\newtheorem{prop:fator 0}[prop:base2]{Proposição}
\newtheorem{prop:quadrados positivo}[prop:base2]{Proposição}
\newtheorem{prop:maior que zero positivo}[prop:base2]{Proposição}
\newtheorem{prop:propriedade ordem}[prop:base2]{Proposição}
\newtheorem{prop:monotonicidade ordem}[prop:base2]{Proposição}
\newtheorem{prop:desigualdade triangular}[prop:base2]{Proposição}
\newtheorem{prop:desigualdade delta}[prop:base2]{Proposição}
\newtheorem{prop:completude infimo}[prop:base2]{Proposição}

\newtheorem{prop:aberto}[prop:base2]{Proposição}
\newtheorem{prop:complementar aberto}[prop:base2]{Proposição}
\newtheorem{prop:fechado}[prop:base2]{Proposição}
% ------------------------- Teoremas   --------------------------------
\newtheorem{thm:base2}{Teorema}[chapter]
\newtheorem{thm:majorante naturais}[thm:base2]{Teorema}
\newtheorem{thm:arquimediana}[thm:base2]{Teorema}
\newtheorem{thm:intervalos encaixados}[thm:base2]{Teorema}
\newtheorem{thm:não enumeravel}[thm:base2]{Teorema}
\newtheorem{thm:intervalos encaixados e funcao}[thm:base2]{Teorema}
\newtheorem{thm:racionais e irracionais}[thm:base2]{Teorema}


% ------------------------- Corolários --------------------------------
\newtheorem{cor:base2}{Corolário}[chapter]
\newtheorem{cor:menos com menos}[cor:base2]{Corolário}
\newtheorem{cor:sub e add}[cor:base2]{Corolário}
\newtheorem{cor:regra de sinais1}[cor:base2]{Corolário}
\newtheorem{cor:regra de sinais2}[cor:base2]{Corolário}
\newtheorem{cor:quadrados iguais}[cor:base2]{Corolário}
\newtheorem{cor:mudanca de ordem}[cor:base2]{Corolário}
\newtheorem{cor:reais nao enumeraveis}[cor:base2]{Corolário}
\newtheorem{cor:0 1 não enumerável}[cor:base2]{Corolário}
