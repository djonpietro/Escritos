\section{Adição}

\begin{prop: landau 1}
    \label{prop:landau 1}
    Para todo $x$ e $y$ naturais vale que $x \neq y \rightarrow S(x) \neq S(y)$
\begin{proof}
	Se $S(x) = S(y)$, então pelo axioma \ref{ax:unico sucessor} teríamos $x = y$.
\end{proof}
\end{prop: landau 1}

\begin{prop: landau 2}
    Todo natural é diferente do seu sucessor
    \[
        \forall x \in \mathbb{N},\ (S(x) \neq x)
    \]
\begin{proof}
    Seja $\mathfrak{M}$ o subconjunto dos naturais para os quais a proposição vale.
	Temos que $0 \neq S(0)$ pelo axioma \ref{ax:sucessor 0}, logo $0 \in \mathfrak{M}$.
	Suponha que $x \neq S(x)$. Disso segue que $S(x) \neq S(S(x))$ pelo Teorema
	\ref{prop:landau 1}, e $S(x) \in \mathfrak{M}$, e pelo \ref{ax:inducao} temos que
	$\mathfrak{M} = \mathbb{N}$.
\end{proof}
\end{prop: landau 2}

\begin{prop: landau 3}
    Todo natural diferente de 0 tem um antecessor, isto é, um segundo elemento cujo
    sucessor é o primeiro.
    \[
        \forall x \in \mathbb{N},\ (x \neq 0 \rightarrow \exists u \in \mathbb{N},\ (S(u) = x))
    \]
\begin{proof}
    Seja $\mathfrak{M}$ o subconjunto dos naturais para os quais a proposição vale.
    É verdade que $0 \in \mathfrak{M}$ por vacuidade. Supondo que $x \in \mathfrak{M}$,
    então $S(S(u)) = S(x)$, e $S(x) \in \mathfrak{M}$. Logo $\mathfrak{M} = \mathbb{N}$
    pelo axioma \ref{ax:inducao}.
\end{proof}
\end{prop: landau 3}

O axioma \ref{ax:unico sucessor} garantirá que o antecessor de todo natural é único.

\begin{def:adicao}
    Chamaremos de adição um operação binária $+: \mathbb{N}^2 \to \mathbb{N}$ tal que
    $+(x, y) = x + y$ que satisfaz:
    \begin{enumerate}
        \item $\forall x \in \mathbb{N},\ (x + 0 = x) $
        \item $\forall x, y \in \mathbb{N},\ ( x + S(y) = S(x+y) )$
    \end{enumerate}
\end{def:adicao}

\begin{prop: landau 5}
    A operação de adição é associativa, isto é,
    \[
        \forall x, y, z \in \mathbb{N},\  (x + y) + z = x + (y + z)
    \]
\begin{proof}
	Fixando $x$ e $y$, seja $\mathfrak{M} \subset \mathbb{N}$ o conjunto dos naturais
	$z$ para quais a proposição vale.
	Com efeito
	\[
	    (x+y) + 0 = x + y = x + (y+0)
	\]
	e $0 \in \mathfrak{M}$. Supondo que $z \in \mathfrak{M}$, segue então
	\[
	    (x + y) + z = x + (y+z)
	\]
	Consequentemente,
	\[
	    (x + y) + S(z) = S((x+y)+z) = S(x + (y+z)) = x + S(y+z) = x + (y + S(z))
	\]
	e $S(z) \in \mathfrak{M}$. Logo a proposição vale para todos os naturais, isto é,
	$\mathbb{N} = \mathfrak{M}$.
\end{proof}
\end{prop: landau 5}

\begin{prop:soma 1}
	O sucessor de todo natural $x$ pode ser escrito como $S(x) = x + S(0)$.
\begin{proof}
	\[
	    x + S(0) = S(x + 0) = S(x)
	\]
\end{proof}
\end{prop:soma 1}

\begin{prop:comuta 0}
\label{prop:comuta 0}
	A adição de um natural $x$ e 0 é comutativa
	\[
	    x + 0 = 0 + x
	\]
\begin{proof}
	Seja $\mathfrak{M}$ o subconjuntos dos naturais para os quais isso vale.
	Com efeito, se $x = 0$, então
	\[
	    x + 0 = 0 + 0 = 0 + x.
	\]
	Supondo que vale para um natural $x$, então
	\[
	    S(x) + 0 = S(x) = S(x + 0) = S(0 + x) = 0 + S(x),
	\]
	e a proposição vale para $S(x)$, logo $\mathfrak{M} = \mathbb{N}$.
\end{proof}
\end{prop:comuta 0}

\begin{prop:comuta 1}
    A adição de um natural $x$ com $S(0)$ é comutativa.
\begin{proof}
    Seja $\mathfrak{M}$ o subconjuntos dos naturais para os quais isso vale.
	Se $x = 0$, então
	\[
	    S(0) + x = S(0) = S(0 + 0) = 0 + S(0) = x + S(0).
	\]
	Suponha que vale para $x \in \mathbb{N}$, ou seja, $x + S(0) = S(0) + x$.
	Disso temos que
	\[
	    S(x) + S(0) = x + S(0) + S(0) = S(0) + x + S(0) = S(0) + S(x)
	\]
	Logo, a proposição vale para $S(x)$, e para todos os naturais.
\end{proof}
\end{prop:comuta 1}

\begin{prop: landau 6}
    A operação de adição é comutativa, isto é,
    \[
        \forall x, y \in \mathbb{N},\  x + y  = y + x
    \]
\begin{proof}
	Fixe $y$ e seja $\mathfrak{M}$ o conjunto dos naturais para os quais a proposição vale.
	Supondo que $x = 0$, então a comutação segue de \ref{prop:comuta 0}.
	Suponha que vale para $x$, e teremos então:
\begin{align*}
	x + y &= y + x\\
	S(x + y) &= S(y + x)\\
	x + S(y) &= y + S(x)\\
	x + y + S(0) &= y + S(x)\\
	x + S(0) + y &= y + S(x)\\
	S(x) + y &= y + S(x)
\end{align*}
Conclui-se que $\mathfrak{M} = \mathbb{N}$, já que $S(x) \in \mathfrak{M}$.
\end{proof}
\end{prop: landau 6}

\begin{prop: landau 7}
\label{prop:landau 7}
    Para quaisquer $x$ e $y$ naturais e com $x \neq 0$, temos que
    \[
        y \neq x + y
    \]
\begin{proof}
	Fixe $x$, e seja $\mathfrak{M}$ o conjuntos dos naturais para os quais a proposição
	vale. Se $y = 0$, então vale que $x \neq y$ por hipótese, e $0 \in \mathfrak{M}$.
	Suponha que a proposição vale para $y$, e $y \neq x + y$. Logo
	\begin{align*}
		S(y) &\neq S(x + y)\\
		S(y) &\neq x + S(y)\\
    \end{align*}
    e $S(y) \in \mathfrak{M}$. Portanto $\mathfrak{M} = \mathbb{N}$.
\end{proof}
\end{prop: landau 7}

\begin{prop: landau 8}
    Se \[ y \neq z,\]então\[x + y \neq x + z\]
\begin{proof}
    Fixe $y$ e $z$, e seja $\mathfrak{M}$ o conjunto dos naturais para os quais a proposição
    vale. Com efeito
    \begin{gather*}
        y \neq z\\
        0 + y \neq 0 + z
    \end{gather*}
    e temos $0 \in \mathfrak{M}$. Agora suponha que
    \[
        y \neq z \rightarrow x + y \neq x + z.
    \]
    Presumindo então que $y \neq z$, segue:
    \begin{gather*}
        x + y \neq x + z\\
        S(x + y) \neq S(x + z)\\
        S(x) + y \neq S(x) + z\\
    \end{gather*}
    e $S(x) \in \mathfrak{M}$. Logo $\mathfrak{M} = \mathbb{N}$.
\end{proof}
\end{prop: landau 8}

\begin{prop: landau 9}
\label{prop:landau 9}
	Sejam $x$ e $y$ naturais com $x \neq 0$. Exatamente uma das proposições abaixo deve
	ocorrer:
	\begin{enumerate}
	    \item $x = y$
		\item $\exists u,\ ( x = y + u)$, com $x$ e $y$ não ambos nulos.
		\item $\exists v,\ ( y = x + v)$, com $x$ e $y$ não ambos nulos.
    \end{enumerate}
\begin{proof}
    Nenhuma das proposições podem ocorrer simultaneamente dois a dois visto a proposição
    \ref{prop:landau 7}. Fixando $x$, seja então $\mathfrak{M}$ o conjunto dos naturais em
    que ocorre uma, e somente uma, das proposições listadas.

    Se $y = 0$, então
    \[
       x = x + 0 = x + y
    \]
    que é o caso 2 com $u = x$, e $0 \in \mathfrak{M}$.
    Suponha que $y \in \mathfrak{M}$, ou seja, que exatamente uma das porposições valem.
    Temos três casos
    \begin{enumerate}
        \item Se $x = y$, então $S(y) = S(x)$ e $S(x) = x + S(0)$. Fazendo $v = S(0)$,
        temos
        \[
            S(y) = x + v
        \]
        e a proposição 3 vale para $S(y)$.

        \item Se existe natural $u$ tal que $x = y + u$, então temos outros dois casos
        \begin{enumerate}
            \item $u = 0$, e $x = y$, voltando ao caso 1.
            \item $u \neq 0$, e há natural $w$ tal que $S(w) = u$. Segue então que
            \[
                x = y + S(w) = S(y) + w
            \]
            e a proposição 2 vale também para $S(y)$.
        \end{enumerate}

        \item Se existe natural $v$ tal que $y = x + v$, então
        \[
            S(y) = S(x + v) = x + S(v)
        \]
        e a proposição 3 vale para $S(y)$.
    \end{enumerate}
    Visto os casos, podemos dizer então que $S(y) \in \mathfrak{M}$, $\mathfrak{M} = \mathbb{N}$.
\end{proof}
\end{prop: landau 9}
