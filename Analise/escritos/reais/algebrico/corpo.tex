\section{Corpo dos Números Reais}

% \subsection{Definição de Corpo}

% Na Álgebra Abstrata, chama-se grupo toda estrutura composta por um conjunto, e uma operação
% que satisfaz as propriedades de fechamento, associatividade, elemento neutro e elemento
% inverso. Quando a operação do grupo é comutativa, ele é chamado de abelino. Por exemplo,
% os inteiros com a operação de adição são um grupo abeliano. Por sua vez, os anéis
% são grupos abelianos dotados de uma segunda operação (normalmente identificada como
% multiplicação) que é associativa, possui elemento neutro e distribui-se sobre a sobre
% a primeira operação (geralmente deominada adição). Finalmente, os corpos são estruturas
% algébricas que consistem de anéis em que todo elemento não nulo possui elemento inverso
% relativo à operação de multiplicação.

% Os números reais foram um corpo com as operações de $+$ e $\cdot$. Ademais, $(\mathbb{R},
% +, \cdot)$ consiste dum corpo ordenado pela relação $<$ (ou também $\leq$), pois, além
% de ser uma ordem total, é satisfeito que
% \begin{itemize}
% 	\item $\forall a, b, c \in \mathbb{R}, (a < b  \rightarrow a+c < b+c)$
% 	\item $\forall a, b \in \mathbb{R}, (0 < a \land 0 < b \rightarrow 0 < ab)$
% \end{itemize}

\subsection{Propriedades Algébricas Fundamentais}

A seguir, presumiremos as seguintes propriedades das operações algébricas sobre os
números reais $a$, $b$, $c$ e $d$:

\begin{enumerate}
    \item Associatividade
    \begin{center}
        \begin{multicols}{2}

            $(a + b) + c = a + (b + c)$

            $(a \cdot b) \cdot c = a \cdot (b \cdot c)$
        \end{multicols}
    \end{center}

    \item Comutatividade
    \begin{center}
        \begin{multicols}{2}
            $a + b = b + a$

            $a \cdot b = b \cdot a$
        \end{multicols}
    \end{center}

    \item Elemento neutro
    \begin{center}
        \begin{multicols}{2}
            $a + 0 = a$

            $a \cdot 1 = a$
        \end{multicols}
    \end{center}

    \item Elemento Inverso
    \begin{center}
        \begin{multicols}{2}
            $a + (-a) = 0$

            $a \neq 0 \land a \cdot \frac{1}{a} = 1$
        \end{multicols}
    \end{center}

    \item Distributiva
    \begin{center}
        \begin{multicols}{2}
            $a \cdot (b + c) = ab + ac$

            $a \cdot (b - c) = ab - ac$
        \end{multicols}
    \end{center}
    \item Propriedades de Igualdade: se $a = b$, então
    \begin{center}
        \begin{multicols}{2}
            $a + c = b + c$

            $a \cdot c = b \cdot c$

            $a - c = b - c$

            $\displaystyle\frac{a}{c} =\frac{b}{c}$
        \end{multicols}
    \end{center}
\end{enumerate}

\begin{prop:mult por 0}
    Para todo real $a$ tem-se $a \cdot 0 = 0$
\begin{proof}
    Presuma que $a \cdot 0 = k$. Com efeito
    \begin{gather*}
        a (0 + 0) = a \cdot 0\\
        k + k = k\\
        k + k - k = k - k\\
        k = 0
    \end{gather*}
    Conclui-se que $a \cdot 0 = 0$
\end{proof}
\end{prop:mult por 0}

\begin{prop:mult menos 1}
\label{prop:mult menos 1}
    Para todo real $a$ vale que $a \cdot (-1) = -a$
\begin{proof}
    É verdade que $a \cdot (1 + (-1)) = 0$, donde segue
    \begin{gather*}
        a + a \cdot (-1) = 0\\
        a + a \cdot (-1) - a = 0 - a\\
        a \cdot (-1) = -a
    \end{gather*}
\end{proof}
\end{prop:mult menos 1}

\begin{cor:menos com menos}
	Para todo real $b$, tem-se que $-(-b) = b$
\begin{proof}
    Na proposição \ref{prop:mult menos 1}, substitua $a$ por $-b$, obtendo
    \[
        (-b) \cdot (-1) = -(-b)
    \]
    Mas pelo elemento neutro da adição temos que
    \[
        (-b) + (-(-b)) = 0
    \]
    do que segue
    \begin{gather*}
        b + (-b) + (-(-b)) = b\\
        -(-b) = b
    \end{gather*}
\end{proof}
\end{cor:menos com menos}

\begin{cor:sub e add}
	Para todo real $a$ conclui-se que $a + (-a) = a - a = 0$
\begin{proof}
	\begin{gather*}
        (-1) \cdot a = -a\\
        a + (-1) \cdot a = a-a\\
        a + (-a) = a-a
	\end{gather*}
	Logo, $a-a = 0$
\end{proof}
\end{cor:sub e add}

\begin{cor:regra de sinais1}
    Para todos reais $a$ e $b$ tem-se $a \cdot (-b) = - (ab)$
\begin{proof}
	Segue da proposição \ref{prop:mult menos 1}, da associatividade e comutatividade da
	multiplicação.
\end{proof}
\end{cor:regra de sinais1}

\begin{cor:regra de sinais2}
	Para quaisquer reais $a$ e $b$ tem-se que $(-a)(-b) = ab$
\begin{proof}
	\begin{align*}
		(-a)(-b) &=\\
		(-1)(-1)(ab)&=\\
		-(-1)(ab)&=ab
	\end{align*}
\end{proof}
\end{cor:regra de sinais2}

\begin{prop:fator 0}
	Para quaisquer reais $a$ e $b$, se $ab = 0$, então $a = 0$ ou $b = 0$
\begin{proof}
	Suponha por contradição que $ab = 0$, $a \neq 0$ e $b \neq 0$. Então
	\begin{gather*}
		\frac{ab}{b} = \frac{0}{b}\\
		a = 0
\end{gather*}
    o que é absurdo, logo a proposição vale.
\end{proof}
\end{prop:fator 0}

\begin{cor:quadrados iguais}
    Sejam $a, b \in \mathbb{R}$. Se $a^2 = b^2$, então $a = \pm b$
\begin{proof}
	Com efeito, uma vez que
	\[
	    (a+b)(a-b) = a^2 - b^2 = 0
	\]
	então
	\[
	   (a+b) = 0 \quad \text{ ou } \quad (a-b) = 0
	\]
	donde se conclui que $a = -b$ ou $a = b$ respectivamente.
\end{proof}
\end{cor:quadrados iguais}
