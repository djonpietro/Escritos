\section{Conjuntos Enumeráveis}

\subsection{Boa Ordenação}

Como citado brevemente quando discutimos os números naturais, para alguns conjuntos
é possível explicitar uma ordenação total de seus elementos de modo que qualquer
subconjunto possua um menor elemento. Essas estruturas -- o conjunto dotado da relação
de ordem --  são chamadas de boas ordenações, ou conjuntos bem-ordenados.

Na Teoria Axiomática de Conjuntos de Zermelo-Fraenkel com o Axioma da Escolha (ZFC),
toda boa ordem está associada a um número ordinal, que caracteriza a ordem do conjunto.
Os números 1, 2, 3, etc., são números ordinais finitos, pois caracterizam a ordem
de conjuntos finitos; por exemplo: a classe dos conjuntos de ordem 3 são aqueles que, uma
vez ordenados, terão um primeiro, um segundo e um terceiro elemento. No entanto, conjuntos
infinitos também estão associados a ordinais, a exemplo de $\mathbb{N}$ associado ao
ordinal $\omega$, dito o primeiro ordinal transifnito.

Por mais contra-intuitivo que seja, ainda podemos contruir ordens maiores do que $\omega$.
Por exemplo, se ordenamos os naturais de modo que primeiro listamos todos os pares e todos
os ímpares em seguida, então o ordinal dessa ordenação é $\omega + \omega$. Seguindo esse
raciocínio, ainda seria possível definir ordens cada vez maiores com ordinais $\omega^2$,
$\omega^\omega$, ou $\omega^{\omega^\omega}$.

O Axioma de Escolha da ZFC pode provar o Teorema da Boa Ordenação, o qual diz que
qualquer conjunto pode ser bem ordenado. Todavia, o resultado apenas enuncia a existência
de uma boa ordenação, mas não induz qualquer método para a obter.

\subsection{Conjuntos Finitos e Enumeráveis}

A grandeza de uma ordem não implica na quantidade de elementos que há
originalmente nos conjuntos. Como visto, a depender da ordenação que damos aos naturais,
podemos obter ordens com ordinais $\omega$ ou $\omega + \omega$, no entato, os naturais
continuam tendo a mesma quantidade de elementos nos dois casos. A quantidade de membros
num conjunto é formalizada pela ideia de cardinal, definida como o menor ordinal ao qual
um conjunto pode ser associado por uma boa ordenação. Dois conjuntos terão mesma
cardinalidade (ditos equipotentes) se entre eles existe uma bijeção.

Conjuntos finitos são aqueles que possuem uma bijeção com algum subconjunto próprio dos
naturais. Desse modo, há uma função bijetora $f$ denominada função de contagem que associa
cada elemento deles a um termo numa sequência de naturais de $1$ até $n$, em que $n$ é
identificado como a cardinalidade do cojunto.

Uma classe de conjuntos que inclui os finitos são os conjuntos enumeráveis. Esses conjuntos
consistem naqueles para os quais existe uma bijeção entre eles e qualquer subconjunto dos
naturais (incluindo o próprio). Dessa forma, um conjunto é enumerável é menor ou tão grande
quanto o conjunto dos naturais.

\subsection{Conjuntos Não Enumeráveis e os Reais}

Georg Cantor (1845-1918), matemático alemão que inaugurou a Teoria dos Conjuntos,
provou que a cardinalidade dos naturais, dos inteiros e dos racionais são iguais, ou seja,
são todos enumeráveis. Ademais, provou que existem conjuntos maiores do que os naturais,
Seu argumento era simples, mas elegante: crie uma lista de todos os números entre 0 e 1 com
suas respresentação decimais, e construa um número no mesmo intervalo tal que o enésimo
decimal do número seja diferente do decimal na mesma posição do enésimo termo na lista.
Conclui-se que o número construído não fora listado, e não possível listar todos os números
entre 0 e 1. Esse argumento é conhecido como argumento da diagonalização.
Tais conjuntos que não podem ser colocados em correspondência um para um com os naturais
são chamados de não enumeráveis.

Na história da matmática, um dos problemas mais famosos foram os dos Incomensuráveis,
que eram números que não podiam ser medidos, isto é, não poeria ser dados na forma de uma
razão entre inteiros $a/b$, como por exemplo a a diagonal do quadrado
de lado unitário. Modernamente, os Incumensuráveis foram rebatizados de irracionais, e,
ao longo do tempo, mostrou-se que diversos números são irracionais, como $\sqrt{2}$,
$\pi$ e a constante euleriana $e$. São justamente os números irracionais que completam
as lacunas deixadas pelos racionais, e, juntos, foram o conjunto dos números Reais.
Como visto, os reais possuem cardinalidade maior do que dos números naturais.
