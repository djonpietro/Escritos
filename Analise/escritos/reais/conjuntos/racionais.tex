\section{Racionais}

\subsection{Frações}

Uma extensão dos naturais e inteiros são os números racionais. A sua construção
se dá de maneira análoga aos dos inteiros, tal que todo racional é uma classe de
equivalência de um par $(a, b)$, com $a$ e $b$ inteiros, e $b$ não nulo, sendo
representado pela fração
\[
    \frac{a}{b}.
\]
Disso, temos que o racional $3$ pode tem como representantes $3 / 1$, $15 / 5$ ou
$-6 / (-2)$, enquanto o racional $1/2$ tem como representantes $2/4$ ou $\frac{-10}{-20}$.

A motivação para os racionais está no fato que a operação de divisão não é fechada nos
inteiros, pois $4 \div 2$ é inteiro, mas $9 \div 4$ não. Para eles, também estão bem
definidas as quatro operações básicas: adição, multiplicação, subtração e divisão. Além
disso, os racionais são dotados de um propriedade fundamental, tal que, sejam $a/b$ e
$c/d$ racionais e
\[
   \frac{a}{b} = \frac{c}{d},
\]
então
\[
    ad = bc
\]

\subsection{Representação Decimal}

Os racionais admitem uma representação em termos posicionais, isto é, a cada algarismo
é atribuído um valor relativamente a sua posição no número. Acontece que o mesmo já
ocorria para os naturais e inteiros, pois cada posição à esquerda siginifica um aumeto
de dez vez no valor absoluto do algarismo. Por exemplo
\[
    321 = 3 \cdot 10^2 + 2 \cdot 10^1 + 1 \cdot 10^0.
\]
a representação decimal dos recionais é análoga, com a diferença que os algarismos
podem ter seus valores diminuidos por potências cada vez maiores de dez. Exemplo:
\[
    24,35 = 2 \cdot 10^1 + 4 \cdot 10^0 + 3 \cdot 10^{-1} + 5 \cdot 10^{-2}.
\]

A partir da representação em fração de um racional, pode-se obter sua presentação decimal
através de uma versão adaptada do algoritmo de divisão. No entanto, vale ressaltar que
essa representação pode não ser finita e, especificamente nos casos de representação
infinita, resulta nas chamadas dízimas periódicas. Nelas, os algarismos nas casas
relativas às potências negativas, possuem um padrão de repetição previsível.
