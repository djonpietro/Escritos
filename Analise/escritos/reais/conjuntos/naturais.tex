\section{Naturais e Inteiros}

\subsection{Axiomas de Peano}

O conjunto dos números naturais, denotado por $\mathbb{N}$, é tradicionalmente definido
por um esquema de axiomas que especificam sua estrutura e natureza: os Axiomas de Peano,
propostos pelo matemático italiano Giuseppe Peano em 1889. Os axiomas podem ser divididos
em duas coleções, aqueles que definem o comportamento de relação de igualdade,
estebelecendo suas propriedades de reflexividade, simetria e transitividade, e aqueles
que falam sobre os naturais em si, postulando que:

\begin{enumerate}
	\item o número $0$ é um natural.
	\item Existe uma função $S: \mathbb{N} \to \mathbb{N}$ a ser denominada sucessor.
	\item Não existe natural cujo sucessor é $0$.
	\item Naturais com os mesmos sucessores são iguais
	\item Se um subconjunto $\mathfrak{M}$ dos naturais inclui o $0$, e presumindo que
	$x \in \mathfrak{M}$, pode-se concluir que $S(x) \in \mathfrak{M}$, então $\mathfrak{M}$
	inclui todos os naturais.
\end{enumerate}

O primeiro axioma diz que os Naturais são um conjunto não vazio, pois possuem ao menos
um elemento. Os axiomas de 2 a 4, afirmam a existência de uma aplicação injetora $S$
cuja imagem não inclui o $0$. O axioma 5 também é conhecido como Princípio de Indução,
afirmando que, se dado conjunto inclui o $0$, e é possível aplicar a
função $S$ sucessivamente de modo que os novos elementos também estejam nele,
então o conjunto trata-se do próprio conjunto dos naturais.

A partir dos axiomas, é possível demonstrar a existência e unicidade das operações de
adição e multplicação dos naturais, satisfazendo as propriedades típicas, como
associatividade e comutatividade. A partir deles também define-se uma ordenação por
meio da relação de ``maior que'', representada por $<$, que satisfaz a definição
de relação de ordem estrita total, isto é
\begin{itemize}
	\item Assimétrica -- $\forall x, y,\ (x < y \rightarrow y \cancel{<} x )$
	\item Totalidade -- $\forall x, y,\ ( x < y \lor y < x)$
	\item Não-Reflexiva -- $\forall x,\ ( x \cancel{<} x )$
	\item Transitiva -- $\forall x, y, z,\ ( x < y \land y < z \rightarrow x < z )$
\end{itemize}
Outras ordens também podem ser definidas, como a parcial, que exclui a propriedade
de assimetria e inclui a reflexiva, isto é, todo elemento está relacionado consigo
mesmo. Uma vez que os naturais possuem uma ordem total e, nessa ordem, há um menor
elemento -- o número $0$ -- então dizemos que os naturais estão bem ordenados.
Formalmente, um conjunto é bem ordenado se para ele existe uma ordem total e qualquer
um de seus subconjuntos possui um menor elemento nessa ordem.

\subsection{Príncípio de Indução}

O axioma 5, como mencinado, define o famoso Princípio de Indução. No entanto, esse
enunciado possuem outras formulações que tornam mais prático o seu uso como uma técnica
de demonstração para provar propreidades sobre elementos que satisfazem uma definição
recursiva.

Recursão é uma forma de definição que envolve uma coleção de casos bases e regras
de construção para criar novos casos, chamados de recursivos. Por exemplo, nos Axiomas
de Peano o único caso base é o $0$, e os casos recursivos são aqueles aplicando-se a
função sucessor sucessivamente: $S(0) = 1$, $S(S(0)) = 2$, e assim em diante. Uma recursão
bem definida é aquela que, para qualquer caso recursivo, é possível retornar aos casos
bases com base nas regras de construção. Logo, definições que precisam passar por
infinitos casos antes de chegar num caso base ou circulares não são recursões bem
definidas.

O Princípio de Indução Finita, por sua vez, é normalmente aplicado sobre uma definição
recursiva para provar uma propriedade $\varphi$ que vale para todos os casos. Para usá-la,
deve-se provar que $\varphi$ vale para todos os casos bases e, em seguida, supondo que
vale para um caso recursivo, mostra que deve ser verdade para o próximo caso obtido pelas
regras de construção. Essa formulação do princípio é mais geral (mas não mais forte) que
a dada pelo axioma 5, pois pode ser mais facilmente apicada a qualquer conjunto que possa
ter seus elementos definidos recursivamente em função de alguns elementos ditos base.

\subsection{Números Inteiros}

A primeira extensão dos naturais é a dos chamados números inteiros, denotados por $\mathbb{Z}$.
A necessidade desses números surge no fato de que os naturais não são fechados sobre subtração;
temos que $ 5 - 3$ é um natural, mas $3 - 5$ não é, por exemplo. Com isso, uma construção
comum para os inteiros consiste em definí-los enquanto pares ordenados $(a, b)$, com
$a$ e $b$ naturais, denotando
\[
    a - b.
\]
Repare que infinitos pares podem representar o mesmo inteiro, e, por isso, dizemos que
cada inteiro é uma classe de equivalência desses pares. Exemplo: o inteiro $3$ pode ser
rperesentado como $5 -2$, $7 -4$ e $10- 7$, enquanto que o inteiro $-2$ possui as
representações equivalentes de $0 - 2$, $4 - 6$ e $8 - 10$.

Para os inteiros também estão bem definidas as operações de adição, subtração e multiplicação.
O conjunto $\mathbb{Z}$ também possui uma ordem total -- tanto parcial quanto estrita -- mas
não são bem ordenados, pois nem todo subconjunto possui um menor elemento, como por exemplo
\[
    \{x \in \mathbb{Z} \mid x < 2\}
\]
o qual incluirá o $1$, $0$ e todos os números negativos.
