\section{Topologia dos Números Reais}

\subsection{Conjuntos Abertos}

Os pontos num intervalo real na forma $(\alpha, \beta)$ possuem a propriedade de que
qualquer ponto está envolto por infinitos pontos ainda dentro do intervalo nos dois
sentidos. É essa noção que iremos formalizar como \textbf{pontos interiores}.

\begin{def:interior}
    Seja $X$ um subconjunto de $\mathbb{R}$. O conjunto dos pontos interiores de $X$,
    denotado por $\inter X$, é aquele que reúne todos os pontos $a \in \mathbb{R}$ para
    os quais existe $\epsilon > 0$ tal que
    \[
        I = (a - \epsilon, a + \epsilon) \subset X.
    \]
\end{def:interior}

Chamaremos um conjunto de aberto quando ele for igual ao conjunto dos seus pontos
interiores. Dessa forma, qualquer ponto do conjunto está envolto, nos dois sentidos,
de infinitos outros pontos também interiores ao conjunto.

\begin{def:aberto}
	Um conjunto $X \subset \mathbb{R}$ é dito aberto se, e somente se, $X = \inter X$.
\end{def:aberto}

\noindent
Uma nota importante é que, por não ter elementos, o conjunto vazio é aberto por vacuidade.

A vizinhança $V$ de um ponto $a$ é qualquer aberto $A \subset V$ que tenha o $a$ como
membro.
\begin{def:aberto}
	Seja $a \in \mathbb{R}$ e $V \subset \mathbb{R}$. Diz-se que $V$ é vizinhança para
	$a$ se existe aberto $A \subset V$ tal que $x \in A$.
\end{def:aberto}
\noindent
Uma consequência direta dessa definição é que todo conjunto aberto é uma vizinhança para
todos os seus pontos.

Mostremos, agora, que a interseção de quaisquer abertos é um aberto, e que a reunião
de uma família potencialmente infinita de abertos também é um aberto.

\begin{prop:aberto}
    Seja $\mathcal{F}$ uma família de conjuntos abertos. É
    verdade que
    \begin{enumerate}
        \item $\forall A_1, A_2 \in \mathcal{F},\ (\, A_1 \cap A_2 \text{ é aberto}\,)$
        \item $\bigcup_{A \in \mathcal{F}} A$ é aberto
    \end{enumerate}
\begin{proof}
    (1) A conclusão é direta se $A_1 \cap A_2 = \varnothing$. Do contrário, existem
    $\epsilon_1 > 0$ e $\epsilon_2 > 0$ tais que
    \[
        (a - \epsilon_1, a + \epsilon_1) \subset A_1 \quad \text{ e } \quad
        (a - \epsilon_2, a + \epsilon_2) \subset A_2.
    \]
    Sem perda de generalidade, considere que $\epsilon_1 < \epsilon_2$, levando a que
    \[
        (a - \epsilon_1, a + \epsilon_1) \subset (a - \epsilon_2, a + \epsilon_2)
    \]
    Visto que o intervalo $(a - \epsilon_1, a + \epsilon_1)$ está contido tanto em $A_1$
    quanto em $A_2$, então esse intervalo está contido em $A_1 \cap A_2$, e $a$ é, pois,
    um ponto interior desse último.

    (2) Se $a \in A$ e $A \in \mathcal{F}$, então $A$ é vizinhança de $a$. Desse modo,
    na reunião dos abertos de $\mathcal{F}$, $A$ permanecerá uma vizinhança de $a$. Como
    vale para todo ponto em algum aberto na reunião, a reunião de abertos também será um
    aberto.
\end{proof}
\end{prop:aberto}

Uma sutileza na proposição é que somente a interseção de um número finito de abertos foi
demonstrada ser aberto, o que não necessariamente vale quando é feita a interseção de
infinitos abertos. Por outro lado, a união de um número potencialmente infinito de abertos
é sim um aberto.

\subsection{Conjuntos Fechados}

Um ponto $a$ é de aderência no conjunto $X$ se ele qualquer intervalo centrado nele
posssui pontos em comum com $X$.

\begin{def:ponto de aderencia}
    Seja $a \in \mathbb{R}$ e $X \subset \mathbb{R}$. O ponto $a$ é aderente a $X$,
    ou é ponto de aderência do conjunto, se, e somente se, toda vizinhança $V$ que contém
    $a$ é satisfeito que
    \[
        V \cap X \neq \varnothing.
    \]
\end{def:ponto de aderencia}
\noindent
A noção de fecho dum conjunto é dada como a reunião de todos os seus pontos de aderência.
\begin{def:fecho}
	Chamamos fecho dum conjunto $X \subset \mathbb{R}$, denotado por $\overline{X}$,
	o conjunto dos pontos de aderência de $X$.
\end{def:fecho}

Como nossa definição de aderente não proibiu intervalos degenerados, então temos que
qualquer ponto de $x$ será aderente a $X$, de modo a implicar que $X \subset \overline{X}$.
Dessa última, também conlui-se que, se $X \subset Y$, então $\overline{X} \subset \overline{Y}$
por transitividade da relação de inclusão. O conceito de fecho também serve para definir
conjuntos fechados.

\begin{def:fechado}
    Um conjunto $X \subset \mathbb{R}$ é dito fechado se, e somente se, $X = \overline{X}$.
\end{def:fechado}

\begin{prop:complementar aberto}
    Um conjunto é fechado se, e somente se, seu complementar é aberto.
\begin{proof}
	Suponha que $X$ é fechado e seja $A = \mathbb{R} \setminus X$ seu complementar.
	Com efeito, se $a \in A$, então $a \notint X$, o que siginifica que $a$ não é
	ponto de aderência de $X$, e existe vizinhaça $V$ tal que $a \in V$ e
	$V \cap X = \varnothing$. É verdade que $V \subset A$, pois, senão, teríamos que
	algum ponto de $V$ pertenceria a $X$, já que esse é complementar de $A$. Portanto,
	como todo ponto em $A$ admite uma vizinhança, então $A$ é aberto.

	Reciprocamente, se $A$ é aberto, então para todo ponto $a \notin A$, logo $a \in X$,
	existe uma vizinhança $V$ que não está contida em $A$ que possui, pois,
	interseção com $X$. Por definição, $a$ é ponto de aderência em $X$ que também é
	membro desse, e $X$ é fechado.
\end{proof}
\end{prop:complementar aberto}

\begin{prop:fechado}
    Seja $\mathcal{F}$ uma família arbitrária de conjuntos fechados. É verdade que
    \begin{enumerate}
    	\item $\forall X_1, X_2 \in \mathcal{F},\ (X_1 \cup X_2 \text{ é fechado })$
    	\item $\bigcap_{X \in \mathcal{F}} X$ é fechado.
    \end{enumerate}
\begin{proof}
	(1) Suponha que $x$ seja aderente a $U = X_1 \cup X_2$. Nesse caso, toda vizinhança
	$V$ de $x$ contém pontos de $U$, o que significa que $V$ contém pontos de $X_1$ ou
	pontos de $X_2$. Por consequência, $x$ deve ser aderente a $X_1$ ou $X_2$, e como
	ambos são fechados, então $x$ é membro de algum deles, logomembro de $U$, e $U$ é fechado.

	(2) Se $x$ é aderente a $I = \bigcap{X \in \mathcal{F}} X$, e $V$ é vizinhança para
	$x$, então $V$ não pode ser disjunto de nenhum fechado de $\mathcal{F}$, de sorte que
	$x$ é aderente a todo conjunto desse último. Como são todos fechados, então $x$
	pertence a todo $X \in \mathcal{F}$, e $x \in I$.
\end{proof}
\end{prop:fechado}
