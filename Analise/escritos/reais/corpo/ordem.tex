\section{Ordenação e Desigualdades}

\subsection{Ordem}

Presumiremos daqui em diante a existência de um conjunto $\mathbb{P} \subset \mathbb{R}$
dotado das seguintes propriedades, supondo $a, b$ e $c \in \mathbb{P}$:

\begin{enumerate}
    \item Fechamaento sob $+$ e $\cdot$
    \[
        \forall a, b \in \mathbb{P},\ (\, a + b \in P \text{ e } a \cdot b \in P\,)
    \]
    \item Tricotomia
    \[
        \forall a \in \mathbb{R},\ (\,a \in \mathbb{P} \lor -a \in \mathbb{P} \lor a = 0\,)
    \]
\end{enumerate}

\begin{prop:quadrados positivo}
	O quadrado de um número real não nulo é membro de $\mathbb{P}$.
	\[
	    \forall a \in \mathbb{R},\ (\, a \neq 0 \rightarrow a^2 \in \mathbb{P}\,)
	\]
\begin{proof}
    Se $a \in \mathbb{P}$, então $a^2 \in \mathbb{P}$ por fechamento da multiplicação.
    Se $a \notin \mathbb{P}$, então $-a \in \mathbb{P}$ por hipótese e tricotomia. Por
    fechamento, teremos $a^2 \in \mathbb{P}$.
\end{proof}
\end{prop:quadrados positivo}

\begin{def:ordem}
	Dizemos que $a > b$ se, e somente, se, $a-b \in \mathbb{P}$. Ademais, $a < b$ se,
	e somente se, $b > a$.
\end{def:ordem}

\begin{prop:maior que zero positivo}
	Todo real $a > 0$ é membro de $\mathbb{P}$.
\begin{proof}
	Por definição, $a - 0 \in \mathbb{P}$, logo $a \in \mathbb{P}$.
\end{proof}
\end{prop:maior que zero positivo}

\begin{prop:propriedade ordem}
	A relação $<$ é uma relação de ordem total estrita em $\mathbb{R}$.
\begin{proof}
	Devemos demonstrar que a relação satisfaz irreflexividade, totalidade e transitividade.

	1. A relação é irreflexiva, pois $a - a = 0$, e, por tricotomia, $0 \notin \mathbb{P}$.

	2. Para totalidade, suponha que $a\, \cancel{<}\, b$, o que significa que
	$b - a \notin \mathbb{P}$, e, por tricotomia, $a - b \in \mathbb{P}$, logo
	$b < a$. O caso de $b\, \cancel{<}\, a$ é análogo, em que se concluirá que $a < b$.

	3. É verdade que a relação é transitiva. Suponha que $a < b$ e $b < c$, e por definição
	\[
	    b - a \in \mathbb{P} \land c - b \in \mathbb{P}.
	\]
	Visto que
	\[
	    (c - b) + (b - a) = c - a
	\]
	então conclui-se que $c - a \in \mathbb{P}$ por fechamento, e $a < c$.
\end{proof}
\end{prop:propriedade ordem}

\begin{prop:monotonicidade ordem}
	A relação $<$ é monótona sob adição e multiplicação por elemento em $\mathbb{P}$. Logo,
	para quaisquer reais $a$, $b$ e $c$
	\begin{center}
    	\begin{multicols}{2}
    	    $a < b \rightarrow a + c < b + c$

    		$a < b \land c > 0 \rightarrow a \cdot c < b \cdot c$
        \end{multicols}
    \end{center}
\begin{proof}
	Com efeito, $b - a \in \mathbb{P}$, mas tabmém temos que
	\[
	    b - a = b + c - a - c = (b - c) - (a + c)
	\]
	donde se conclui que $a + c < b + c$. Ademais, se $c \in \mathbb{P}$, então
	\[
	    c(b - a) \in \mathbb{P}
	\]
	por fechamento, porém
	\[
        c(b - a) = bc - ac
	\]
	e $ac < bc$.
\end{proof}
\end{prop:monotonicidade ordem}

\begin{cor:mudanca de ordem}
    Para quaisquer reais $x, y$ e $z$
    \[
        a < b \land c < 0 \rightarrow bc < ac
    \]
\begin{proof}
    Por tricotomia, $-c > 0$, e, por fechamento, $-c(b-a) \in \mathbb{P}$. Logo,
    \[
        -c(b-a) = ac - bc
    \]
    implica que $bc < ac$.
\end{proof}
\end{cor:mudanca de ordem}

\subsection{Inclusão dos Naturais e Inteiros}

Na construção dos naturais pelos axiomas de Peano, a adição é definida tal como $0$ é
o seu elemento neutro, o que está de acordo com que enuciáramos, e $S(x+y) = x + S(y)$
(lembrando que $S$ é a função sucessor). Em especial, $S(x) = x + S(0)$, e temos que
$S(0) = 1$, logo, $S(x) = x + 1$. Uma vez que $1 > 0$, então $1 \in \mathbb{P}$. A
partir dessas conclusões, iremos porvar que os reais incluem os naturais e inteiros.

\begin{thm:reais incluem naturais}
	O conjunto dos naturais é subconjunto dos reais.
	\[
	    \mathbb{N} \subset \mathbb{R}
	\]
\begin{proof}
	A prova se dará por indução. Suponha que $\mathfrak{M}$ seja o subconjunto
	dos naturais que estejam contidos nos reais. Temos que $0 \in \mathfrak{M}$,
	pois ele é o elemento neutro da adição nos reais por definição. Suponha que
	$x$ seja um natural membro do conjunto dos reais. Sabemos que $1$ é um número
	real por ser o elemento neutro da multiplicação, e também que $1 \in \mathbb{P}$,
	implicando que $S(x) = x + 1$ deve ser um número real por fechamento,
	logo $\mathfrak{M} = \mathbb{N}$ pelo Princípio de Indução.
\end{proof}
\end{thm:reais incluem naturais}

\begin{thm:reais incluem inteiros}
    O conjunto dos inteiros é subconjunto dos reais.
	\[
	    \mathbb{Z} \subset \mathbb{R}
	\]
\begin{proof}
    Um inteiro $z$ é igual a $z = a - b$ para $a$ e $b$ naturais, logo, reais.
    Como $-b$ é o inverso aditivo de $b$, então $-b$ é um número real, logo
    \[
        z = a-b = a + (-b)
    \]
    é um número real, e os inteiros estão inclusos nos reais.
\end{proof}
\end{thm:reais incluem inteiros}

\begin{thm:reais incluem racionais}
    O conjunto dos racionais é subconjunto dos reais.
   	\[
   	    \mathbb{Q} \subset \mathbb{R}
   	\]
\begin{proof}
	Seja o racional $a = \frac{p}{q}$, com $p$ e $q$ inteiros e $q \neq 0$.
	Já que $p, q \in \mathbb{R}$ e $\frac{1}{q}$ é o inverso multiplicativo de $q$,
	então o $\frac{1}{q}$ deve ser um número real. Como os reais são um corpo, logo
	fechados sob multiplicação, então $\mathbb{Q} \subset \mathbb{R}$.
\end{proof}
\end{thm:reais incluem racionais}

\subsection{Módulo}

\begin{def:modulo}
	O módulo, ou valor absoluto, de um número real $a$ é definido como
	\[
	    |x| = \begin{cases}
					x,&\text{ se } 0 \leq x\\
					-x,&\text{ se } x < 0
					\end{cases}
	\]
\end{def:modulo}

\begin{prop:desigualdade triangular}
	Para quaisquer reais $x$ e $y$ vale que
	\begin{center}
		$| x + y | \leq |x| + |y|$

		$ | xy | = |x| \cdot |y|$
    \end{center}
\begin{proof}
    1. Temos que $|x| \geq x$ e $|y| \geq y$, donde segue que $|x| + |y| \geq x + y$. Da
    mesma forma, $|x| \geq -x$ e $|y| \geq -y$, do qual se conclui que $|x| + |y| \geq -(x+y)$.
    Dos dois casos segue que
    \[
        | x + y | \leq |x| + |y|
    \]

    2. Para a segunda expressão, basta mostrar ambos os membros da igualdade possuem o
    mesmo quadrado, já que são ambos maiores que $0$. Temos que $(|xy|)^2 = x^2y^2$
    independente se $xy$ está em $\mathbb{P}$ ou não. Pelo mesmo argumento, temos que
    \[
        |x|^2 \cdot |y|^2 = x^2y^2
    \]
    do qual segue a igualdade que queríamos mostrar.
\end{proof}
\end{prop:desigualdade triangular}

\begin{prop:desigualdade delta}
	Para quaisquer reais $x$, $a$ e $\delta$, temos que
	\[
	    |x - a| \leq \delta \Longrightarrow a - \delta \leq x \leq a + \delta
	\]
\begin{proof}
	Com efeito, temos que $x-a \leq \delta$ e $-(x-a) \leq \delta$. Da primeira
	desigualdade, segue que
   	\[
        x \leq a + \delta
    \]
    enquanto que, da segunda,
    \begin{gather*}
        -\delta \leq x - a\\
        a - \delta \leq x
    \end{gather*}
\end{proof}
\end{prop:desigualdade delta}

\subsection{A Reta Real}

O conjunto denotado por $\mathbb{R}$ até então equivale ao conjunto dos
números reais positivos, representado por $\mathbb{R}^+$. O complemento
de $\mathbb{R}^+$ menos o conjunto dos reais nulos, ou seja, o conjunto
unitário $\{0\}$, equivale ao conjunto dos reais negativos, representado
por $\mathbb{R}^-$. Dessa forma, temos que
\[
    \mathbb{R} = \mathbb{R}^- \cup \{0\} \cup \mathbb{R}^+.
\]

Iremos admitir também uma notação para definir subconjuntos dos reais,
chamada de intervalos reais, de modo que
\begin{center}
\begin{multicols}{2}
$[a, b] = \{x \in \mathbb{R} \mid a \leq x \leq b\}$

$(a, b] = \{x \in \mathbb{R} \mid a < x \leq b\}$

$[a, b) = \{x \in \mathbb{R} \mid a \leq x < b\}$

$(a, b) = \{x \in \mathbb{R} \mid a < x < b\}$

$(-\infty, b] = \{x \in \mathbb{R} \mid x \leq b\}$

$(-\infty, b) = \{x \in \mathbb{R} \mid x < b\}$

$[a, \infty) = \{x \in \mathbb{R} \mid a \leq x \}$

$(a, \infty) = \{x \in \mathbb{R} \mid a < x \}$
\end{multicols}
\end{center}
Os intervalos que incluem os seus extremos -- o primeiro listado -- são chamados de de
fechados, enquanto que os que não incluem pelo menos um de seus extremos são abertos
(o caso de todos os outros listados).

Uma forma de ilustrar geometricamente os números reais é por meio da
reta real.  Nesta reta vale que, se $a$ e $b$ são reais atribuídos aos pontos $P$ e $Q$
respectivamente e $a < b$, então $P$ está à esquerda de $Q$. Tomando um ponto $O$ que seja
tribuído ao $0$, ponto esse chamado de origem, teremos que todos os pontos à direita de $O$
constituem os reais positivos, ou seja, $\mathbb{R}^+$ ou $(0, \infty)$. Os pontos à esquerda
são os reais negativos: $\mathbb{R^-}$ ou $(-\infty, 0)$. Com os mesmos pontos $P$ e $Q$
de antes, temos que o segmento de reta que tem $P$ e $Q$ inclusos equivale a $[a,b]$. Se
$P$ e $Q$ não forem os extremos, então esse segmento equivale a $(a, b)$.

Todavia, nada nos garante que é possível associar um número real a \textbf{todo} ponto da
reta; noutras palavras, que essa reta não possui  ``buracos''. Algebricamente, isso
equivale ao fato de que nenhumas das propriedades mostradas até então diferencia o conjunto
dos reais dos números racionais. Explicitar que $\mathbb{Q}$ é um subconjunto próprio de
$\mathbb{R}$ será a motivação da próima seção.
