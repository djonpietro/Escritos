\section{Propriedades Algébricas}

\subsection{Propriedades Primitivas}

Assumiremos como primitva a existência de um conjunto $\mathbb{R}$ denominado de conjunto
dos números reais e dotado das operações de adição e multiplicação, as quais satisfazem as
propriedades de corpo algébrico. Isso significa que, para quaisuquer elementos
$a, b, c \in \mathbb{R}$, é suposto que as operações de $+$ e $\cdot$ são fechadas em
$\mathbb{R}$ e satisfazem:

\begin{enumerate}
    \item Associatividade
    \begin{center}
        \begin{multicols}{2}

            $(a + b) + c = a + (b + c)$

            $(a \cdot b) \cdot c = a \cdot (b \cdot c)$
        \end{multicols}
    \end{center}

    \item Comutatividade
    \begin{center}
        \begin{multicols}{2}
            $a + b = b + a$

            $a \cdot b = b \cdot a$
        \end{multicols}
    \end{center}

    \item Existência do Elemento neutro
    \begin{center}
        \begin{multicols}{2}
            $a + 0 = a$

            $a \cdot 1 = a$
        \end{multicols}
    \end{center}

    \item Existência do Elemento Inverso
    \begin{center}
        \begin{multicols}{2}
            $a + (-a) = 0$

            $a \neq 0 \land a \cdot \frac{1}{a} = 1$
        \end{multicols}
    \end{center}

    \item Distribuição da Multiplicação sobre Adição
    \begin{center}
        \begin{multicols}{2}
            $a \cdot (b + c) = ab + ac$

            $a \cdot (b - c) = ab - ac$
        \end{multicols}
    \end{center}

    \item Princípio de Igualdade: se $a = b$, então
    \begin{center}
        \begin{multicols}{2}
            $a + c = b + c$

            $a \cdot c = b \cdot c$

            $a - c = b - c$

            $\displaystyle\frac{a}{c} = \displaystyle\frac{b}{c}$
        \end{multicols}
    \end{center}
\end{enumerate}

Em $\mathbb{R}$ também definiremos os subconjuntos dos naturais $\mathbb{N}$, dos inteiros
$\mathbb{Z}$ e dos racionais $\mathbb{Q}$. Primeiramente, o conjunto $\mathbb{N}$ satisfaz
que:
\begin{enumerate}
	\item $0 \in \mathbb{N}$.

	\item $x \in \mathbb{N} \rightarrow x + 1 \in \mathbb{N}$.
\end{enumerate}
O conjunto dos inteiros é definido como:
\[
    \mathbb{Z} = \{z \in \mathbb{R} \mid \exists x, y \in \mathbb{N},\ z = x - y\}
\]
Por fim, os racionais são dados são definidos de tal modo que:
\[
    \mathbb{Q} = \{a \in \mathbb{R} \mid \exists p, q \in \mathbb{Z},\ a = \frac{p}{q} \land q \neq 0\}
\]
Dado que todo natural $x = x - 0$, então $\mathbb{N} \subset \mathbb{Z}$. Outrossim,
temos que todo inteiro $z = \frac{z}{1}$, de sorte que $\mathbb{Z} \subset \mathbb{Q}$.
Portanto,
\[
    \mathbb{N} \subset \mathbb{Z} \subset \mathbb{Q} \subset \mathbb{R}
\]

\subsection{Consequências das Propriedades}

A seguir, provaremos alguns dos comportamentos esperados para as operações de adição
e multiplicação com números reais. A primeira delas trata-se do $0$ ser um elemento
absorvente na multiplicação.

\begin{prop:mult por 0}
    Para todo real $a$ tem-se $a \cdot 0 = 0$
\begin{proof}
    Presuma que $a \cdot 0 = k$. Com efeito
    \begin{gather*}
        a (0 + 0) = a \cdot 0\\
        k + k = k\\
        k + k - k = k - k\\
        k = 0
    \end{gather*}
    Conclui-se que $a \cdot 0 = 0$
\end{proof}
\end{prop:mult por 0}

A seguir, veremos que um número real multiplicado por $-1$ deve ser igual ao seu oposto.
\begin{prop:mult menos 1}
\label{prop:mult menos 1}
    Para todo real $a$ vale que $a \cdot (-1) = -a$
\begin{proof}
    É verdade que $a \cdot (1 + (-1)) = 0$, donde segue
    \begin{gather*}
        a + a \cdot (-1) = 0\\
        a + a \cdot (-1) - a = 0 - a\\
        a \cdot (-1) = -a
    \end{gather*}
\end{proof}
\end{prop:mult menos 1}

Com o que foi mostrado na proposição \ref{prop:mult menos 1} segue que: (1) o oposto
do oposto de um número real é a identidade desse número, e que somar um número com
seu oposto significa subtrair dele próprio e, consequentemente, a operação de subtração
$a - b$ equivale a $a + (-b)$.
\begin{cor:menos com menos}
	Para todo real $b$, tem-se que $-(-b) = b$
\begin{proof}
    Na proposição \ref{prop:mult menos 1}, substitua $a$ por $-b$, obtendo
    \[
        (-b) \cdot (-1) = -(-b)
    \]
    Mas pelo elemento neutro da adição temos que
    \[
        (-b) + (-(-b)) = 0
    \]
    do que segue
    \begin{gather*}
        b + (-b) + (-(-b)) = b\\
        -(-b) = b
    \end{gather*}
\end{proof}
\end{cor:menos com menos}

\begin{cor:sub e add}
	Para todo real $a$ conclui-se que $a + (-a) = a - a = 0$
\begin{proof}
	\begin{gather*}
        (-1) \cdot a = -a\\
        a + (-1) \cdot a = a-a\\
        a + (-a) = a-a
	\end{gather*}
	Logo, $a-a = 0$
\end{proof}
\end{cor:sub e add}

A seguir, mostraremos a regra de sinais da multiplicação.
\begin{cor:regra de sinais1}
    Para todos reais $a$ e $b$ tem-se $a \cdot (-b) = - (ab)$
\begin{proof}
	Segue da proposição \ref{prop:mult menos 1}, da associatividade e comutatividade da
	multiplicação.
\end{proof}
\end{cor:regra de sinais1}

\begin{cor:regra de sinais2}
	Para quaisquer reais $a$ e $b$ tem-se que $(-a)(-b) = ab$
\begin{proof}
	\begin{align*}
		(-a)(-b) &=\\
		(-1)(-1)(ab)&=\\
		-(-1)(ab)&=ab
	\end{align*}
\end{proof}
\end{cor:regra de sinais2}

Uma propriedade útil em alguns contextos é que, dado que o produto de dois números é
nulo, então pelo menos um dos fatores era nulo.
\begin{prop:fator 0}
\label{prop:fator 0}
	Para quaisquer reais $a$ e $b$, se $ab = 0$, então $a = 0$ ou $b = 0$
\begin{proof}
	Suponha por contradição que $ab = 0$, $a \neq 0$ e $b \neq 0$. Então
	\begin{gather*}
		\frac{ab}{b} = \frac{0}{b}\\
		a = 0
\end{gather*}
    o que é absurdo, logo a proposição vale.
\end{proof}
\end{prop:fator 0}

Da proposição \ref{prop:fator 0} podemos mostrar que quando dois quadrados são iguais,
então as bases são iguais ou opostas.
\begin{cor:quadrados iguais}
    Sejam $a, b \in \mathbb{R}$. Se $a^2 = b^2$, então $a = \pm b$
\begin{proof}
	Uma vez que
	\[
	    (a+b)(a-b) = a^2 - b^2 = 0
	\]
	então
	\[
	   (a+b) = 0 \quad \text{ ou } \quad (a-b) = 0
	\]
	donde se conclui que $a = -b$ ou $a = b$ respectivamente.
\end{proof}
\end{cor:quadrados iguais}
