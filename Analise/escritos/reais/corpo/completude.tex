\section{Completude}

\subsection{Axioma do Supremo}

\begin{def:majorante}
    Um conjunto $X \subset \mathbb{R}$ é dito limitado superiormente caso exista um
    real $b$, chamado de majorante ou cota superior, tal que
    \[
        \forall x \in X, x \leq b.
    \]
    O conjunto $X$ será limitado inferiormente se existe real $a$, denominado minorante
    ou cota inferior, satisfazendo
    \[
        \forall x \in X, a \leq x.
    \]
\end{def:majorante}

\begin{def:supremo}
    Se $X \subset \mathbb{R}$ é um conjunto limitado superiormente, então o seu supremo,
    denotado por $\sup X$, será o elemento que satisfaz
    \begin{enumerate}
        \item $\sup X$ é majorante de $X$.

        \item $\forall a \in \mathbb{R},\ (a \text{ é majorante de } X \rightarrow \sup X \leq a)$
    \end{enumerate}
    Alternativamente, se $X$ é
    limitado inferiormente, então o seu ínfimo, representado por $\inf X$, é o elemento
    em $\mathbb{R}$ que satisfaz
    \begin{enumerate}
        \item $\inf X$ é minorante de $X$.

        \item $\forall a \in \mathbb{R},\ (a \text{ é minorante de } X \rightarrow a \leq \inf X)$
    \end{enumerate}
\end{def:supremo}

Sinteticamente, o supremo é o menor dos majorantes, enquanto que o ínfimo é o maior dos
minorantes. Esses conceitos também são usados para definir formalmente as ideias usuais
de elemento máximo ou mínimo de um conjunto.

\begin{def:maximo}
	Se $X \subset \mathbb{R}$ possui um supremo $a$, ele será o máximo de $X$, denotado
	por $\max X$, se, e somente se, for membro de $X$. Se $b$ for o ínfimo de $X$, então
	ele será o mínimo de $X$, com notação $\min X$, caso também seja elemento de $X$.
\end{def:maximo}

Enunciaremos, finalmente, a propriedade que difere os racionais dos reais. Ela é
conhecida como Axioma da Completude ou do Supremo, o qual define $\mathbb{R}$ como
um corpo ordenado completo.

\begin{ax:supremo}
	Todo subconjunto $X \subset \mathbb{R}$ não vazio e limitado superiormente possui
	um supremo.
\end{ax:supremo}

Com base nesse axioma, vamos provar um resultado análogo ao enunciado: todo conjunto com
uma cota inferior deve ter um ínfimo.

\begin{prop:completude infimo}
    Todo subconjunto $X \subset \mathbb{R}$ não vazio e limitado inferiormente possui
	um ínfimo.
\begin{proof}
	Sendo $X$ um subconjunto dos reais não vazio e com um minorante, digamos que o
	conjunto $X'$ seja aquele tal que:
	\[
	    X' = \{x' \in \mathbb{R} \mid x' = -x \land x \in X \}.
	\]
	Da hipótese segue que $X'$ é não vazio e possui, pois, um supremo $s$ pelo Axioma
	da Completude. Sendo satisfeito que
	\[
	    \forall x' \in X',\ (\, x' \leq s\,)
	\]
	então deve ser válido que
	\[
	    \forall x \in X,\ (\, -s \leq x\,)
	\]
	e $-s$ é um minorante de $X$. Ademais, $-s$ deve ser ínfimo, já que, caso contrário,
	haveria $m \in \mathbb{R}$ tal que $-s < m < x$ para todo $x \in X$, implicando que
	$x' < -m < s$ para todo $x' \in \mathbb{X'}$, e $s$ não seria supremo.
\end{proof}
\end{prop:completude infimo}

\subsection{Propriedade Arquimediana}

A propriedade arquimediana é simples em seu enunciado, mas detém um significado mais
profundo. Em essência, ela enuncia que não existem elementos ``infinitesimais'' ou
``infinitos'' no conjunto dos reais, isto é, elementos infinitamente pequenos ou
grandes, respectivamente, no conjunto dos números reais. Qualquer número real positivo $a$
que tomarmos, por mais pequeno em magnitude que seja, podemos sempre multiplicá-lo por
algum outro número positivo cujo resultado será maior que um outro positivo $b$ dado.
Inversamente, podemos dividir qualquer positivo $b$, por maior que seja, de tal forma que
o quociente será menor do que um segundo positivo $a$ dado.

Há três modos formais mais comuns de enunciar essa proprieade, e iremos mostrar que todas
elas são equivalentes.

\begin{thm:arquimediana}
\label{thm:arquimediana}
	São equivalentes as afirmações:
	\begin{enumerate}
		\item O conjuntos dos naturais não possui supremo.
		\item O ínfimo do conjunto $X = \{1/n \in \mathbb{R} \mid n \in \mathbb{N}\}$ é $0$.
		\item para quaisquer reais positivos $a$ e $b$, existe $n \in \mathbb{N}$ tal que $b < na$.
    \end{enumerate}
\begin{proof}
	(1 $\rightarrow$ 2) Se os naturais não possuem supremo, então, para todo $c \in \mathbb{R}^+$,
	temos que $\frac{1}{c}$ não pode ser majorante de $\mathbb{N}$. Existe, desse modo,
	um $n \in \mathbb{N}$ tal que
	\begin{gather*}
		\frac{1}{c} < n\\
		\frac{1}{n} < c.
    \end{gather*}
    Como consequência dos fatos que $X$ tem $0$ como minorante e nenhum real positivo pode
    ser minorante de $X$, então $\inf X = 0$.

    (2 $\rightarrow$ 3) Suponha (2) e, por contradição, que existem $a$ e $b$ reais
    positivos tal que, para todo natural $n$, tenha-se $na < b$. Dessa desigualdade obtemos
    que
    \[
        \frac{a}{b} < \frac{1}{n}.
    \]
    isso implicaria que o lado esquerdo da desigualdade seria um minorante de $X$. Contudo,
    pela hipótse, teríamos $\frac{a}{b} < 0$, o que contradiz o fechamento dos reais
    positivos, provando que da afirmação 2 segue a terceira.

    (3 $\rightarrow$ 1) Suponha (3) e que os naturais possuam um supremo $s$. Seria
    verdade então que
    \[
        \forall m \in \mathbb{N},\ (\,m < s\,)
    \]
    No entanto, da hipótse segue que
    \[
        \exists n \in \mathbb{N},\ (\, s < nm\,)
    \]
    e $m < s < nm$. Como $n \cdot m \in \mathbb{N}$, $s$ não é supremo,
    chegando a uma contradição.
\end{proof}
\end{thm:arquimediana}

\begin{thm:majorante naturais}
\label{thm:majorante naturais}
	O conjunto $\mathbb{N}$ não possui um supremo.
\begin{proof}
	Suponha por contradição que $s = \sup \mathbb{N}$. Com efeito, $s - 1$ não será
	supremo de $\mathbb{N}$, o que equivale a dizer que
	\[
	    \exists n \in \mathbb{N},\ (\,s - 1 < n\,).
	\]
	o que nos leva a $s < n + 1$, mas o lado direito da desigualdade é um natural,
	contradizendo a suposição de que $s$ é o supremo dos naturais.
\end{proof}
\end{thm:majorante naturais}

Sendo o Teorema \ref{thm:majorante naturais} válido e equivalente as demais afirmações no
teorema \ref{thm:arquimediana}, então todas as afirmações são válidas. Concluimos que os
números reais são um conjunto arquimediano.

\subsection{Outras Consequências do Axioma da Completude}

\begin{thm:intervalos encaixados}
\label{thm:intervalos encaixados}
    Uma sequência descrescente de intervalos reais $I_1, I_2, \ldots$ encaixados, isto é,
   \[
        I_1 \supset I_2 \supset \ldots,
   \]
   possui pelo menos um $c \in \mathbb{R}$ satisfazendo
   \[
        c \in \bigcap_{k = 1}^\infty I_k
   \]
\begin{proof}
    Seja $I_k = [a_k, b_k]$, temos que
    \[
        a_1 \leq a_2 \leq \ldots \leq a_n \leq \ldots \leq b_n \leq \ldots \leq b_2 \leq b_1.
    \]
    O conjunto dos extremos inferiores $A = \{a_1, a_2, \ldots\}$, pelo Axioma de Completude,
    possui um supremo $c$. Esse supremo, por definição, será menor que qualquer majorante
    $b_k$, para $k \in \mathbb{N}^*$. Por consequência, $c \in I_k$ para todo $k \in \mathbb{N}$.
\end{proof}
\end{thm:intervalos encaixados}

\begin{thm:intervalos encaixados e funcao}
\label{thm:não enumeravel}
    Toda função $f: \mathbb{N} \to (a, b)$ admite uma sequência decrescente
    de intervalos encaixados satisfazendo $f(k) \notin I_k$.
\begin{proof}
    Para isso, começe fazendo $I_0 = [a_0, b_0]$ tal que
    \[
        a < a_0 < b_0 < b
    \]
    e $f(0) < a_0$. Depois, com o intervalo $I = [a_k, b_k]$ construído,
    para $k \in \mathbb{N}^*$, construa $I_{k+1}$ de tal modo que, se $f(k+1) \notin I_n$,
    então $I_{n+1}$, senão, $f(n+1)$ terá de ser diferente de pelo meno um dos extremos de
    $I_k$, digamos $a_k$. Desse modo, como $a_k < f(k+1)$, faça $a_{k+1} = a_k$ e
    \[
        b_{k+1} = \frac{a_k + f(n+1)}{2}.
    \]
\end{proof}
\end{thm:intervalos encaixados e funcao}

\begin{thm:não enumeravel}
    Todo intervalo real $(a, b)$, com $a, b \in \mathbb{R}$, é não enumerável.
\begin{proof}
    Provemos que uma função $f: \mathbb{N} \to (a, b)$ não pode ser sobrejetiva.
    Com efeito, essa função admite uma sequência decrescente de intervalos encaixados
    \[
        I_0 \supset I_1 \supset \ldots
    \]
    satisfazendo $f(k) \notin I_k$. Pelo Teorema \ref{thm:intervalos encaixados},
    existe $c \in \mathbb{R}$ que pertence a todos esses intervalos e, por construção,
    não poderia ser imagem de nenhum $n \in \mathbb{N}$. Logo, toda função $f$ de naturais
    para um intervalo real não pode ser sobrejetora.
\end{proof}
\end{thm:não enumeravel}

\begin{cor:reais nao enumeraveis}
    O conjunto dos números reais é não enumerável.
\begin{proof}
	Caso fosse, todo subconjunto seu seria enumerável, o que é absurdo, pois itervalos
	não o são.
\end{proof}
\end{cor:reais nao enumeraveis}

Sendo o conjunto $\mathbb{Q}$ dos números racionais enumerável e $\mathbb{R}$ não
enumerável, então há um número infinito de elementos, infinito esse maior que dos
números naturais, que não são racionais. Esses números são chamados de
irracionais, pois não podem ser escritos como uma fração com numerador e denominador
inteiros, e seu conjunto é denotado por $\mathbb{I}$. Portanto, conseguimos mostrar que o
axioma do supremo garante que os números racionais é subconjunto próprio dos números reais
