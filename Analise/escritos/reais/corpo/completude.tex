\section{Completude}

\subsection{Supremo e Propriedade Arquimediana}

\begin{def:majorante}
    Um conjunto $X \subset \mathbb{R}$ é dito limitado superiormente caso exista um
    real $b$, chamado de majorante ou cota superior, tal que
    \[
        \forall x \in X, x \leq b.
    \]
    O conjunto $X$ será limitado inferiormente se existe real $a$, denominado minorante
    ou cota inferior, satisfazendo
    \[
        \forall x \in X, a \leq x.
    \]
\end{def:majorante}

\begin{def:supremo}
    Se $X \subset \mathbb{R}$ é um conjunto limitado superiormente, então o seu supremo,
    denotado por $\sup X$, é o menor dos seus majorantes. Alternativamente, se $X$ é
    limitado inferiormente, então o seu ínfimo, representado por $\inf X$, é o maior
    dos seus minorantes.
\end{def:supremo}

Evidentemente, todo subconjunto dos reais limitado superiormente possui um supremo.
O mesmo vale se ele for limitado inferiormente: terá um ínfimo.

\begin{def:maximo}
	Se $X \subset \mathbb{R}$ possui um supremo $a$, ele será o máximo de $X$, denotado
	por $\max X$, se, e somente se, for membro de $X$. Se $b$ for o ínfimo de $X$, então
	ele será o mínimo de $X$, com notação $\min X$, caso também seja elemento de $X$.
\end{def:maximo}

\begin{thm:arquimediana}
\label{thm:arquimediana}
	São equivalentes as afirmações:
	\begin{enumerate}
		\item O conjuntos dos naturais não possui supremo.
		\item O ínfimo do conjunto $X = \{1/n \in \mathbb{R} \mid n \in \mathbb{N}\}$ é $0$.
		\item para quaisquer reais positivos $a$ e $b$, existe $n \in \mathbb{N}$ tal que $b < na$.
    \end{enumerate}
\begin{proof}
	(1 $\rightarrow$ 2) Suponha a afirmação (1).
	Visto que os reais positivos são fechados por multiplicação, então se
	$x \in X$, segue que $x > 0$, e $0$ é minorante de $X$. Seja $c$ um real positivo,
	então por hipótese existe um natural $n$ tal que
	\begin{gather*}
		\frac{1}{c} < n\\
		\frac{1}{n} < c
    \end{gather*}
    do qual se conclui que $0$ deverá ser o ínfimo de $X$.

    (2 $\rightarrow$ 3) Suponha (2) e, por contradição, que existem $a$ e $b$ reais
    positivos tal que, para todo natural $n$, tenha-se $na < b$. Dessa desigualdade obtemos
    que
    \[
        \frac{a}{b} < \frac{1}{n}.
    \]
    isso implicaria que o lado esquerdo da desigualdade seria um minorante de $X$ e, pelo
    fechamento dos postivos, o ínfimo de $X$, que contradiz a hipótese, logo um absurdo.

    (3 $\rightarrow$ 1) Suponha (3) e que $m \in \mathbb{N}$ e $s \in \mathbb{R}^+$ tal
    que $m < s$. Uma vez que são ambos positvos, então existe natural $n$ tal que
    $n \cdot m < s$. Como $n \cdot m \in \mathbb{N}$, então os naturais não podem ter
    um supremo.
\end{proof}
\end{thm:arquimediana}

\begin{thm:majorante naturais}
\label{thm:majorante naturais}
	O conjunto $\mathbb{N}$ não possui um supremo.
\begin{proof}
	Suponha por contradição que $s = \sup \mathbb{N}$. Com efeito, $s - 1$ não será
	supremo de $\mathbb{N}$, o que equivale a dizer que
	\[
	    \exists n \in \mathbb{N},\ (\,s - 1 < n\,).
	\]
	o que nos leva a $s < n + 1$, mas o lado direito da desigualdade é um natural,
	contradizendo a suposição de que $s$ é o supremo dos naturais.
\end{proof}
\end{thm:majorante naturais}

Sendo o Teorema \ref{thm:majorante naturais} válido e equivalente as demais afirmações no
teorema \ref{thm:arquimediana}, então todas as afirmações são válidas. Em especial, a
terceira afirmação do Teorema \ref{thm:arquimediana} é dita como propriedade arquimediana,
e, por consequência, o conjunto dos reais é arquimediano.
